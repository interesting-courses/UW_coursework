\documentclass[10pt]{article}
\usepackage[T1]{fontenc}

% Document Details
\newcommand{\CLASS}{AMATH 584}
\newcommand{\assigmentnum}{Assignment 1}

\usepackage[margin = 1.15in, top = 1.25in, bottom = 1.in]{geometry}

\usepackage{titling}
\setlength{\droptitle}{-6em}   % This is your set screw
\date{}
\renewcommand{\maketitle}{
	\clearpage
	\begingroup  
	\centering
	\LARGE \sffamily\textbf{\CLASS} \Large \assigmentnum\\[.8em]
	\large Tyler Chen\\[1em]
	\endgroup
	\thispagestyle{empty}
}
 % Title Styling
\usepackage{tocloft}
\renewcommand{\cfttoctitlefont}{\Large\sffamily\bfseries}
\renewcommand{\cftsecfont}{\normalfont\sffamily\bfseries}
\renewcommand{\cftsubsecfont}{\normalfont\sffamily}
\renewcommand{\cftsubsubsecfont}{\normalfont\sffamily}

\makeatletter
\let\oldl@section\l@section
\def\l@section#1#2{\oldl@section{#1}{\sffamily\bfseries#2}}

\let\oldl@subsection\l@subsection
\def\l@subsection#1#2{\oldl@subsection{#1}{\sffamily#2}}

\let\oldl@subsubsection\l@subsubsection
\def\l@subsubsection#1#2{\oldl@subsubsection{#1}{\sffamily#2}}
 % General Styling


\usepackage{enumitem}

% Figures
\usepackage{subcaption}

% TikZ and Graphics
\usepackage{tikz, pgfplots}
\pgfplotsset{compat=1.12}
\usetikzlibrary{patterns,arrows}
\usepgfplotslibrary{fillbetween}

\usepackage{pdfpages}
\usepackage{adjustbox}

\usepackage{lscape}
\usepackage{titling}
\usepackage[]{hyperref}


% Header Styling
\usepackage{fancyhdr}
\pagestyle{fancy}
\lhead{\sffamily \CLASS}
\rhead{\sffamily Chen \textbf{\thepage}}
\cfoot{}

% Paragraph Styling
\setlength{\columnsep}{1cm}
\setlength{\parindent}{0pt}
\setlength{\parskip}{5pt}
\renewcommand{\baselinestretch}{1}

% TOC Styling
\usepackage{tocloft}
\iffalse
\renewcommand{\cftsecleader}{\cftdotfill{\cftdotsep}}

\renewcommand\cftchapafterpnum{\vskip6pt}
\renewcommand\cftsecafterpnum{\vskip10pt}
\renewcommand\cftsubsecafterpnum{\vskip6pt}

% Adjust sectional unit title fonts in ToC
\renewcommand{\cftchapfont}{\sffamily}
\renewcommand{\cftsecfont}{\bfseries\sffamily}
\renewcommand{\cftsecnumwidth}{2em}
\renewcommand{\cftsubsecfont}{\sffamily}
\renewcommand{\cfttoctitlefont}{\hfill\bfseries\sffamily\MakeUppercase}
\renewcommand{\cftaftertoctitle}{\hfill}

\renewcommand{\cftchappagefont}{\sffamily}
\renewcommand{\cftsecpagefont}{\bfseries\sffamily}
\renewcommand{\cftsubsecpagefont}{\sffamily}
\fi
 % General Styling
% Code Display Setup
\usepackage{listings,lstautogobble}
\usepackage{lipsum}
\usepackage{courier}
\usepackage{catchfilebetweentags}

\lstset{
	basicstyle=\small\ttfamily,
	breaklines=true, 
	frame = single,
	rangeprefix=,
	rangesuffix=,
	includerangemarker=false,
	autogobble = true
}


\usepackage{algorithmicx}
\usepackage{algpseudocode}

\newcommand{\To}{\textbf{to}~}
\newcommand{\DownTo}{\textbf{downto}~}
\renewcommand{\algorithmicdo}{\hspace{-.2em}\textbf{:}}
 % Code Display Setup
% AMS MATH Styling
\usepackage{amsmath, amssymb}
\newcommand{\qed}{\hfill\(\square\)}

%\newtheorem*{lemma}{Lemma} 
%\newtheorem*{theorem}{Theorem}
%\newtheorem*{definition}{Definition}
%\newtheorem*{prop}{Proposition}
%\renewenvironment{proof}{{\bfseries Proof.}}{}


% mathcal
\newcommand{\cA}{\ensuremath{\mathcal{A}}}
\newcommand{\cB}{\ensuremath{\mathcal{B}}}
\newcommand{\cC}{\ensuremath{\mathcal{C}}}
\newcommand{\cD}{\ensuremath{\mathcal{D}}}
\newcommand{\cE}{\ensuremath{\mathcal{E}}}
\newcommand{\cF}{\ensuremath{\mathcal{F}}}
\newcommand{\cG}{\ensuremath{\mathcal{G}}}
\newcommand{\cH}{\ensuremath{\mathcal{H}}}
\newcommand{\cI}{\ensuremath{\mathcal{I}}}
\newcommand{\cJ}{\ensuremath{\mathcal{J}}}
\newcommand{\cK}{\ensuremath{\mathcal{K}}}
\newcommand{\cL}{\ensuremath{\mathcal{L}}}
\newcommand{\cM}{\ensuremath{\mathcal{M}}}
\newcommand{\cN}{\ensuremath{\mathcal{N}}}
\newcommand{\cO}{\ensuremath{\mathcal{O}}}
\newcommand{\cP}{\ensuremath{\mathcal{P}}}
\newcommand{\cQ}{\ensuremath{\mathcal{Q}}}
\newcommand{\cR}{\ensuremath{\mathcal{R}}}
\newcommand{\cS}{\ensuremath{\mathcal{S}}}
\newcommand{\cT}{\ensuremath{\mathcal{T}}}
\newcommand{\cU}{\ensuremath{\mathcal{U}}}
\newcommand{\cV}{\ensuremath{\mathcal{V}}}
\newcommand{\cW}{\ensuremath{\mathcal{W}}}
\newcommand{\cX}{\ensuremath{\mathcal{X}}}
\newcommand{\cY}{\ensuremath{\mathcal{Y}}}
\newcommand{\cZ}{\ensuremath{\mathcal{Z}}}

% mathbb
\usepackage{bbm}
\newcommand{\bOne}{\ensuremath{\mathbbm{1}}}

\newcommand{\bA}{\ensuremath{\mathbb{A}}}
\newcommand{\bB}{\ensuremath{\mathbb{B}}}
\newcommand{\bC}{\ensuremath{\mathbb{C}}}
\newcommand{\bD}{\ensuremath{\mathbb{D}}}
\newcommand{\bE}{\ensuremath{\mathbb{E}}}
\newcommand{\bF}{\ensuremath{\mathbb{F}}}
\newcommand{\bG}{\ensuremath{\mathbb{G}}}
\newcommand{\bH}{\ensuremath{\mathbb{H}}}
\newcommand{\bI}{\ensuremath{\mathbb{I}}}
\newcommand{\bJ}{\ensuremath{\mathbb{J}}}
\newcommand{\bK}{\ensuremath{\mathbb{K}}}
\newcommand{\bL}{\ensuremath{\mathbb{L}}}
\newcommand{\bM}{\ensuremath{\mathbb{M}}}
\newcommand{\bN}{\ensuremath{\mathbb{N}}}
\newcommand{\bO}{\ensuremath{\mathbb{O}}}
\newcommand{\bP}{\ensuremath{\mathbb{P}}}
\newcommand{\bQ}{\ensuremath{\mathbb{Q}}}
\newcommand{\bR}{\ensuremath{\mathbb{R}}}
\newcommand{\bS}{\ensuremath{\mathbb{S}}}
\newcommand{\bT}{\ensuremath{\mathbb{T}}}
\newcommand{\bU}{\ensuremath{\mathbb{U}}}
\newcommand{\bV}{\ensuremath{\mathbb{V}}}
\newcommand{\bW}{\ensuremath{\mathbb{W}}}
\newcommand{\bX}{\ensuremath{\mathbb{X}}}
\newcommand{\bY}{\ensuremath{\mathbb{Y}}}
\newcommand{\bZ}{\ensuremath{\mathbb{Z}}}

% alternative mathbb
\newcommand{\NN}{\ensuremath{\mathbb{N}}}
\newcommand{\RR}{\ensuremath{\mathbb{R}}}
\newcommand{\CC}{\ensuremath{\mathbb{C}}}
\newcommand{\ZZ}{\ensuremath{\mathbb{Z}}}
\newcommand{\EE}{\ensuremath{\mathbb{E}}}
\newcommand{\PP}{\ensuremath{\mathbb{P}}}
\newcommand{\VV}{\ensuremath{\mathbb{V}}}
\newcommand{\cov}{\ensuremath{\text{Co}\VV}}
% Math Commands

\newcommand{\st}{~\big|~}
\newcommand{\stt}{\text{ st. }}
\newcommand{\ift}{\text{ if }}
\newcommand{\thent}{\text{ then }}
\newcommand{\owt}{\text{ otherwise }}

\newcommand{\norm}[1]{\left\lVert#1\right\rVert}
\newcommand{\snorm}[1]{\lVert#1\rVert}
\newcommand{\ip}[1]{\ensuremath{\left\langle #1 \right\rangle}}
\newcommand{\pp}[3][]{\frac{\partial^{#1}#2}{\partial #3^{#1}}}
\newcommand{\dd}[3][]{\frac{\d^{#1}#2}{\d #3^{#1}}}
\renewcommand{\d}{\ensuremath{\mathrm{d}}}

\newcommand{\indep}{\rotatebox[origin=c]{90}{$\models$}}




 % Math shortcuts
% Problem
\usepackage{floatrow}

\newenvironment{problem}[1][]
{\pagebreak
\noindent\rule{\textwidth}{1pt}\vspace{0.25em}
{\sffamily \textbf{#1}}
\par
}
{\par\vspace{-0.5em}\noindent\rule{\textwidth}{1pt}}

\newenvironment{solution}[1][]
{{\sffamily \textbf{#1}}
\par
}
{}

 % Problem Environment


\hypersetup{
   colorlinks=true,       % false: boxed links; true: colored links
   linkcolor=violet,          % color of internal links (change box color with linkbordercolor)
   citecolor=green,        % color of links to bibliography
   filecolor=magenta,      % color of file links
   urlcolor=cyan           % color of external links
}

\usepackage{arydshln}

\begin{document}
\maketitle

\begin{problem}[Exercise 1.1]
Let \( B \) be a \( 4\times 4 \) matrix to which we apply the following operations:
\begin{enumerate}[topsep=0pt,itemsep=-1ex,partopsep=1ex,parsep=1ex]
	\item double column 1,
	\item halve row 3,
	\item add row 3 to to row 1,
	\item interchange columns 1 and 4,
	\item subtract row 2 from each of the other rows,
	\item replace column 4 by column 3,
	\item delete column 1 (so that the column dimension is reduced by 1).
\end{enumerate}
\begin{enumerate}
	\item[(a)] Write the result as a product of eight matrices .
	\item[(b)] Write it again as a product \( ABC \) (same \( B \)) of three matrices.
\end{enumerate}
\end{problem}

\begin{solution}[Solution]
\begin{enumerate}
	\item[(a)] We have, \( O_5O_3 O_2 B O_1 O_4O_6 \) where,
	\begin{align*}
	O_1 = \left[\begin{array}{cccc}2&0&0&0 \\ 0&1&0&0 \\ 0&0&1&0 \\ 0&0&0&1\end{array}\right] &&
	O_2 = \left[\begin{array}{cccc}1&0&0&0 \\ 0&1&0&0 \\ 0&0&1/2&0 \\ 0&0&0&1\end{array}\right] &&
	O_3 = \left[\begin{array}{cccc}1&0&1&0 \\ 0&1&0&0 \\ 0&0&1&0 \\ 0&0&0&1\end{array}\right] &&
	O_4 = \left[\begin{array}{cccc}0&0&0&1 \\ 0&1&0&0 \\ 0&0&1&0 \\ 1&0&0&0\end{array}\right] \\
	O_5 = \left[\begin{array}{cccc}1&-1&0&0 \\ 0&1&0&0 \\ 1&-1&1&0 \\ 0&-1&0&1\end{array}\right] &&
	O_6 = \left[\begin{array}{cccc}1&0&0&0 \\ 0&1&0&0 \\ 0&0&1&1 \\ 0&0&0&0\end{array}\right] &&
	O_7 = \left[\begin{array}{ccc}0&0&0 \\ 1&0&0 \\ 0&1&0 \\ 0&0&1\end{array}\right]
	\end{align*}
	\item[(b)] We now simply the expression from (a) as, \( ABC\) where,
	\begin{align*}
	A = O_5O_3O_2= \left[\begin{array}{cccc}1&-1&1/2&0 \\ 0&1&0&0 \\ 0&-1&1/2&0 \\ 0&-1&0&1\end{array}\right] && 
	C = O_1O_4O_6O_7 =\left[\begin{array}{cccc}0&0&0&0 \\ 0&1&0&0 \\ 0&0&1&1 \\ 0&0&0&0\end{array}\right]
	\end{align*}
\end{enumerate}
	
We first manually manipulate the inputed matrix. We then define the matrices listed above. Finally, all three methods are compared. 
\lstinputlisting{hw1.py}

Running the function for a few different values of \( B \) always returns \verb|True| indicating that the three methods are equivalent (at least for the tested matrices). A sample output is displayed below.
\begin{lstlisting}
>> exercise_1_1(sp.matrix(sp.random.rand(4,4)))
>> [[ 0.         -0.07326807  0.33590766  0.33590766]
[ 0.          0.91030668  0.63417526  0.63417526]
[ 0.         -0.46052944 -0.4908797  -0.4908797 ]
[ 0.         -0.28526664 -0.29515107 -0.29515107]]
[[ 0.         -0.07326807  0.33590766  0.33590766]
[ 0.          0.91030668  0.63417526  0.63417526]
[ 0.         -0.46052944 -0.4908797  -0.4908797 ]
[ 0.         -0.28526664 -0.29515107 -0.29515107]]
[[ 0.         -0.07326807  0.33590766  0.33590766]
[ 0.          0.91030668  0.63417526  0.63417526]
[ 0.         -0.46052944 -0.4908797  -0.4908797 ]
[ 0.         -0.28526664 -0.29515107 -0.29515107]]
True
\end{lstlisting}
\end{solution}
	
\begin{problem}[Exercise 2.1]
Show that if a matrix \( A \) is both triangular and unitary, then it is diagonal.
\end{problem}

\begin{solution}[Solution]
Suppose a matrix \( A\in\CC^{m\times m} \), \( m\geq2 \), is both triangular and unitary. We have \( A^*A=I=AA^* \), so one of \( A \) or \( A^* \) is upper triangular. Thus, without loss of generality assume \( A \) is upper triangular. 

Since \( A \) is upper triangular we have \( A_{ij}=0 \) for \( i>j \).

Consider the product \( AA^*=I \). We then have,
\[ 1= I_{mm} = \sum_{i=1}^{m}A_{mi}A^*_{im} = A_{mm}A^*_{mm} + \sum_{i=1}^{m-1}A_{mi}A^*_{im} = A_{mm}A^*_{mm} \]
Note that this condition implies \( A_{mm} \neq 0 \).

Now observe for any index \( 1\leq j\leq m-1 \),
\[ 0 = I_{jm} = \sum_{i=1}^{m}A_{mi}A^*_{ij} = A_{mm}A^*_{mj} + \sum_{i=1}^{m-1}A_{mi}A^*_{ij} =  A_{mm}A^*_{mj}  \]
Since \( A_{mm}\neq 0 \) we have \( A^*_{mj}=0 \). Therefore \( A_{jm}=\overline{A^*_{mj}}=0 \).

This proves that the last column of \( A \) is zero, except the diagonal entry.

Consider the \( k \)-th order leading principal sub matrix \( A_k \) formed by deleting the last \( m-k \) rows. That is the sub matrix with entries \( A_{ij} \) for \( 1\leq i,j\leq k \). This is displayed below as the top left corner of \( A \)

\begin{align*} 
    A  = 
    \left[\begin{array}{ccc;{2pt/2pt}cc} 
        A_{11} &\cdots & A_{1k} & \\
        \vdots &  & \vdots & & \\ 
        A_{k1} & \cdots & A_{kk} & & \\ 
        \hdashline[2pt/2pt] \\
        & & & & \\
    \end{array}\right] 
        && 
        A_k = 
        \left[\begin{array}{ccc} 
        A_{11} &\cdots & A_{1k} \\ 
        \vdots &  & \vdots  \\ 
        A_{k1} & \cdots & A_{kk}  
        \end{array}\right]
\end{align*}

Clearly \( A_k \) inherits (upper) triangular from \( A \) as \( A_{ij}=0 \) for \( i>j \). Moreover, considering block matrix multiplication we see \( A_kA_k^*= I_k \), where \( I_k \) is the identity matrix in \( \CC^{k\times k} \). That is, \( A_k \) is also unitary (in \( \CC^{k\times k} \)).

Therefore, by the above result, \( A_{jk}=0 \) for any index \( 1\leq j \leq k-1 \). But \( k \) can be any index \( 1\leq k\leq m \) so we see that \( A_{jk}=0 \) for all \( j<k \). That is, \( A \) is lower triangular. By hypothesis \( A \) us upper triangular as well. This proves \( A \) is diagonal. \qed

\end{solution}
	
\begin{problem}[Exercise 2.2]
The Pythagorean theorem asserts that for a set of \( n \) orthogonal vectors \( \{x_i\} \),
\[ \norm{\sum_{i=1}^{n}x_i}^2 = \sum_{i=1}^{n}\norm{x_i}^2 \]
\begin{enumerate}
	\item[(a)] Prove this in the case \( n=2 \) by explicit computation of \( \norm{x_1+x_2}^2 \).
	\item[(a)] Show that this computation also establishes the general case, by induction
\end{enumerate}
\end{problem}

\begin{solution}[Solution]
We make the assumption that \( x_i\in\CC^m \) for \( m \in \ZZ \). Suppose the \( x_i \) are orthogonal. That is, \( x_i^*x_j = 0 \) for \( i\neq j \). We denote the \( k \)-th component of \( x_i \) by \( x_{ik} \).
\begin{enumerate}
	\item[(a)] By orthogonality we have, \( x_1^*x_2 = x_2^*x_1 = 0 \). Thus, 
	\begin{align*}
		\norm{x_1+x_2}^2=(x_1+x_2)^*(x_1+x_2) &= (x_1^*+x_2^*)(x_1+x_2) \\
		&\hspace{-3em}=x_1^*x_1+x_1^*x_2+x_2^*x_1+x_2^*x_2 \\
		&=\norm{x_1}^2+0+0+\norm{x_2}^2 = \norm{x_1}^2+\norm{x_2}^2 \tag*{\qed}
	\end{align*}
	
	\item[(b)] Suppose \( \norm{\sum_{i=1}^{n-1}x_i}^2 = \sum_{i=1}^{n-1}\norm{x_i}^2 \) for some \( n \). Then, using the above result,
	\[ \norm{\sum_{i=1}^{n}x_i}^2 = \norm{x_n+\sum_{i=1}^{n-1}x_i}^2 = \norm{x_n}^2+\norm{\sum_{i=1}^{n-1}x_i}^2 = \norm{x_n}^2+\sum_{i=1}^{n-1}\norm{x_i}^2 = \sum_{i=1}^{n}\norm{x_i}^2 \]
	
	Thus, using the result from (a) as the base step for induction, for all integer \( n\geq1 \), we have, \[ \norm{\sum_{i=1}^{n}x_i}^2 = \sum_{i=1}^{n}\norm{x_i}^2  \tag*{\qed}\]
	
\end{enumerate}
\end{solution}
	
\begin{problem}[Exercise 2.3]
Let \( A\in\CC^{m\times m } \) be hermitian. An eigenvector of \( A \) is a nonzero vector \( x\in\CC^m \) such that \( Ax=\lambda x \) for some \( \lambda\in\CC \), the corresponding eigenvalue.
\begin{enumerate}
	\item[(a)] Prove that all the eigenvalues of \( A \) are real.
	\item[(b)] Prove that if \( x \) and \( y \) are eigenvectors corresponding to distinct eigenvalues, then \( x \) and \( y \) are orthogonal.
\end{enumerate}
\end{problem}

\begin{solution}[Solution]
Let \( A\in\CC^{m\times m} \) be hermitian. That is, \( A=A^* \).
\begin{enumerate}
\item[(a)] Suppose \( x \) is an eigenvector of \( A \) with corresponding eigenvalue \( \lambda \). Then \( Ax=\lambda x \). Recalling that for scalar \( c \), vectors \( u,v \) and matrices \( A,B \) that \( u^*cv = cu^*v \), that \( (cA)^* = \overline{c}A^* \), and that \( (AB)^* = B^*A^* \) we have the following chain of equalities,
\[ \lambda\norm{x}^2 = \lambda x^*x = x^*\lambda x = x^*Ax = x^*A^*x = (x^*Ax)^* = (x^*\lambda x)^* = x^*\overline{\lambda}x = \overline{\lambda}x^*x= \overline{\lambda}\norm{x}^2 \]

Since \( x \) is an eigenvector, \( x \) is nonzero. Thus, \( \norm{x}> 0 \). In particular, this means that \( \norm{x}^2\neq0 \). Thus \( \lambda = \overline{\lambda} \) proving \( \lambda \) is real. \qed


\item[(b)] Suppose  \( y \) is an eigenvector of \( A \) with corresponding eigenvalue \( \gamma \neq \lambda \). Recall from (a) that \( \lambda = \overline{\lambda} \). This gives the following chain of equalities,
\[ \gamma x^*y = x^*\gamma y = x^*Ay = x^*A^*y = (y^*Ax)^* = (y^*\lambda x)^* = x^*\overline{\lambda}y = x^*\lambda y = \lambda x^*y \]

Therefore, \( \gamma x^*y = \lambda x^*y \) so, \[ 0 = \lambda (x^*y) - \gamma (x^*y)  = (\lambda-\gamma)(x^*y) \]

However, since \( \lambda\neq \gamma \), then \( (\lambda-\gamma)\neq 0  \). This proves \( x^*y = 0 \). That is, that \( x \) and \( y \) are orthogonal. \qed
\end{enumerate}
\end{solution}
	
\begin{problem}[Exercise 3.2]
Let \( \norm{\cdot} \) denote any norm on \( \CC^m \) and also the induced matrix norm on \( \CC^{m\times m} \). Show that \( \rho(A)\leq\norm{A} \), where \( \rho(A) \) is the spectral radius of \( A \), i.e., the largest absolute value \( |\lambda| \) of an eigenvalue \( \lambda \) of \( A \).
\end{problem}

\begin{solution}[Solution]
Let \( \norm{\cdot} \) denote any norm on \( \CC^m \) and also the induced matrix norm on \( \CC^{m\times m} \). Denote the largest absolute value eigenvalue of \( A \) by \( \lambda \) and let \( x \) be the corresponding eigenvector. Then, by definition of supremum,
\[ \rho(A) = |\lambda| = \dfrac{|\lambda|\norm{x}}{\norm{x}} = \dfrac{\norm{\lambda x}}{\norm{x}} = \dfrac{\norm{Ax}}{\norm{x}} \leq \sup_{z\neq0}\dfrac{\norm{Az}}{\norm{z}} = \norm{A} \tag*{\qed} \]
\end{solution}
	
\begin{problem}[Exercise 3.3]
Vector and matrix \( p \)-norms are related by various inequalities, often involving the dimensions \( m \) or \( n \). For each of the following, verify the inequality and give and example of a nonzero vector or matrix (for general \( m,n \)) for which equality is achieved. In this problem \( x \) is an \( m \)-vector and \( A \) is an \( m\times n \) matrix.
\begin{enumerate}
	\item[(a)] \( \norm{x}_\infty\leq\norm{x}_2 \),
	\item[(b)] \( \norm{x}_2\leq\sqrt{m}\norm{x}_\infty \),
	\item[(c)] \( \norm{A}_\infty\leq\sqrt{n}\norm{A}_2 \),
	\item[(d)] \( \norm{A}_2\leq\sqrt{m}\norm{A}_\infty \),
\end{enumerate}
\end{problem}

\begin{solution}[Solution]
Let \( x\in\CC^m \). Clearly \( |x_i|\leq \max_{1\leq i\leq m}|x_i|=\norm{x}_\infty \) for all \( 1\leq i \leq m \).
\begin{enumerate}
	\item[(a)]  Let \( j \) be an index such that \( |x_j|=\norm{x}_\infty \). Then,
	\[ \norm{x}_\infty = |x_j| = \left(|x_j|^2\right)^{1/2} \leq \left(|x_j|^2+\sum_{i\neq j}|x_i|^2\right)^{1/2} \leq \left(\sum_{i=1}^{m}|x_i|^2\right)^{1/2} = \norm{x}_2 \]
	Equality is obtained when \( x \) has exactly one nonzero component \( x_i \), in which case \( \norm{x}_\infty = x_i = (|x_i|^2)^{1/2} = \norm{x}_2 \).
	
	\item[(b)] Similarly,
	\begin{align*} \norm{x}_2 = \left(\sum_{i=1}^{m}|x_i|^2\right)^{1/2}  &\leq \left( \sum_{i=1}^{m} \left(\max_{1\leq i\leq m}|x_i|\right)^2 \right)^{1/2} \\ &= \left( m\left(\max_{1\leq i\leq m}|x_i|\right)^2 \right)^{1/2} = \sqrt{m}\max_{1\leq i\leq m}|x_i| = \sqrt{m}\norm{x}_\infty \end{align*}
	Equality is obtained when all components of \( x \) are equal, in which case \( \norm{x}_2 = \left(\sum_{i=1}^{m}|x_i|^2\right)^{1/2} = \left(m|x_i|^2\right)^{1/2} = \sqrt{m}\norm{x}_\infty \).
\end{enumerate}

We now have \( \norm{x}_\infty\leq \norm{x}_2\leq \sqrt{m}\norm{x}_\infty \) for \( x\in \CC^m \). Let \( A\in\CC^{m\times n} \). Note that for any vector \( u\in\CC^n \), \( Au\in\CC^m \).

\begin{enumerate}
	\item[(c)] Denote the \( i \)-th row of \( A \) by \( a_i^* \) and define \( x_0 \in \CC^n \) to be the vector with all entries equal to 1. Then observe \( \norm{a_i^*}_1 = a_i \)
	\[ \norm{A}_\infty = \sup_{u\neq0}\dfrac{\norm{Au}_\infty}{\norm{u}_\infty} \leq \sup_{u\neq0}\dfrac{\norm{Au}_2}{\norm{u}_\infty} \leq \sup_{u\neq0}\dfrac{\norm{Au}_2}{\norm{u}_2/\sqrt{n}} = \sqrt{n}\sup_{u\neq0}\dfrac{\norm{Au}_2}{\norm{u}_2} = \sqrt{n}\norm{A}_2 \]
	Denote the vector with zeros in all components except for a 1 in the \( j \)-th component by \( e_j \). Denote the vector with all ones by \( 1 \).
	
	Now suppose \( e_j \) has length \( m \) and \( 1 \) has length \( n \). Let \( A=ae_j1^* \) for some scalar \( a \). Then \( A \) is dimension \( m\times n \) and looks like the zero matrix with the \( i \)-th row constant and equal to \( a \).
	
	Then clearly \( \norm{A}_\infty = n|a| \). Moreover, by our matrix norm rules for outer products, \( \norm{A}_2 = |a|\norm{e_j}_2\norm{1^*}_2 = |a|1\sqrt{n} = \sqrt{m}|n| = \norm{A}_\infty/\sqrt{n} \) so equality is obtained.	,
	
	
	\item[(d)]
	\[ \norm{A}_2 = \sup_{u\neq 0}\dfrac{\norm{Au}_2}{\norm{u}_2} = \leq \sup_{u\neq0}\dfrac{\sqrt{m}\norm{Au}_\infty}{\norm{u}_2} \leq  \sup_{u\neq0}\dfrac{\sqrt{m}\norm{Au}_\infty}{\norm{u}_\infty} = \sqrt{m}\sup_{u\neq0}\dfrac{\norm{Au}_\infty}{\norm{u}_\infty} = \sqrt{m}\norm{A}_\infty \]
	Suppose \( e_j \) has length \( n \) and \( 1 \) has length \( m \). Let \( A=a1e_j^* \) for some scalar \( a \). Then \( A \) is dimension \( m\times n \) and looks like the zero matrix with the \( j \)-th column constant and equal to \( a \).
	
	Then clearly \( \norm{A}_\infty = |a| \). Moreover, by our matrix norm rules for outer products, \( \norm{A}_2 = |a|\norm{1}_2\norm{e_j^*}_2 = |a|\sqrt{m}1 = \sqrt{m}|a| = \sqrt{m}\norm{A}_\infty \), so equality is obtained.
	
	
\end{enumerate}
\end{solution}

\end{document}
