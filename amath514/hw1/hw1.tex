\documentclass[10pt]{article}
\usepackage[T1]{fontenc}

% Document Details
\newcommand{\CLASS}{AMATH 514}
\newcommand{\assigmentnum}{Assignment 1}

\usepackage[margin = 1.15in, top = 1.25in, bottom = 1.in]{geometry}

\usepackage{titling}
\setlength{\droptitle}{-6em}   % This is your set screw
\date{}
\renewcommand{\maketitle}{
	\clearpage
	\begingroup  
	\centering
	\LARGE \sffamily\textbf{\CLASS} \Large \assigmentnum\\[.8em]
	\large Tyler Chen\\[1em]
	\endgroup
	\thispagestyle{empty}
}
 % Title Styling
\usepackage{tocloft}
\renewcommand{\cfttoctitlefont}{\Large\sffamily\bfseries}
\renewcommand{\cftsecfont}{\normalfont\sffamily\bfseries}
\renewcommand{\cftsubsecfont}{\normalfont\sffamily}
\renewcommand{\cftsubsubsecfont}{\normalfont\sffamily}

\makeatletter
\let\oldl@section\l@section
\def\l@section#1#2{\oldl@section{#1}{\sffamily\bfseries#2}}

\let\oldl@subsection\l@subsection
\def\l@subsection#1#2{\oldl@subsection{#1}{\sffamily#2}}

\let\oldl@subsubsection\l@subsubsection
\def\l@subsubsection#1#2{\oldl@subsubsection{#1}{\sffamily#2}}
 % General Styling


\usepackage{enumitem}

% Figures
\usepackage{subcaption}

% TikZ and Graphics
\usepackage{tikz, pgfplots}
\pgfplotsset{compat=1.12}
\usetikzlibrary{patterns,arrows}
\usepgfplotslibrary{fillbetween}

\usepackage{pdfpages}
\usepackage{adjustbox}

\usepackage{lscape}
\usepackage{titling}
\usepackage[]{hyperref}


% Header Styling
\usepackage{fancyhdr}
\pagestyle{fancy}
\lhead{\sffamily \CLASS}
\rhead{\sffamily Chen \textbf{\thepage}}
\cfoot{}

% Paragraph Styling
\setlength{\columnsep}{1cm}
\setlength{\parindent}{0pt}
\setlength{\parskip}{5pt}
\renewcommand{\baselinestretch}{1}

% TOC Styling
\usepackage{tocloft}
\iffalse
\renewcommand{\cftsecleader}{\cftdotfill{\cftdotsep}}

\renewcommand\cftchapafterpnum{\vskip6pt}
\renewcommand\cftsecafterpnum{\vskip10pt}
\renewcommand\cftsubsecafterpnum{\vskip6pt}

% Adjust sectional unit title fonts in ToC
\renewcommand{\cftchapfont}{\sffamily}
\renewcommand{\cftsecfont}{\bfseries\sffamily}
\renewcommand{\cftsecnumwidth}{2em}
\renewcommand{\cftsubsecfont}{\sffamily}
\renewcommand{\cfttoctitlefont}{\hfill\bfseries\sffamily\MakeUppercase}
\renewcommand{\cftaftertoctitle}{\hfill}

\renewcommand{\cftchappagefont}{\sffamily}
\renewcommand{\cftsecpagefont}{\bfseries\sffamily}
\renewcommand{\cftsubsecpagefont}{\sffamily}
\fi
 % General Styling
% Code Display Setup
\usepackage{listings,lstautogobble}
\usepackage{lipsum}
\usepackage{courier}
\usepackage{catchfilebetweentags}

\lstset{
	basicstyle=\small\ttfamily,
	breaklines=true, 
	frame = single,
	rangeprefix=,
	rangesuffix=,
	includerangemarker=false,
	autogobble = true
}


\usepackage{algorithmicx}
\usepackage{algpseudocode}

\newcommand{\To}{\textbf{to}~}
\newcommand{\DownTo}{\textbf{downto}~}
\renewcommand{\algorithmicdo}{\hspace{-.2em}\textbf{:}}
 % Code Display Setup
% AMS MATH Styling
\usepackage{amsmath, amssymb}
\newcommand{\qed}{\hfill\(\square\)}

%\newtheorem*{lemma}{Lemma} 
%\newtheorem*{theorem}{Theorem}
%\newtheorem*{definition}{Definition}
%\newtheorem*{prop}{Proposition}
%\renewenvironment{proof}{{\bfseries Proof.}}{}


% mathcal
\newcommand{\cA}{\ensuremath{\mathcal{A}}}
\newcommand{\cB}{\ensuremath{\mathcal{B}}}
\newcommand{\cC}{\ensuremath{\mathcal{C}}}
\newcommand{\cD}{\ensuremath{\mathcal{D}}}
\newcommand{\cE}{\ensuremath{\mathcal{E}}}
\newcommand{\cF}{\ensuremath{\mathcal{F}}}
\newcommand{\cG}{\ensuremath{\mathcal{G}}}
\newcommand{\cH}{\ensuremath{\mathcal{H}}}
\newcommand{\cI}{\ensuremath{\mathcal{I}}}
\newcommand{\cJ}{\ensuremath{\mathcal{J}}}
\newcommand{\cK}{\ensuremath{\mathcal{K}}}
\newcommand{\cL}{\ensuremath{\mathcal{L}}}
\newcommand{\cM}{\ensuremath{\mathcal{M}}}
\newcommand{\cN}{\ensuremath{\mathcal{N}}}
\newcommand{\cO}{\ensuremath{\mathcal{O}}}
\newcommand{\cP}{\ensuremath{\mathcal{P}}}
\newcommand{\cQ}{\ensuremath{\mathcal{Q}}}
\newcommand{\cR}{\ensuremath{\mathcal{R}}}
\newcommand{\cS}{\ensuremath{\mathcal{S}}}
\newcommand{\cT}{\ensuremath{\mathcal{T}}}
\newcommand{\cU}{\ensuremath{\mathcal{U}}}
\newcommand{\cV}{\ensuremath{\mathcal{V}}}
\newcommand{\cW}{\ensuremath{\mathcal{W}}}
\newcommand{\cX}{\ensuremath{\mathcal{X}}}
\newcommand{\cY}{\ensuremath{\mathcal{Y}}}
\newcommand{\cZ}{\ensuremath{\mathcal{Z}}}

% mathbb
\usepackage{bbm}
\newcommand{\bOne}{\ensuremath{\mathbbm{1}}}

\newcommand{\bA}{\ensuremath{\mathbb{A}}}
\newcommand{\bB}{\ensuremath{\mathbb{B}}}
\newcommand{\bC}{\ensuremath{\mathbb{C}}}
\newcommand{\bD}{\ensuremath{\mathbb{D}}}
\newcommand{\bE}{\ensuremath{\mathbb{E}}}
\newcommand{\bF}{\ensuremath{\mathbb{F}}}
\newcommand{\bG}{\ensuremath{\mathbb{G}}}
\newcommand{\bH}{\ensuremath{\mathbb{H}}}
\newcommand{\bI}{\ensuremath{\mathbb{I}}}
\newcommand{\bJ}{\ensuremath{\mathbb{J}}}
\newcommand{\bK}{\ensuremath{\mathbb{K}}}
\newcommand{\bL}{\ensuremath{\mathbb{L}}}
\newcommand{\bM}{\ensuremath{\mathbb{M}}}
\newcommand{\bN}{\ensuremath{\mathbb{N}}}
\newcommand{\bO}{\ensuremath{\mathbb{O}}}
\newcommand{\bP}{\ensuremath{\mathbb{P}}}
\newcommand{\bQ}{\ensuremath{\mathbb{Q}}}
\newcommand{\bR}{\ensuremath{\mathbb{R}}}
\newcommand{\bS}{\ensuremath{\mathbb{S}}}
\newcommand{\bT}{\ensuremath{\mathbb{T}}}
\newcommand{\bU}{\ensuremath{\mathbb{U}}}
\newcommand{\bV}{\ensuremath{\mathbb{V}}}
\newcommand{\bW}{\ensuremath{\mathbb{W}}}
\newcommand{\bX}{\ensuremath{\mathbb{X}}}
\newcommand{\bY}{\ensuremath{\mathbb{Y}}}
\newcommand{\bZ}{\ensuremath{\mathbb{Z}}}

% alternative mathbb
\newcommand{\NN}{\ensuremath{\mathbb{N}}}
\newcommand{\RR}{\ensuremath{\mathbb{R}}}
\newcommand{\CC}{\ensuremath{\mathbb{C}}}
\newcommand{\ZZ}{\ensuremath{\mathbb{Z}}}
\newcommand{\EE}{\ensuremath{\mathbb{E}}}
\newcommand{\PP}{\ensuremath{\mathbb{P}}}
\newcommand{\VV}{\ensuremath{\mathbb{V}}}
\newcommand{\cov}{\ensuremath{\text{Co}\VV}}
% Math Commands

\newcommand{\st}{~\big|~}
\newcommand{\stt}{\text{ st. }}
\newcommand{\ift}{\text{ if }}
\newcommand{\thent}{\text{ then }}
\newcommand{\owt}{\text{ otherwise }}

\newcommand{\norm}[1]{\left\lVert#1\right\rVert}
\newcommand{\snorm}[1]{\lVert#1\rVert}
\newcommand{\ip}[1]{\ensuremath{\left\langle #1 \right\rangle}}
\newcommand{\pp}[3][]{\frac{\partial^{#1}#2}{\partial #3^{#1}}}
\newcommand{\dd}[3][]{\frac{\d^{#1}#2}{\d #3^{#1}}}
\renewcommand{\d}{\ensuremath{\mathrm{d}}}

\newcommand{\indep}{\rotatebox[origin=c]{90}{$\models$}}




 % Math shortcuts
% Problem
\usepackage{floatrow}

\newenvironment{problem}[1][]
{\pagebreak
\noindent\rule{\textwidth}{1pt}\vspace{0.25em}
{\sffamily \textbf{#1}}
\par
}
{\par\vspace{-0.5em}\noindent\rule{\textwidth}{1pt}}

\newenvironment{solution}[1][]
{{\sffamily \textbf{#1}}
\par
}
{}

 % Problem Environment

\newcommand{\note}[1]{\textcolor{red}{\textbf{Note:} #1}}

\hypersetup{
   colorlinks=true,       % false: boxed links; true: colored links
   linkcolor=violet,          % color of internal links (change box color with linkbordercolor)
   citecolor=green,        % color of links to bibliography
   filecolor=magenta,      % color of file links
   urlcolor=cyan           % color of external links
}


\begin{document}
\maketitle

\begin{problem}[Exercise 1.7]
Find, both with the Dijkstra-Prim algorithm and with Kruskal's algorithm, a spanning tree of minimum length in the graph in Figure 1.5

\begin{figure}[h]\centering
\tikzstyle{vertex}=[circle,fill=black,scale=.4]
\begin{tikzpicture}[scale=0.5]
    
    \foreach \pos/\name in {{(0,6)/a},{(4,6)/b},{(8,6)/c},{(12,6)/d},
                            {(2,3)/e},{(6,3)/f},{(10,3)/g},
                            {(0,0)/h},{(4,0)/i},{(8,0)/j},{(12,0)/k}}
        \node[vertex] (\name) at \pos {};

    \foreach \source/\dest/\weight in {a/b/2, b/c/4, c/d/1,
                                       a/h/3,a/e/5,e/b/4,b/f/4,f/c/6,c/g/2,g/d/3,d/k/2,
                                       e/f/5,f/g/3,
                                       h/e/4,e/i/3,i/f/5,f/j/7,j/g/4,g/k/2,
                                       h/i/3,i/j/6,j/k/3}
        \path[draw] (\source) -- node[fill=white,scale=.8] {$\weight$} (\dest);
\end{tikzpicture}
    \caption{}
\end{figure}
\end{problem}

\begin{solution}[Solution]
We arbitrarily choose the top left node as the starting node and proceed adding the lowest weight edge in the cut of the existing tree which maintains the tree structure.
\begin{figure}[H]\centering
    \begin{subfigure}{.3\textwidth}\centering
        \tikzstyle{vertex}=[circle,fill=black,scale=.4]
        \begin{tikzpicture}[scale=0.3]
    
        \foreach \pos/\name in {{(0,6)/a},{(4,6)/b},{(8,6)/c},{(12,6)/d},
                                {(2,3)/e},{(6,3)/f},{(10,3)/g},
                                {(0,0)/h},{(4,0)/i},{(8,0)/j},{(12,0)/k}}
            \node[vertex] (\name) at \pos {};
        
        \foreach \source/\dest in {a/b}
            \path[draw,red,line width=2pt] (\source) -- node[fill=white,scale=.8] {} (\dest);

        \foreach \source/\dest/\weight in {a/b/2, b/c/4, c/d/1,
                                           a/h/3,a/e/5,e/b/4,b/f/4,f/c/6,c/g/2,g/d/3,d/k/2,
                                           e/f/5,f/g/3,
                                           h/e/4,e/i/3,i/f/5,f/j/7,j/g/4,g/k/2,
                                           h/i/3,i/j/6,j/k/3}
            \path[draw] (\source) -- node[fill=white,scale=.8] {$\weight$} (\dest);
    \end{tikzpicture}
        \caption{}
    \end{subfigure}
    %
    \begin{subfigure}{.3\textwidth}\centering
        \tikzstyle{vertex}=[circle,fill=black,scale=.4]
        \begin{tikzpicture}[scale=0.3]
    
        \foreach \pos/\name in {{(0,6)/a},{(4,6)/b},{(8,6)/c},{(12,6)/d},
                                {(2,3)/e},{(6,3)/f},{(10,3)/g},
                                {(0,0)/h},{(4,0)/i},{(8,0)/j},{(12,0)/k}}
            \node[vertex] (\name) at \pos {};

         \foreach \source/\dest in {a/b,b/c}
            \path[draw,red,line width=2pt] (\source) -- node[fill=white,scale=.8] {} (\dest);
       
        \foreach \source/\dest/\weight in {a/b/2, b/c/4, c/d/1,
                                           a/h/3,a/e/5,e/b/4,b/f/4,f/c/6,c/g/2,g/d/3,d/k/2,
                                           e/f/5,f/g/3,
                                           h/e/4,e/i/3,i/f/5,f/j/7,j/g/4,g/k/2,
                                           h/i/3,i/j/6,j/k/3}
            \path[draw] (\source) -- node[fill=white,scale=.8] {$\weight$} (\dest);
    \end{tikzpicture}
        \caption{}
    \end{subfigure}
    %
    \begin{subfigure}{.3\textwidth}\centering
        \tikzstyle{vertex}=[circle,fill=black,scale=.4]
        \begin{tikzpicture}[scale=0.3]
    
        \foreach \pos/\name in {{(0,6)/a},{(4,6)/b},{(8,6)/c},{(12,6)/d},
                                {(2,3)/e},{(6,3)/f},{(10,3)/g},
                                {(0,0)/h},{(4,0)/i},{(8,0)/j},{(12,0)/k}}
            \node[vertex] (\name) at \pos {};

         \foreach \source/\dest in {a/b,b/c,c/d}
            \path[draw,red,line width=2pt] (\source) -- node[fill=white,scale=.8] {} (\dest);
       
        \foreach \source/\dest/\weight in {a/b/2, b/c/4, c/d/1,
                                           a/h/3,a/e/5,e/b/4,b/f/4,f/c/6,c/g/2,g/d/3,d/k/2,
                                           e/f/5,f/g/3,
                                           h/e/4,e/i/3,i/f/5,f/j/7,j/g/4,g/k/2,
                                           h/i/3,i/j/6,j/k/3}
            \path[draw] (\source) -- node[fill=white,scale=.8] {$\weight$} (\dest);
    \end{tikzpicture}
        \caption{}
    \end{subfigure}
    %
    \begin{subfigure}{.3\textwidth}\centering
        \tikzstyle{vertex}=[circle,fill=black,scale=.4]
        \begin{tikzpicture}[scale=0.3]
    
        \foreach \pos/\name in {{(0,6)/a},{(4,6)/b},{(8,6)/c},{(12,6)/d},
                                {(2,3)/e},{(6,3)/f},{(10,3)/g},
                                {(0,0)/h},{(4,0)/i},{(8,0)/j},{(12,0)/k}}
            \node[vertex] (\name) at \pos {};

         \foreach \source/\dest in {a/b,b/c,c/d,d/k}
            \path[draw,red,line width=2pt] (\source) -- node[fill=white,scale=.8] {} (\dest);
       
        \foreach \source/\dest/\weight in {a/b/2, b/c/4, c/d/1,
                                           a/h/3,a/e/5,e/b/4,b/f/4,f/c/6,c/g/2,g/d/3,d/k/2,
                                           e/f/5,f/g/3,
                                           h/e/4,e/i/3,i/f/5,f/j/7,j/g/4,g/k/2,
                                           h/i/3,i/j/6,j/k/3}
            \path[draw] (\source) -- node[fill=white,scale=.8] {$\weight$} (\dest);
    \end{tikzpicture}
        \caption{}
    \end{subfigure}
    %
    \begin{subfigure}{.3\textwidth}\centering
        \tikzstyle{vertex}=[circle,fill=black,scale=.4]
        \begin{tikzpicture}[scale=0.3]
    
        \foreach \pos/\name in {{(0,6)/a},{(4,6)/b},{(8,6)/c},{(12,6)/d},
                                {(2,3)/e},{(6,3)/f},{(10,3)/g},
                                {(0,0)/h},{(4,0)/i},{(8,0)/j},{(12,0)/k}}
            \node[vertex] (\name) at \pos {};

         \foreach \source/\dest in {a/b,b/c,c/d,d/k,k/g}
            \path[draw,red,line width=2pt] (\source) -- node[fill=white,scale=.8] {} (\dest);
       
        \foreach \source/\dest/\weight in {a/b/2, b/c/4, c/d/1,
                                           a/h/3,a/e/5,e/b/4,b/f/4,f/c/6,c/g/2,g/d/3,d/k/2,
                                           e/f/5,f/g/3,
                                           h/e/4,e/i/3,i/f/5,f/j/7,j/g/4,g/k/2,
                                           h/i/3,i/j/6,j/k/3}
            \path[draw] (\source) -- node[fill=white,scale=.8] {$\weight$} (\dest);
    \end{tikzpicture}
        \caption{}
    \end{subfigure}
       %
    \begin{subfigure}{.3\textwidth}\centering
        \tikzstyle{vertex}=[circle,fill=black,scale=.4]
        \begin{tikzpicture}[scale=0.3]
    
        \foreach \pos/\name in {{(0,6)/a},{(4,6)/b},{(8,6)/c},{(12,6)/d},
                                {(2,3)/e},{(6,3)/f},{(10,3)/g},
                                {(0,0)/h},{(4,0)/i},{(8,0)/j},{(12,0)/k}}
            \node[vertex] (\name) at \pos {};

         \foreach \source/\dest in {a/b,b/c,c/d,d/k,k/g,g/f}
            \path[draw,red,line width=2pt] (\source) -- node[fill=white,scale=.8] {} (\dest);
       
        \foreach \source/\dest/\weight in {a/b/2, b/c/4, c/d/1,
                                           a/h/3,a/e/5,e/b/4,b/f/4,f/c/6,c/g/2,g/d/3,d/k/2,
                                           e/f/5,f/g/3,
                                           h/e/4,e/i/3,i/f/5,f/j/7,j/g/4,g/k/2,
                                           h/i/3,i/j/6,j/k/3}
            \path[draw] (\source) -- node[fill=white,scale=.8] {$\weight$} (\dest);
    \end{tikzpicture}
        \caption{}
    \end{subfigure}       %
    \begin{subfigure}{.3\textwidth}\centering
        \tikzstyle{vertex}=[circle,fill=black,scale=.4]
        \begin{tikzpicture}[scale=0.3]
    
        \foreach \pos/\name in {{(0,6)/a},{(4,6)/b},{(8,6)/c},{(12,6)/d},
                                {(2,3)/e},{(6,3)/f},{(10,3)/g},
                                {(0,0)/h},{(4,0)/i},{(8,0)/j},{(12,0)/k}}
            \node[vertex] (\name) at \pos {};

         \foreach \source/\dest in {a/b,b/c,c/d,d/k,k/g,g/f,k/j}
            \path[draw,red,line width=2pt] (\source) -- node[fill=white,scale=.8] {} (\dest);
       
        \foreach \source/\dest/\weight in {a/b/2, b/c/4, c/d/1,
                                           a/h/3,a/e/5,e/b/4,b/f/4,f/c/6,c/g/2,g/d/3,d/k/2,
                                           e/f/5,f/g/3,
                                           h/e/4,e/i/3,i/f/5,f/j/7,j/g/4,g/k/2,
                                           h/i/3,i/j/6,j/k/3}
            \path[draw] (\source) -- node[fill=white,scale=.8] {$\weight$} (\dest);
    \end{tikzpicture}
        \caption{}
    \end{subfigure}
   %
    \begin{subfigure}{.3\textwidth}\centering
        \tikzstyle{vertex}=[circle,fill=black,scale=.4]
        \begin{tikzpicture}[scale=0.3]
    
        \foreach \pos/\name in {{(0,6)/a},{(4,6)/b},{(8,6)/c},{(12,6)/d},
                                {(2,3)/e},{(6,3)/f},{(10,3)/g},
                                {(0,0)/h},{(4,0)/i},{(8,0)/j},{(12,0)/k}}
            \node[vertex] (\name) at \pos {};

         \foreach \source/\dest in {a/b,b/c,c/d,d/k,k/g,g/f,k/j,a/h}
            \path[draw,red,line width=2pt] (\source) -- node[fill=white,scale=.8] {} (\dest);
       
        \foreach \source/\dest/\weight in {a/b/2, b/c/4, c/d/1,
                                           a/h/3,a/e/5,e/b/4,b/f/4,f/c/6,c/g/2,g/d/3,d/k/2,
                                           e/f/5,f/g/3,
                                           h/e/4,e/i/3,i/f/5,f/j/7,j/g/4,g/k/2,
                                           h/i/3,i/j/6,j/k/3}
            \path[draw] (\source) -- node[fill=white,scale=.8] {$\weight$} (\dest);
    \end{tikzpicture}
        \caption{}
    \end{subfigure}
   %
    \begin{subfigure}{.3\textwidth}\centering
        \tikzstyle{vertex}=[circle,fill=black,scale=.4]
        \begin{tikzpicture}[scale=0.3]
    
        \foreach \pos/\name in {{(0,6)/a},{(4,6)/b},{(8,6)/c},{(12,6)/d},
                                {(2,3)/e},{(6,3)/f},{(10,3)/g},
                                {(0,0)/h},{(4,0)/i},{(8,0)/j},{(12,0)/k}}
            \node[vertex] (\name) at \pos {};

         \foreach \source/\dest in {a/b,b/c,c/d,d/k,k/g,g/f,k/j,a/h,h/i}
            \path[draw,red,line width=2pt] (\source) -- node[fill=white,scale=.8] {} (\dest);
       
        \foreach \source/\dest/\weight in {a/b/2, b/c/4, c/d/1,
                                           a/h/3,a/e/5,e/b/4,b/f/4,f/c/6,c/g/2,g/d/3,d/k/2,
                                           e/f/5,f/g/3,
                                           h/e/4,e/i/3,i/f/5,f/j/7,j/g/4,g/k/2,
                                           h/i/3,i/j/6,j/k/3}
            \path[draw] (\source) -- node[fill=white,scale=.8] {$\weight$} (\dest);
    \end{tikzpicture}
        \caption{}
    \end{subfigure}
   %
    \begin{subfigure}{.3\textwidth}\centering
        \tikzstyle{vertex}=[circle,fill=black,scale=.4]
        \begin{tikzpicture}[scale=0.3]
    
        \foreach \pos/\name in {{(0,6)/a},{(4,6)/b},{(8,6)/c},{(12,6)/d},
                                {(2,3)/e},{(6,3)/f},{(10,3)/g},
                                {(0,0)/h},{(4,0)/i},{(8,0)/j},{(12,0)/k}}
            \node[vertex] (\name) at \pos {};

         \foreach \source/\dest in {a/b,b/c,c/d,d/k,k/g,g/f,k/j,a/h,h/i,i/e}
            \path[draw,red,line width=2pt] (\source) -- node[fill=white,scale=.8] {} (\dest);
       
        \foreach \source/\dest/\weight in {a/b/2, b/c/4, c/d/1,
                                           a/h/3,a/e/5,e/b/4,b/f/4,f/c/6,c/g/2,g/d/3,d/k/2,
                                           e/f/5,f/g/3,
                                           h/e/4,e/i/3,i/f/5,f/j/7,j/g/4,g/k/2,
                                           h/i/3,i/j/6,j/k/3}
            \path[draw] (\source) -- node[fill=white,scale=.8] {$\weight$} (\dest);
    \end{tikzpicture}
        \caption{}
    \end{subfigure}
\end{figure}


We order the edges. Since the choice of edge in a tie is arbitrary we work from top to bottom left to right. We add edges in this order if they maintain a forest structure (no cycles).
\begin{figure}[H]\centering
    \begin{subfigure}{.3\textwidth}\centering
        \tikzstyle{vertex}=[circle,fill=black,scale=.4]
        \begin{tikzpicture}[scale=0.3]
    
        \foreach \pos/\name in {{(0,6)/a},{(4,6)/b},{(8,6)/c},{(12,6)/d},
                                {(2,3)/e},{(6,3)/f},{(10,3)/g},
                                {(0,0)/h},{(4,0)/i},{(8,0)/j},{(12,0)/k}}
            \node[vertex] (\name) at \pos {};
        
        \foreach \source/\dest in {c/d}
            \path[draw,red,line width=2pt] (\source) -- node[fill=white,scale=.8] {} (\dest);

        \foreach \source/\dest/\weight in {a/b/2, b/c/4, c/d/1,
                                           a/h/3,a/e/5,e/b/4,b/f/4,f/c/6,c/g/2,g/d/3,d/k/2,
                                           e/f/5,f/g/3,
                                           h/e/4,e/i/3,i/f/5,f/j/7,j/g/4,g/k/2,
                                           h/i/3,i/j/6,j/k/3}
            \path[draw] (\source) -- node[fill=white,scale=.8] {$\weight$} (\dest);
    \end{tikzpicture}
        \caption{}
    \end{subfigure}
    %
    \begin{subfigure}{.3\textwidth}\centering
        \tikzstyle{vertex}=[circle,fill=black,scale=.4]
        \begin{tikzpicture}[scale=0.3]
    
        \foreach \pos/\name in {{(0,6)/a},{(4,6)/b},{(8,6)/c},{(12,6)/d},
                                {(2,3)/e},{(6,3)/f},{(10,3)/g},
                                {(0,0)/h},{(4,0)/i},{(8,0)/j},{(12,0)/k}}
            \node[vertex] (\name) at \pos {};

         \foreach \source/\dest in {c/d,a/b}
            \path[draw,red,line width=2pt] (\source) -- node[fill=white,scale=.8] {} (\dest);
       
        \foreach \source/\dest/\weight in {a/b/2, b/c/4, c/d/1,
                                           a/h/3,a/e/5,e/b/4,b/f/4,f/c/6,c/g/2,g/d/3,d/k/2,
                                           e/f/5,f/g/3,
                                           h/e/4,e/i/3,i/f/5,f/j/7,j/g/4,g/k/2,
                                           h/i/3,i/j/6,j/k/3}
            \path[draw] (\source) -- node[fill=white,scale=.8] {$\weight$} (\dest);
    \end{tikzpicture}
        \caption{}
    \end{subfigure}
    %
    \begin{subfigure}{.3\textwidth}\centering
        \tikzstyle{vertex}=[circle,fill=black,scale=.4]
        \begin{tikzpicture}[scale=0.3]
    
        \foreach \pos/\name in {{(0,6)/a},{(4,6)/b},{(8,6)/c},{(12,6)/d},
                                {(2,3)/e},{(6,3)/f},{(10,3)/g},
                                {(0,0)/h},{(4,0)/i},{(8,0)/j},{(12,0)/k}}
            \node[vertex] (\name) at \pos {};

         \foreach \source/\dest in {c/d,a/b,c/g}
            \path[draw,red,line width=2pt] (\source) -- node[fill=white,scale=.8] {} (\dest);
       
        \foreach \source/\dest/\weight in {a/b/2, b/c/4, c/d/1,
                                           a/h/3,a/e/5,e/b/4,b/f/4,f/c/6,c/g/2,g/d/3,d/k/2,
                                           e/f/5,f/g/3,
                                           h/e/4,e/i/3,i/f/5,f/j/7,j/g/4,g/k/2,
                                           h/i/3,i/j/6,j/k/3}
            \path[draw] (\source) -- node[fill=white,scale=.8] {$\weight$} (\dest);
    \end{tikzpicture}
        \caption{}
    \end{subfigure}
    %
    \begin{subfigure}{.3\textwidth}\centering
        \tikzstyle{vertex}=[circle,fill=black,scale=.4]
        \begin{tikzpicture}[scale=0.3]
    
        \foreach \pos/\name in {{(0,6)/a},{(4,6)/b},{(8,6)/c},{(12,6)/d},
                                {(2,3)/e},{(6,3)/f},{(10,3)/g},
                                {(0,0)/h},{(4,0)/i},{(8,0)/j},{(12,0)/k}}
            \node[vertex] (\name) at \pos {};

         \foreach \source/\dest in {c/d,a/b,c/g,d/k}
            \path[draw,red,line width=2pt] (\source) -- node[fill=white,scale=.8] {} (\dest);
       
        \foreach \source/\dest/\weight in {a/b/2, b/c/4, c/d/1,
                                           a/h/3,a/e/5,e/b/4,b/f/4,f/c/6,c/g/2,g/d/3,d/k/2,
                                           e/f/5,f/g/3,
                                           h/e/4,e/i/3,i/f/5,f/j/7,j/g/4,g/k/2,
                                           h/i/3,i/j/6,j/k/3}
            \path[draw] (\source) -- node[fill=white,scale=.8] {$\weight$} (\dest);
    \end{tikzpicture}
        \caption{}
    \end{subfigure}
    %
    \begin{subfigure}{.3\textwidth}\centering
        \tikzstyle{vertex}=[circle,fill=black,scale=.4]
        \begin{tikzpicture}[scale=0.3]
    
        \foreach \pos/\name in {{(0,6)/a},{(4,6)/b},{(8,6)/c},{(12,6)/d},
                                {(2,3)/e},{(6,3)/f},{(10,3)/g},
                                {(0,0)/h},{(4,0)/i},{(8,0)/j},{(12,0)/k}}
            \node[vertex] (\name) at \pos {};

         \foreach \source/\dest in {c/d,a/b,c/g,d/k,a/h}
            \path[draw,red,line width=2pt] (\source) -- node[fill=white,scale=.8] {} (\dest);
       
        \foreach \source/\dest/\weight in {a/b/2, b/c/4, c/d/1,
                                           a/h/3,a/e/5,e/b/4,b/f/4,f/c/6,c/g/2,g/d/3,d/k/2,
                                           e/f/5,f/g/3,
                                           h/e/4,e/i/3,i/f/5,f/j/7,j/g/4,g/k/2,
                                           h/i/3,i/j/6,j/k/3}
            \path[draw] (\source) -- node[fill=white,scale=.8] {$\weight$} (\dest);
    \end{tikzpicture}
        \caption{}
    \end{subfigure}
       %
    \begin{subfigure}{.3\textwidth}\centering
        \tikzstyle{vertex}=[circle,fill=black,scale=.4]
        \begin{tikzpicture}[scale=0.3]
    
        \foreach \pos/\name in {{(0,6)/a},{(4,6)/b},{(8,6)/c},{(12,6)/d},
                                {(2,3)/e},{(6,3)/f},{(10,3)/g},
                                {(0,0)/h},{(4,0)/i},{(8,0)/j},{(12,0)/k}}
            \node[vertex] (\name) at \pos {};

         \foreach \source/\dest in {c/d,a/b,c/g,d/k,a/h,f/g}
            \path[draw,red,line width=2pt] (\source) -- node[fill=white,scale=.8] {} (\dest);
       
        \foreach \source/\dest/\weight in {a/b/2, b/c/4, c/d/1,
                                           a/h/3,a/e/5,e/b/4,b/f/4,f/c/6,c/g/2,g/d/3,d/k/2,
                                           e/f/5,f/g/3,
                                           h/e/4,e/i/3,i/f/5,f/j/7,j/g/4,g/k/2,
                                           h/i/3,i/j/6,j/k/3}
            \path[draw] (\source) -- node[fill=white,scale=.8] {$\weight$} (\dest);
    \end{tikzpicture}
        \caption{}
    \end{subfigure}
   %
    \begin{subfigure}{.3\textwidth}\centering
        \tikzstyle{vertex}=[circle,fill=black,scale=.4]
        \begin{tikzpicture}[scale=0.3]
    
        \foreach \pos/\name in {{(0,6)/a},{(4,6)/b},{(8,6)/c},{(12,6)/d},
                                {(2,3)/e},{(6,3)/f},{(10,3)/g},
                                {(0,0)/h},{(4,0)/i},{(8,0)/j},{(12,0)/k}}
            \node[vertex] (\name) at \pos {};

         \foreach \source/\dest in {c/d,a/b,c/g,d/k,a/h,f/g,e/i}
            \path[draw,red,line width=2pt] (\source) -- node[fill=white,scale=.8] {} (\dest);
       
        \foreach \source/\dest/\weight in {a/b/2, b/c/4, c/d/1,
                                           a/h/3,a/e/5,e/b/4,b/f/4,f/c/6,c/g/2,g/d/3,d/k/2,
                                           e/f/5,f/g/3,
                                           h/e/4,e/i/3,i/f/5,f/j/7,j/g/4,g/k/2,
                                           h/i/3,i/j/6,j/k/3}
            \path[draw] (\source) -- node[fill=white,scale=.8] {$\weight$} (\dest);
    \end{tikzpicture}
        \caption{}
    \end{subfigure}
   %
    \begin{subfigure}{.3\textwidth}\centering
        \tikzstyle{vertex}=[circle,fill=black,scale=.4]
        \begin{tikzpicture}[scale=0.3]
    
        \foreach \pos/\name in {{(0,6)/a},{(4,6)/b},{(8,6)/c},{(12,6)/d},
                                {(2,3)/e},{(6,3)/f},{(10,3)/g},
                                {(0,0)/h},{(4,0)/i},{(8,0)/j},{(12,0)/k}}
            \node[vertex] (\name) at \pos {};

         \foreach \source/\dest in {c/d,a/b,c/g,d/k,a/h,f/g,e/i,h/i}
            \path[draw,red,line width=2pt] (\source) -- node[fill=white,scale=.8] {} (\dest);
       
        \foreach \source/\dest/\weight in {a/b/2, b/c/4, c/d/1,
                                           a/h/3,a/e/5,e/b/4,b/f/4,f/c/6,c/g/2,g/d/3,d/k/2,
                                           e/f/5,f/g/3,
                                           h/e/4,e/i/3,i/f/5,f/j/7,j/g/4,g/k/2,
                                           h/i/3,i/j/6,j/k/3}
            \path[draw] (\source) -- node[fill=white,scale=.8] {$\weight$} (\dest);
    \end{tikzpicture}
        \caption{}
    \end{subfigure}
   %
    \begin{subfigure}{.3\textwidth}\centering
        \tikzstyle{vertex}=[circle,fill=black,scale=.4]
        \begin{tikzpicture}[scale=0.3]
    
        \foreach \pos/\name in {{(0,6)/a},{(4,6)/b},{(8,6)/c},{(12,6)/d},
                                {(2,3)/e},{(6,3)/f},{(10,3)/g},
                                {(0,0)/h},{(4,0)/i},{(8,0)/j},{(12,0)/k}}
            \node[vertex] (\name) at \pos {};

         \foreach \source/\dest in {c/d,a/b,c/g,d/k,a/h,f/g,e/i,h/i,j/k}
            \path[draw,red,line width=2pt] (\source) -- node[fill=white,scale=.8] {} (\dest);
       
        \foreach \source/\dest/\weight in {a/b/2, b/c/4, c/d/1,
                                           a/h/3,a/e/5,e/b/4,b/f/4,f/c/6,c/g/2,g/d/3,d/k/2,
                                           e/f/5,f/g/3,
                                           h/e/4,e/i/3,i/f/5,f/j/7,j/g/4,g/k/2,
                                           h/i/3,i/j/6,j/k/3}
            \path[draw] (\source) -- node[fill=white,scale=.8] {$\weight$} (\dest);
    \end{tikzpicture}
        \caption{}
    \end{subfigure}   %
    \begin{subfigure}{.3\textwidth}\centering
        \tikzstyle{vertex}=[circle,fill=black,scale=.4]
        \begin{tikzpicture}[scale=0.3]
    
        \foreach \pos/\name in {{(0,6)/a},{(4,6)/b},{(8,6)/c},{(12,6)/d},
                                {(2,3)/e},{(6,3)/f},{(10,3)/g},
                                {(0,0)/h},{(4,0)/i},{(8,0)/j},{(12,0)/k}}
            \node[vertex] (\name) at \pos {};

         \foreach \source/\dest in {c/d,a/b,c/g,d/k,a/h,f/g,e/i,h/i,j/k,b/d}
            \path[draw,red,line width=2pt] (\source) -- node[fill=white,scale=.8] {} (\dest);
       
        \foreach \source/\dest/\weight in {a/b/2, b/c/4, c/d/1,
                                           a/h/3,a/e/5,e/b/4,b/f/4,f/c/6,c/g/2,g/d/3,d/k/2,
                                           e/f/5,f/g/3,
                                           h/e/4,e/i/3,i/f/5,f/j/7,j/g/4,g/k/2,
                                           h/i/3,i/j/6,j/k/3}
            \path[draw] (\source) -- node[fill=white,scale=.8] {$\weight$} (\dest);
    \end{tikzpicture}
        \caption{}
    \end{subfigure}
\end{figure}

\end{solution}


\begin{problem}[Exercise 1.9]
    Let \( G=(V,E) \) be a graph and let \( l:E\to \RR \) be a `length' function. Call a forest \( F \) good if \( l(F') \geq l(F) \) for each forest \( F' \) satisfying \( |F'|=|F| \).

    Let \( F \) be a good forest and \( e \) be an edge not in \( F \) such that \( F\cup\{e\} \) is a forest and such that (among all such \( e \)) \( l(e) \) is as small as possible. Show that \( F\cup\{e\} \) is good again.
\end{problem}

\begin{solution}[Solution]
We first prove the following:

Let \( G=(V,E) \) be a graph, and \( (V,F),(V,X) \) be forests in \( G \) with \( |F| < |X| < \infty \). Then there is an edge \( x\in X \) such that \( (V,F\cup\{x\}) \) is a forest in \( G \).


Denote the connected components of \( F \) by \( (V_1,F_1), (V_2,F_2), ..., (V_k,F_k) \). 

Define \( Y = \cup_{j=1}^{k}\{ (u,v)\in X | u\in V_j \text{ and } v\in V_j \} \). That is, \( Y \) is the set of edges of \( X \) not contained in any cut \( \delta(V_j) \).

Observe that since \( X \) is acyclic, \( Y \) is also acyclic. Thus, each set \( \{(u,v)\in X | u,v\in V_j\} \) has size at most \( |V_j|-1 = |F_j| \). This means \( Y \) has size at most \( \sum_{j=1}^{k}|F_j| = |F| \).

Therefore, since \( X,Y \) are finite and \( Y\subseteq X \), \( |X\setminus Y| = |X| - |Y| > |X|-|F| > 0 \). This means there is some edge \( x=(u,v)\in X\setminus Y \). That is, there is some \( x=(u,v)\in X \) satisfying,
\begin{align*}
    \forall j\in\{1,2,...,k\}, (u,v)\notin \{(u,v)\in X|u\in V_j \text{ and }v\in V_j\} 
\end{align*}
Equivalently, there is some edge \( x=(u,v)\in X \) satisfying,
\begin{align*}
    \forall j\in\{1,2,...,k\}, u\not\in V_j \text{ or } v\notin V_j
\end{align*}

That is both edges of \( x \) cannot be in the same set \( V_j \) for any \( j \). In other words, \( x\in\delta(V_j) \) for some \( j \).


Then \( (V,F\cup\{x\}) \) is still a forest since adding \( x \) to \( F \) will not not induce a cycle. \qed


We now prove the main result. Indeed, let \( F \) be a good forest and let \( X \) be any forest in \( G \) satisfying \( |X| = |F\cup\{e\}| = |F|+1 \), where \( e \) is chosen such that \( F\cup\{e\} \) is a forest and \( l(e) \) is as small as possible.

By above, there is some \( x\in X \) such that \( F\cup\{x\} \) is a forest. By our choice of \( e \) we have \( l(x)\geq l(e) \).

Moreover, \( X\setminus\{x\} \) is a forest satisfying \( |X\setminus\{x\}| = |F| \). Since \( F \) is good, \( l(X\setminus\{x\}) \geq l(F) \).

Therefore,
\begin{align*}
    l(X) = l((X\setminus\{x\})\cup\{x\}) = l(X\setminus\{x\}) + l(x) \geq l(F)+l(e) = l(F\cup\{e\}) 
\end{align*}

This proves \( F\cup\{e\} \) is a good forest. \qed
\end{solution}

\begin{problem}[Exercise 10.1]
    Let \( X \) be a finite set and \( \mathcal{I}\subseteq 2^X \). Suppose (i) \( \emptyset \in \mathcal{I} \) and (ii) if \( Y\in \mathcal{I} \) and \( Z\subseteq Y \), then \( Z\in\mathcal{I} \).

    Show the following statements are equivalent:
    \begin{itemize}[nolistsep]
        \item[(iii)] if \( Y,Z\in\mathcal{I} \) and \( |Y|<|Z| \) then \( Y\cup\{x\}\in\mathcal{I} \) for some \( x\in Z\setminus Y \).
        \item[(3)] for any subset \( Y \) of \( X \), any two bases of \( Y \) have the same cardinality.
    \end{itemize}
\end{problem}


\begin{solution}[Solution]
Suppose \( (iii) \) and let \( Y\subseteq X \) with \( U,V \) bases for \( Y \). 

Suppose further, for the sake of contradiction, that \( U \) and \( V \) have different cardinalities. Without loss of generality, assume \( |U|<|V|\leq |Y| \). Then, by (iii) there is some \( v\in V\setminus U \) such that \( U\cup\{v\}\in\mathcal{I} \).

Since \( U \) is a proper subset of \( U\cup\{v\} \) and \( U \) is a basis (inclusionwise maximial) we must have \( U\cup\{v\} = Y \). Then \( |Y| = |U|+1 \), so since \( |V|>|U| \), and \( |Y|\geq|V| \), we have \( |V|=|Y| \). Moreover, since \( V\subseteq Y \), we require \( V=Y \). But then \( U \subsetneq V \) is not inclusionwise maximial in \( Y=V=U\cup\{v\} \), contradicting the hypothesis that \( U \) is a basis for \( Y \).

This proves \( U \) and \( V \) have the same cardinality.

Now, suppose not (iii). That is, let \( Y,Z\in\mathcal{I} \) with \( |Y|<|Z| \) and suppose \( Y\cup\{x\} \notin\mathcal{I} \) for any \( x\in Z\setminus Y \).

Let \( U = Z\cup Y \). Suppose \( Y \) is not a basis for \( U \). That is, that there is some \( W\in\mathcal{I} \) with \( Y\subsetneq W\subsetneq U  \). By size arguments it is obvious that \( W\cap Z \neq \emptyset \) so that there is some \( x\in Z\cap W \subseteq Z \). But then, \( Y,\{x\} \) are subsets of \( W \), so their union \( Y\cup\{x\} \) is a subset of \(  W \). By (ii), \( Y\cup\{x\}\in \mathcal{I} \), contradicting the hypothesis. Therefore \( Y \) is a basis for \( U \).


If \( Z \) is also inclusionwise maximial in \( U \), then \( Y,Z \) are bases for \( U \) of different cardinality. If \( Z \) is not inclusionwise maximal, then there is some \( W\supsetneq Z \) which is inclusionwise maximal (since \( U \) is finite). But clearly, \( |W|>|Z|>|Y| \) meaning \( W \) has a different cardinality than \( Y \).

This proves two bases of a subset of \( X \) need not have the same cardinality.

We have now shown (iii) \( \Longrightarrow \) (3) and not (iii) \( \Longrightarrow \) not (3). This proves (iii) \( \Longleftrightarrow \) (3). \qed
\end{solution}

\end{document}
