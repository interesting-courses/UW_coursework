\documentclass[10pt]{article}
\usepackage[T1]{fontenc}

% Document Details
\newcommand{\CLASS}{AMATH 514}
\newcommand{\assigmentnum}{Assignment 7}

\usepackage[margin = 1.15in, top = 1.25in, bottom = 1.in]{geometry}

\usepackage{titling}
\setlength{\droptitle}{-6em}   % This is your set screw
\date{}
\renewcommand{\maketitle}{
	\clearpage
	\begingroup  
	\centering
	\LARGE \sffamily\textbf{\CLASS} \Large \assigmentnum\\[.8em]
	\large Tyler Chen\\[1em]
	\endgroup
	\thispagestyle{empty}
}
 % Title Styling
\usepackage{tocloft}
\renewcommand{\cfttoctitlefont}{\Large\sffamily\bfseries}
\renewcommand{\cftsecfont}{\normalfont\sffamily\bfseries}
\renewcommand{\cftsubsecfont}{\normalfont\sffamily}
\renewcommand{\cftsubsubsecfont}{\normalfont\sffamily}

\makeatletter
\let\oldl@section\l@section
\def\l@section#1#2{\oldl@section{#1}{\sffamily\bfseries#2}}

\let\oldl@subsection\l@subsection
\def\l@subsection#1#2{\oldl@subsection{#1}{\sffamily#2}}

\let\oldl@subsubsection\l@subsubsection
\def\l@subsubsection#1#2{\oldl@subsubsection{#1}{\sffamily#2}}
 % General Styling


\usepackage{enumitem}

% Figures
\usepackage{subcaption}

% TikZ and Graphics
\usepackage{tikz, pgfplots}
\pgfplotsset{compat=1.12}
\usetikzlibrary{patterns,arrows}
\usepgfplotslibrary{fillbetween}

\usepackage{pdfpages}
\usepackage{adjustbox}

\usepackage{lscape}
\usepackage{titling}
\usepackage[]{hyperref}


% Header Styling
\usepackage{fancyhdr}
\pagestyle{fancy}
\lhead{\sffamily \CLASS}
\rhead{\sffamily Chen \textbf{\thepage}}
\cfoot{}

% Paragraph Styling
\setlength{\columnsep}{1cm}
\setlength{\parindent}{0pt}
\setlength{\parskip}{5pt}
\renewcommand{\baselinestretch}{1}

% TOC Styling
\usepackage{tocloft}
\iffalse
\renewcommand{\cftsecleader}{\cftdotfill{\cftdotsep}}

\renewcommand\cftchapafterpnum{\vskip6pt}
\renewcommand\cftsecafterpnum{\vskip10pt}
\renewcommand\cftsubsecafterpnum{\vskip6pt}

% Adjust sectional unit title fonts in ToC
\renewcommand{\cftchapfont}{\sffamily}
\renewcommand{\cftsecfont}{\bfseries\sffamily}
\renewcommand{\cftsecnumwidth}{2em}
\renewcommand{\cftsubsecfont}{\sffamily}
\renewcommand{\cfttoctitlefont}{\hfill\bfseries\sffamily\MakeUppercase}
\renewcommand{\cftaftertoctitle}{\hfill}

\renewcommand{\cftchappagefont}{\sffamily}
\renewcommand{\cftsecpagefont}{\bfseries\sffamily}
\renewcommand{\cftsubsecpagefont}{\sffamily}
\fi
 % General Styling
% Code Display Setup
\usepackage{listings,lstautogobble}
\usepackage{lipsum}
\usepackage{courier}
\usepackage{catchfilebetweentags}

\lstset{
	basicstyle=\small\ttfamily,
	breaklines=true, 
	frame = single,
	rangeprefix=,
	rangesuffix=,
	includerangemarker=false,
	autogobble = true
}


\usepackage{algorithmicx}
\usepackage{algpseudocode}

\newcommand{\To}{\textbf{to}~}
\newcommand{\DownTo}{\textbf{downto}~}
\renewcommand{\algorithmicdo}{\hspace{-.2em}\textbf{:}}
 % Code Display Setup
% AMS MATH Styling
\usepackage{amsmath, amssymb}
\newcommand{\qed}{\hfill\(\square\)}

%\newtheorem*{lemma}{Lemma} 
%\newtheorem*{theorem}{Theorem}
%\newtheorem*{definition}{Definition}
%\newtheorem*{prop}{Proposition}
%\renewenvironment{proof}{{\bfseries Proof.}}{}


% mathcal
\newcommand{\cA}{\ensuremath{\mathcal{A}}}
\newcommand{\cB}{\ensuremath{\mathcal{B}}}
\newcommand{\cC}{\ensuremath{\mathcal{C}}}
\newcommand{\cD}{\ensuremath{\mathcal{D}}}
\newcommand{\cE}{\ensuremath{\mathcal{E}}}
\newcommand{\cF}{\ensuremath{\mathcal{F}}}
\newcommand{\cG}{\ensuremath{\mathcal{G}}}
\newcommand{\cH}{\ensuremath{\mathcal{H}}}
\newcommand{\cI}{\ensuremath{\mathcal{I}}}
\newcommand{\cJ}{\ensuremath{\mathcal{J}}}
\newcommand{\cK}{\ensuremath{\mathcal{K}}}
\newcommand{\cL}{\ensuremath{\mathcal{L}}}
\newcommand{\cM}{\ensuremath{\mathcal{M}}}
\newcommand{\cN}{\ensuremath{\mathcal{N}}}
\newcommand{\cO}{\ensuremath{\mathcal{O}}}
\newcommand{\cP}{\ensuremath{\mathcal{P}}}
\newcommand{\cQ}{\ensuremath{\mathcal{Q}}}
\newcommand{\cR}{\ensuremath{\mathcal{R}}}
\newcommand{\cS}{\ensuremath{\mathcal{S}}}
\newcommand{\cT}{\ensuremath{\mathcal{T}}}
\newcommand{\cU}{\ensuremath{\mathcal{U}}}
\newcommand{\cV}{\ensuremath{\mathcal{V}}}
\newcommand{\cW}{\ensuremath{\mathcal{W}}}
\newcommand{\cX}{\ensuremath{\mathcal{X}}}
\newcommand{\cY}{\ensuremath{\mathcal{Y}}}
\newcommand{\cZ}{\ensuremath{\mathcal{Z}}}

% mathbb
\usepackage{bbm}
\newcommand{\bOne}{\ensuremath{\mathbbm{1}}}

\newcommand{\bA}{\ensuremath{\mathbb{A}}}
\newcommand{\bB}{\ensuremath{\mathbb{B}}}
\newcommand{\bC}{\ensuremath{\mathbb{C}}}
\newcommand{\bD}{\ensuremath{\mathbb{D}}}
\newcommand{\bE}{\ensuremath{\mathbb{E}}}
\newcommand{\bF}{\ensuremath{\mathbb{F}}}
\newcommand{\bG}{\ensuremath{\mathbb{G}}}
\newcommand{\bH}{\ensuremath{\mathbb{H}}}
\newcommand{\bI}{\ensuremath{\mathbb{I}}}
\newcommand{\bJ}{\ensuremath{\mathbb{J}}}
\newcommand{\bK}{\ensuremath{\mathbb{K}}}
\newcommand{\bL}{\ensuremath{\mathbb{L}}}
\newcommand{\bM}{\ensuremath{\mathbb{M}}}
\newcommand{\bN}{\ensuremath{\mathbb{N}}}
\newcommand{\bO}{\ensuremath{\mathbb{O}}}
\newcommand{\bP}{\ensuremath{\mathbb{P}}}
\newcommand{\bQ}{\ensuremath{\mathbb{Q}}}
\newcommand{\bR}{\ensuremath{\mathbb{R}}}
\newcommand{\bS}{\ensuremath{\mathbb{S}}}
\newcommand{\bT}{\ensuremath{\mathbb{T}}}
\newcommand{\bU}{\ensuremath{\mathbb{U}}}
\newcommand{\bV}{\ensuremath{\mathbb{V}}}
\newcommand{\bW}{\ensuremath{\mathbb{W}}}
\newcommand{\bX}{\ensuremath{\mathbb{X}}}
\newcommand{\bY}{\ensuremath{\mathbb{Y}}}
\newcommand{\bZ}{\ensuremath{\mathbb{Z}}}

% alternative mathbb
\newcommand{\NN}{\ensuremath{\mathbb{N}}}
\newcommand{\RR}{\ensuremath{\mathbb{R}}}
\newcommand{\CC}{\ensuremath{\mathbb{C}}}
\newcommand{\ZZ}{\ensuremath{\mathbb{Z}}}
\newcommand{\EE}{\ensuremath{\mathbb{E}}}
\newcommand{\PP}{\ensuremath{\mathbb{P}}}
\newcommand{\VV}{\ensuremath{\mathbb{V}}}
\newcommand{\cov}{\ensuremath{\text{Co}\VV}}
% Math Commands

\newcommand{\st}{~\big|~}
\newcommand{\stt}{\text{ st. }}
\newcommand{\ift}{\text{ if }}
\newcommand{\thent}{\text{ then }}
\newcommand{\owt}{\text{ otherwise }}

\newcommand{\norm}[1]{\left\lVert#1\right\rVert}
\newcommand{\snorm}[1]{\lVert#1\rVert}
\newcommand{\ip}[1]{\ensuremath{\left\langle #1 \right\rangle}}
\newcommand{\pp}[3][]{\frac{\partial^{#1}#2}{\partial #3^{#1}}}
\newcommand{\dd}[3][]{\frac{\d^{#1}#2}{\d #3^{#1}}}
\renewcommand{\d}{\ensuremath{\mathrm{d}}}

\newcommand{\indep}{\rotatebox[origin=c]{90}{$\models$}}




 % Math shortcuts
% Problem
\usepackage{floatrow}

\newenvironment{problem}[1][]
{\pagebreak
\noindent\rule{\textwidth}{1pt}\vspace{0.25em}
{\sffamily \textbf{#1}}
\par
}
{\par\vspace{-0.5em}\noindent\rule{\textwidth}{1pt}}

\newenvironment{solution}[1][]
{{\sffamily \textbf{#1}}
\par
}
{}

 % Problem Environment

\newcommand{\note}[1]{\textcolor{red}{\textbf{Note:} #1}}

\hypersetup{
   colorlinks=true,       % false: boxed links; true: colored links
   linkcolor=violet,          % color of internal links (change box color with linkbordercolor)
   citecolor=green,        % color of links to bibliography
   filecolor=magenta,      % color of file links
   urlcolor=cyan           % color of external links
}


\begin{document}
\maketitle


\begin{problem}[Problem 5.1]
    \begin{enumerate}[label=(\roman*)]
    \item Show that a tree has at most one perfect matching
    \item Show (not using Tutte's 1-factor theorem) that a tree \( G = ( V, E )\) has a perfect matching if and only if the subgraph \( G-v \) has exactly one odd component, for each \( v\in V \).
\end{enumerate}
\end{problem}

\begin{solution}
\begin{enumerate}[label=(\roman*)]
    \item 
        
        Let \( G = (V,E) \) be a tree with a perfect matching \( M \).

        Small forests with zero, one, and two vertices clearly have at most one perfect matching.

        Suppose \( G = (V,E) \) is a forest with \( |V| > 1 \) and that all forests with fewer than \( |V| \) nodes have at most one perfect matching. 

        Then, since \( G \) is a forest, there is at least one vertex \( u \) of degree 1. 
        
        If there is no perfect matching we are done. Otherwise, since \( u \) has degree one, there is unique edge \( e = \{ u,v \} \) in \( E \). Therefore \( e \) must be in the matching on \( G \).

        Let \( G' = G \setminus \{u,v\} \). Then \( G' \) is a subgraph of \( G \) and therefore a forest. By the inductive hypothesis, since \( |V'| < |V| \), we have \( G' \) having at most on perfect matching. Therefore, there is at most one perfect matching on \( G' \). 

\item 

    Suppose \( G=(V,E) \) is a tree with a perfect matching and let \( v\in V \). Since \( G \) has a perfect matching it must have an even number of vertices. Therefore \( G-v \) has an odd number of vertices.
        This means at least one component must be odd.


        Suppose, for the sake of contradiction, that \( G-v \) has more than one odd component. Then at least one is not attached to \( v \) by an edge in the matching.
        
        Denote one such component by \( C \).
        Let \( u \) be the vertex in \( C \) such that \( \{v,u\} \in E \). Since \( \{u,v\} \) is not in the perfect matching, \( u \) must be covered by a matching edge in \( C \).
        Therefore \( C \) must have a perfect matching.
        
        This is a contradiction as \( C \) has an odd number of vertices and cannot contain a perfect matching. 

        Therefore \( G-v \) has exactly one odd component.
       
        Conversely, suppose \( G-v \) has exactly one odd component for each \( v\in V \). We provide an algorithm to find a perfect matching.

        Indeed, for each vertex \( v\in V \) add the edge of \( G \) connecting \( v \) to the odd component of \( G-v \) to the output (ignore duplicates).

        Clearly this will produce a set of edges which cover every vertex. It remains to show that the set of edges output is a matching.

        Suppose we are on vertex \( v \) and that the edge \( \{v,u\} \) is added to the matching. This means the components \( C_i \) of \( G-v \) not containing \( u \) are all even.

        Consider \( G-u \). We know \( \{ v \}\cup(\cup_i C_i) \) is a component of \( G-u \). Since each \( C_i \) is even it must be the unique odd component of \( G-u \). Therefore the algorithm will add the edge \( \{u,v\} = \{v,u\} \) to the output. That is, the edges the algorithm outputs are a matching on \( G \). 
        
        This proves a tree \( G \) has a perfect matching if and only if the subgraph \( G-v \) has exactly one odd component, for each \( v\in V \). \qed



\end{enumerate}


\end{solution}

\begin{problem}[Problem 5.2]
    Let \( G \) be a 3-regular graph without any bridge. Show that \( G \) has a perfect matching. (A bridge is an edge \( e \) not contained in any circuit; equivalently, deleting \( e \) increases the number of components; equivalently, \( \{ e \} \) is a cut.)
\end{problem}

\begin{solution}

Write \( G=(V,E) \). Let \( U \subseteq V \) and consider \( G - U \). Let \( C = (W,F) \) be an odd component of \( G - U \). 

Since \( G \) is 3-regular the sum of the degrees (in \( G \)) of the vertices in \( W \), \( \sum_{v\in W} \deg_G(v) = 3|W| \).

However, \( C \) is also a graph. The sum of the degrees of vertices in a graph is even, \( \sum_{v\in W} \deg_C(v) \) is even.

Therefore there are an odd number of edges between \( C \) and \( U \).

Suppose \( C \) were connected to \( U \) by a single edge. Then deleting this edge in \( G \) would mean \( C \) again becomes a component. That is, this edge is a bridge.

Therefore there are at least three edges between \( C \) to \( U \).

Then there can be at most \( |U| \) odd components in \( G - U \). Therefore, by Tutte's 1 factor theorem \( G \) has a perfect matching. \qed



\textit{I got a hint form here: https://math.stackexchange.com/questions/81257/3-regular-graphs-with-no-bridges. I don't have any graph theory background so I hadn't thought of some of theses facts about the degrees of a graph.}






\end{solution}

\begin{problem}[Problem 5.4]
    Let \( G = (V,E) \) be a graph and let \( T \) be a subset of \( V \). Show \( G \) has a matching covering \( T \) if and only if the number of odd components of \( G-W \) contained in \( T \) is at most \( |W| \), for each \( W\subseteq V \).
\end{problem}

\begin{solution}

Construct a new graph \( G_T \) by reflecting the graph \( G \) and connecting each point not in \( T \) to its image in the mirror graph. That is, define \( G_T = (V\cup V', E\cup E'\cup L) \) where:
\begin{itemize}[nolistsep]
    \item For each \( v\in V \) define a new vertex \( v' \). Let \( V' \) denote all such vertices.
    \item For each \( e=\{u,v\}\in E \) define a new edge \( e' = \{u',v'\} \in E' \).
    \item For each \( v\in V\setminus T \) define a new edge \( \{v,v'\} \in L \).
\end{itemize}

For convenience, for every \( W\subseteq V \) denote the set of vertices in the mirror graph by \( W' \). That is, define \( W' = \{ w'\in V' : w\in W \} \). Similarly, for each \( F \subseteq E \) denote the set of edges in the mirror graph by \( F' \). That is, define \( F' = \{f'\in E': f\in F \} \).

Suppose \( G \) has a matching \( M \) covering \( T \). Each point in \( G \) not in \( T \) it part of an edge in \( L \). Thus, \( L\cup M \cup M' \) is a perfect matching in \( G_T \).

Now, suppose \( G_T \) has a perfect matching \( M_T \). Then all the edges \( M\subseteq M_T \) contained in \( E \) are a matching in \( G \) covering \( T \).

Therefore \( G \) has a matching covering \( T \) if and only if \( G_T \) has a perfect matching.


Let \( W\subseteq V \) and suppose the number of odd components of \( G_T-W_T \) is at most \( |W_T| \) for all \( W_T\subseteq V\cup V' \).

Let \( C \) be an odd component of \( G-W \) contained in \( T \). Then \( C' \) is an odd component of \( G'-W' \) contained in \( T' \). Both \( C \) and \( C' \) are odd components of \( G_T - (W\cup W') \) since being contained in \( T \) means they do not touch an edge in \( L \).

By hypothesis the number of odd components of \( G_T - (W\cup W') \) is at most \( |W\cup W'| = 2|W| \). 
Therefore the number of odd components of \( G - W \) contained in \( T \) is less than \( |W| \).

Let \( W_T\subseteq V\cup V' \) and suppose the number of odd components in \( G - W \) contained in \( T \) is at most \( |W| \) for all \( W\subseteq V \).

Partition \( W_T \) into \( W_1, W_2 \) where \( W_1 = \{ w\in V: w\in W_T \} \) and \( W_2 = \{ w\in V: w'\in W_T \} \).

By hypothesis the number of odd components of \( G - W_1 \) contained in \( T \) is at most \( |W_1| \), and the number of odd components of \( G' - W_2 \) contained in \( T' \) is at most \( |W_2| \).

Edges from \( L \) might connect components of \( G - W_1 \) and \( G' - W_2 \), however the new component will be odd only if one of the original components were odd. That is, the number of odd components in \( G_T - W_T \) is at most \( |W_1|+|W_2| = |W_T| \).

Therefore, for all \( W_T\subseteq V\cup  V' \) is at most \( |W_T| \) if and only if for all \( W\subseteq V \), the number of odd components in \( G - W \) contained in \( T \) is at most \( |W| \).


Using Tutte's 1 factor theorem we now have: \( G \) has a matching covering \( T \) if and only if \( G_T \) has a perfect matching if and only if 
for all \( W_T\subseteq V\cup  V' \) the number of odd components is at most \( |W_T| \) if and only if for all \( W\subseteq V \), the number of odd components in \( G - W \) contained in \( T \) is at most \( |W| \). \qed


\end{solution}
\end{document}
