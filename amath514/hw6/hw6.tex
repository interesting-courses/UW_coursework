\documentclass[10pt]{article}
\usepackage[T1]{fontenc}

% Document Details
\newcommand{\CLASS}{AMATH 514}
\newcommand{\assigmentnum}{Assignment 6}

\usepackage[margin = 1.15in, top = 1.25in, bottom = 1.in]{geometry}

\usepackage{titling}
\setlength{\droptitle}{-6em}   % This is your set screw
\date{}
\renewcommand{\maketitle}{
	\clearpage
	\begingroup  
	\centering
	\LARGE \sffamily\textbf{\CLASS} \Large \assigmentnum\\[.8em]
	\large Tyler Chen\\[1em]
	\endgroup
	\thispagestyle{empty}
}
 % Title Styling
\usepackage{tocloft}
\renewcommand{\cfttoctitlefont}{\Large\sffamily\bfseries}
\renewcommand{\cftsecfont}{\normalfont\sffamily\bfseries}
\renewcommand{\cftsubsecfont}{\normalfont\sffamily}
\renewcommand{\cftsubsubsecfont}{\normalfont\sffamily}

\makeatletter
\let\oldl@section\l@section
\def\l@section#1#2{\oldl@section{#1}{\sffamily\bfseries#2}}

\let\oldl@subsection\l@subsection
\def\l@subsection#1#2{\oldl@subsection{#1}{\sffamily#2}}

\let\oldl@subsubsection\l@subsubsection
\def\l@subsubsection#1#2{\oldl@subsubsection{#1}{\sffamily#2}}
 % General Styling


\usepackage{enumitem}

% Figures
\usepackage{subcaption}

% TikZ and Graphics
\usepackage{tikz, pgfplots}
\pgfplotsset{compat=1.12}
\usetikzlibrary{patterns}
\usepgfplotslibrary{fillbetween}

\usepackage{pdfpages}
\usepackage{adjustbox}

\usepackage{lscape}
\usepackage{titling}
\usepackage[]{hyperref}


% Header Styling
\usepackage{fancyhdr}
\pagestyle{fancy}
\lhead{\sffamily \CLASS}
\rhead{\sffamily \textbf{\thepage}}
\cfoot{}

% Paragraph Styling
\setlength{\columnsep}{1cm}
\setlength{\parindent}{0pt}
\setlength{\parskip}{5pt}
\renewcommand{\baselinestretch}{1}

% TOC Styling
\usepackage{tocloft}
\iffalse
\renewcommand{\cftsecleader}{\cftdotfill{\cftdotsep}}

\renewcommand\cftchapafterpnum{\vskip6pt}
\renewcommand\cftsecafterpnum{\vskip10pt}
\renewcommand\cftsubsecafterpnum{\vskip6pt}

% Adjust sectional unit title fonts in ToC
\renewcommand{\cftchapfont}{\sffamily}
\renewcommand{\cftsecfont}{\bfseries\sffamily}
\renewcommand{\cftsecnumwidth}{2em}
\renewcommand{\cftsubsecfont}{\sffamily}
\renewcommand{\cfttoctitlefont}{\hfill\bfseries\sffamily\MakeUppercase}
\renewcommand{\cftaftertoctitle}{\hfill}

\renewcommand{\cftchappagefont}{\sffamily}
\renewcommand{\cftsecpagefont}{\bfseries\sffamily}
\renewcommand{\cftsubsecpagefont}{\sffamily}
\fi
 % General Styling
% Code Display Setup
\usepackage{listings,lstautogobble}
\usepackage{lipsum}
\usepackage{courier}
\usepackage{catchfilebetweentags}

\lstset{
	basicstyle=\small\ttfamily,
	breaklines=true, 
	frame = single,
	rangeprefix=,
	rangesuffix=,
	includerangemarker=false,
	autogobble = true
}


\usepackage{algorithmicx}
\usepackage{algpseudocode}

\newcommand{\To}{\textbf{to}~}
\newcommand{\DownTo}{\textbf{downto}~}
\renewcommand{\algorithmicdo}{\hspace{-.2em}\textbf{:}}
 % Code Display Setup
% AMS MATH Styling
\usepackage{amsmath, amssymb}
\newcommand{\qed}{\hfill\(\square\)}

%\newtheorem*{lemma}{Lemma} 
%\newtheorem*{theorem}{Theorem}
%\newtheorem*{definition}{Definition}
%\newtheorem*{prop}{Proposition}
%\renewenvironment{proof}{{\bfseries Proof.}}{}


% mathcal
\newcommand{\cA}{\ensuremath{\mathcal{A}}}
\newcommand{\cB}{\ensuremath{\mathcal{B}}}
\newcommand{\cC}{\ensuremath{\mathcal{C}}}
\newcommand{\cD}{\ensuremath{\mathcal{D}}}
\newcommand{\cE}{\ensuremath{\mathcal{E}}}
\newcommand{\cF}{\ensuremath{\mathcal{F}}}
\newcommand{\cG}{\ensuremath{\mathcal{G}}}
\newcommand{\cH}{\ensuremath{\mathcal{H}}}
\newcommand{\cI}{\ensuremath{\mathcal{I}}}
\newcommand{\cJ}{\ensuremath{\mathcal{J}}}
\newcommand{\cK}{\ensuremath{\mathcal{K}}}
\newcommand{\cL}{\ensuremath{\mathcal{L}}}
\newcommand{\cM}{\ensuremath{\mathcal{M}}}
\newcommand{\cN}{\ensuremath{\mathcal{N}}}
\newcommand{\cO}{\ensuremath{\mathcal{O}}}
\newcommand{\cP}{\ensuremath{\mathcal{P}}}
\newcommand{\cQ}{\ensuremath{\mathcal{Q}}}
\newcommand{\cR}{\ensuremath{\mathcal{R}}}
\newcommand{\cS}{\ensuremath{\mathcal{S}}}
\newcommand{\cT}{\ensuremath{\mathcal{T}}}
\newcommand{\cU}{\ensuremath{\mathcal{U}}}
\newcommand{\cV}{\ensuremath{\mathcal{V}}}
\newcommand{\cW}{\ensuremath{\mathcal{W}}}
\newcommand{\cX}{\ensuremath{\mathcal{X}}}
\newcommand{\cY}{\ensuremath{\mathcal{Y}}}
\newcommand{\cZ}{\ensuremath{\mathcal{Z}}}

% mathbb
\usepackage{bbm}
\newcommand{\bOne}{\ensuremath{\mathbbm{1}}}

\newcommand{\bA}{\ensuremath{\mathbb{A}}}
\newcommand{\bB}{\ensuremath{\mathbb{B}}}
\newcommand{\bC}{\ensuremath{\mathbb{C}}}
\newcommand{\bD}{\ensuremath{\mathbb{D}}}
\newcommand{\bE}{\ensuremath{\mathbb{E}}}
\newcommand{\bF}{\ensuremath{\mathbb{F}}}
\newcommand{\bG}{\ensuremath{\mathbb{G}}}
\newcommand{\bH}{\ensuremath{\mathbb{H}}}
\newcommand{\bI}{\ensuremath{\mathbb{I}}}
\newcommand{\bJ}{\ensuremath{\mathbb{J}}}
\newcommand{\bK}{\ensuremath{\mathbb{K}}}
\newcommand{\bL}{\ensuremath{\mathbb{L}}}
\newcommand{\bM}{\ensuremath{\mathbb{M}}}
\newcommand{\bN}{\ensuremath{\mathbb{N}}}
\newcommand{\bO}{\ensuremath{\mathbb{O}}}
\newcommand{\bP}{\ensuremath{\mathbb{P}}}
\newcommand{\bQ}{\ensuremath{\mathbb{Q}}}
\newcommand{\bR}{\ensuremath{\mathbb{R}}}
\newcommand{\bS}{\ensuremath{\mathbb{S}}}
\newcommand{\bT}{\ensuremath{\mathbb{T}}}
\newcommand{\bU}{\ensuremath{\mathbb{U}}}
\newcommand{\bV}{\ensuremath{\mathbb{V}}}
\newcommand{\bW}{\ensuremath{\mathbb{W}}}
\newcommand{\bX}{\ensuremath{\mathbb{X}}}
\newcommand{\bY}{\ensuremath{\mathbb{Y}}}
\newcommand{\bZ}{\ensuremath{\mathbb{Z}}}

% alternative mathbb
\newcommand{\NN}{\ensuremath{\mathbb{N}}}
\newcommand{\RR}{\ensuremath{\mathbb{R}}}
\newcommand{\CC}{\ensuremath{\mathbb{C}}}
\newcommand{\ZZ}{\ensuremath{\mathbb{Z}}}
\newcommand{\EE}{\ensuremath{\mathbb{E}}}
\newcommand{\PP}{\ensuremath{\mathbb{P}}}
\newcommand{\VV}{\ensuremath{\mathbb{V}}}
\newcommand{\cov}{\ensuremath{\text{Co}\VV}}
% Math Commands

\newcommand{\st}{~\big|~}
\newcommand{\stt}{\text{ st. }}
\newcommand{\ift}{\text{ if }}
\newcommand{\thent}{\text{ then }}
\newcommand{\owt}{\text{ otherwise }}

\newcommand{\norm}[1]{\left\lVert#1\right\rVert}
\newcommand{\snorm}[1]{\lVert#1\rVert}
\newcommand{\ip}[1]{\ensuremath{\left\langle #1 \right\rangle}}
\newcommand{\pp}[3][]{\frac{\partial^{#1}#2}{\partial #3^{#1}}}
\newcommand{\dd}[3][]{\frac{\d^{#1}#2}{\d #3^{#1}}}
\renewcommand{\d}{\ensuremath{\mathrm{d}}}

\newcommand{\indep}{\rotatebox[origin=c]{90}{$\models$}}




 % Math shortcuts
% Problem
\usepackage{floatrow}

\newenvironment{problem}[1][]
{\pagebreak
\noindent\rule{\textwidth}{1pt}\vspace{0.25em}
{\sffamily \textbf{#1}}
\par
}
{\par\vspace{-0.5em}\noindent\rule{\textwidth}{1pt}}

\newenvironment{solution}[1][]
{{\sffamily \textbf{#1}}
\par
}
{}

 % Problem Environment

\newcommand{\note}[1]{\textcolor{red}{\textbf{Note:} #1}}

\hypersetup{
   colorlinks=true,       % false: boxed links; true: colored links
   linkcolor=violet,          % color of internal links (change box color with linkbordercolor)
   citecolor=green,        % color of links to bibliography
   filecolor=magenta,      % color of file links
   urlcolor=cyan           % color of external links
}


\begin{document}
\maketitle



\begin{problem}[Problem 8.8]
Let \( A \) be a totally unimodular matrix. Show that the columns of \( A \) can be split into two classes such that the sum of the columns in one class, minus the sum of the columns in the other class, gives a vector with entries \( 0 \), \( +1 \), and \( -1 \) only.
\end{problem}

\begin{solution}

Let \( e \) be the vector of all ones. Let \( b = \lfloor \frac{1}{2} (Ae+1) \rfloor \) and \( b' = \lfloor \frac{1}{2} (1-Ae) \rfloor \).  Define a polytope,
\begin{align*}
    P = \left\{ x: \left[\begin{array}{r}A \\ -A \\ I \\ -I \end{array}\right]x \leq \left[\begin{array}{l}b \\ b' \\ 1 \\0\end{array}\right] \right\}
\end{align*}

Then \( P \) is bounded as \( x_i\in[0,1] \) for all \( i \). Moreover, for all integers, \( 2k,2k+1 \),
\begin{align*}
    \left\lfloor ((2k)+1)/2  \right\rfloor = \lfloor k+ 1/2 \rfloor = k \geq (2k)/2,
    && \left\lfloor ((2k+1)+1)/2  \right\rfloor = \lfloor k+1 \rfloor = k+1 \geq (2k+1)/2,
    \\\left\lfloor (1-(2k))/2  \right\rfloor = \lfloor 1/2-k \rfloor = -k \geq - (2k)/2,
    && \left\lfloor (1-(2k+1))/2  \right\rfloor = \lfloor -k \rfloor = -k \geq -(2k+1)/2
\end{align*}
Therefore, since \( Ae \) is an integer, \( \frac{1}{2} e \in P \). 

Since \( P \) is nonempty and bounded \( P \) has a vertex \( v \). The matrix \( [A; -A; I; -I] \) is totally unimodular since \( A \) is totally unimodular meaning \( v \) is integer. In particular, this means \( v_i\in\{0,1\} \) for all \( i \) and, since \( v\in P \),
\begin{align*}
    Av \leq b = \left\lfloor \frac{1}{2} (Ae+1) \right\rfloor \leq \frac{1}{2} (Ae+1) && \Longrightarrow &&
    Ae - 2Av &\geq - 1 \\
    -Av \leq b' = \left\lfloor \frac{1}{2} (1-Ae) \right\rfloor \leq \frac{1}{2} (1-Ae) && \Longrightarrow &&
    Ae - 2Av &\leq 1
\end{align*}

Now define \( z = 1-2v \). Clearly \( z \) is integer with entries in \( \{-1,1\} \). Therefore \( Az \) is integer as \( A \) and \( z \) are each integer. Moreover, since \( Az = A(e-2v) = Ae-2Av \), by above we have, \( -1\leq Az\leq 1 \). Together these mean \( Az \) has entries in \( \{-1,0,1\} \).

Finally take one class as the rows corresponding to \( 1 \) entries in \( z \) and the other class corresponding to \( -1 \) entries in \( z \). Then the result is proved. \qed

\textit{I got a hint online to use these floor functions, but derived the proof without more.}

\iffalse
Define a polyhedra,
\begin{align*}
    P = \left\{ x : \left[\begin{array}{r}A\\-A\\I\\-I\end{array}\right]x \leq \left[\begin{array}{r}1\\1\\1\\1\end{array}\right] \right\}
\end{align*}

Note that \( -1\leq x\leq 1 \) so that \( P \) is bounded. Moreover, and since \( 0\in P \), \( P \) is nonempty.

Therefore \( P \) has vertices, which are integer if \( A \) is totally unimodular.

We claim there is a vertex of \( P \) with no zero-entries.

Indeed, let \( v \) be a vertex of \( P \). Then for all \( i \), \( -1\leq v_i \leq 1\) and \( v_i\in \ZZ \) so \( v_i\in\{-1,0,1\} \). If \( v_i\in\{-1,1\} \) for all \( i \) we are done.

Suppose \( v_i = 0 \) for some \( i \).
\fi


\end{solution}

\begin{problem}[Problem 8.9]
Let \( A \) be a totally unimodular matrix and let \( b \) be an integer vector. Let \( x \) be an integer vector satisfying \( x\geq 0; Ax \leq 2b \). Show that there exists integer vectors \( x'\geq 0 \) and \( x''\geq 0 \) such that \( Ax'\leq b \), \( Ax''\leq b \) and \( x = x'+x'' \).
\end{problem}

\begin{solution}

Define,
\begin{align*}
    P = \left\{ z : \left[\begin{array}{r}A\\I\\-I\\-A\end{array}\right]z \leq \left[\begin{array}{l} b \\ x \\ 0 \\ b-Ax\end{array}\right] \right\}
\end{align*}

Clearly \( P \) is bounded. We have \( A(x/2) = (Ax)/2 \leq 2b/2 = b \). Then \( A(x-x/2) \leq b  \) so \( -A(x/2) \leq Ax-b \). Clearly \( 0\leq x/2 \leq x \). Therefore \( x/2\in P \).

Since \( P \) is nonempty and bounded \( P \) has a vertex \( x' \). The matrix \( [A;I;-I;-A] \) is totally unimodular since \( A \) is totally unimodular meaning \( x' \) is integer.

Define \( x'' = x-x' \). Since \( x'\in P \) and \( x \) is integer we have \( x'' \) integer with \( 0\leq x'' \leq x \). Moreover, since \( -Ax' \leq b-Ax \) we have \( Ax-Ax' \leq b \) so that \( Ax'' = A(x-x') \leq  b \). \qed

\end{solution}

\begin{problem}[Problem 4.15]
    Let \( D = (V,A) \) be a directed graph, and let \( f:A\to\RR_+ \). Let \( \mathcal{C} \) be the collection of directed circuits in \( D \). For each directed circuit \( C \) in \( D \) let \( \chi^C \) be the incidence vector of \( C \). That is, \( \chi^C:A\to\{0,1\} \), with \( \chi^C(a) = 1 \) if \( C \) transverses \( a \) and \( \chi^C(a) = 0 \) otherwise.

Show that \( f \) is a non-negative circulation if and only if there exists a function \( \lambda: \mathcal{C}  \to \RR_+ \) such that,
\begin{align*}
    f = \sum_{C\in \mathcal{C}} \lambda(C) \chi^C
\end{align*}
That is, the non-negative circulations form the code generated by \( \{ \chi^C : C\in \mathcal{C} \} \).
\end{problem}

\begin{solution}

Fix \( \lambda:A\to\RR_+ \) and let \( f = \sum_{C\in\mathcal{C}} \lambda(C) \chi^C \). Consider the flux into and out of a vertex \( v\in V \). We have,
\begin{align*}
    \sum_{a\in \delta^{\text{out}}(v) } f(a) - \sum_{a\in \delta^{\text{in}(a)}} f(a) &= 
    \sum_{a\in \delta^{\text{out}}(v) } \sum_{C\in \mathcal{C} } \lambda(C)\chi^C(a) - \sum_{a\in \delta^{\text{in}(a)}} \sum_{C\in \mathcal{C}} \lambda(c) \chi^C(a) 
    \\ &= 
    \sum_{C\in \mathcal{C} } \left[
        \sum_{a\in \delta^{\text{out}}(v) } \lambda(C)\chi^C(a) - \sum_{a\in \delta^{\text{in}(a)}} \lambda(C) \chi^C(a) = 
        \right]
\end{align*}

Fix \( C\in \mathcal{C} \). If \( C \) does not pass through \( v \) then \( \chi^C(a) = 0 \) for all \( a\in \delta^{\text{in}}(v)\cup \delta^{\text{out}}(v) \). If \( C \) does pass through \( v \), then \( \chi^C(a) = 1 \) for exactly one \( a\in \delta^{\text{in}}(v) \) and exactly one \( a\in\delta^{\text{out}}(v) \). Moreover, since \( \lambda(C) \) is constant (if \( C \) is fixed), then the term \( \lambda(C) \chi^{C}(a) \) appears in both sums. 

Therefore the difference of the two sums is zero. This proves \( f \) is a circulation.

We provide an algorithm to find \( \lambda : A\to\RR_+ \) such that \( f = \sum_{C\in\mathcal{C}} \lambda(C)\chi^C\) for a non-negative circulation \( f \).

At the \( k \)-th step, start with a circulation \( f^{[k-1]} \). If the circulation on each edge of every directed circuit in \( D \) is zero then terminate.

Otherwise, at step \( k \) find a directed circuit \( C_k \) with \( f(a) \neq 0 \) for all \( a\in C_k \). Define,
\begin{align*}
    \lambda(C_k) = \min_{a\in C_k} f(a) 
\end{align*}
Now, define a new circulation \( f^{[k]}:A\to\RR_+ \) by,
\begin{align*}
    f^{[k]}(a) = \begin{cases}
        f^{[k-1]}(a) - \lambda(C_k) & a\in C_k \\
        f^{[k-1]}(a) & \text{ otherwise}
    \end{cases}
\end{align*}

Then clearly \( f^{[k]} \) is a circulation. Moreover, \( f^{[k]} \) has at least one fewer non-zero edge than \( f^{[k-1]} \) since \( f^{[k-1]}(a) = \lambda(C_k) \) for some \( a\in C_k \). Since \( |A|<\infty \) this means the algorithm will terminate (in less than \( |A| \) steps).

Then, starting with \( f^{[0]} = f \) the algorithm will terminate and give us \( \lambda(C_k) \) such that,
\begin{align*}
    f = \sum_{k} \lambda(C_k)\chi^{C_k}(a)
\end{align*}

\end{solution}

\end{document}
