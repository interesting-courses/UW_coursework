\documentclass[10pt]{article}
\usepackage[T1]{fontenc}

% Document Details
\newcommand{\CLASS}{AMATH 514}
\newcommand{\assigmentnum}{Assignment 2}

\usepackage[margin = 1.15in, top = 1.25in, bottom = 1.in]{geometry}

\usepackage{titling}
\setlength{\droptitle}{-6em}   % This is your set screw
\date{}
\renewcommand{\maketitle}{
	\clearpage
	\begingroup  
	\centering
	\LARGE \sffamily\textbf{\CLASS} \Large \assigmentnum\\[.8em]
	\large Tyler Chen\\[1em]
	\endgroup
	\thispagestyle{empty}
}
 % Title Styling
\usepackage{tocloft}
\renewcommand{\cfttoctitlefont}{\Large\sffamily\bfseries}
\renewcommand{\cftsecfont}{\normalfont\sffamily\bfseries}
\renewcommand{\cftsubsecfont}{\normalfont\sffamily}
\renewcommand{\cftsubsubsecfont}{\normalfont\sffamily}

\makeatletter
\let\oldl@section\l@section
\def\l@section#1#2{\oldl@section{#1}{\sffamily\bfseries#2}}

\let\oldl@subsection\l@subsection
\def\l@subsection#1#2{\oldl@subsection{#1}{\sffamily#2}}

\let\oldl@subsubsection\l@subsubsection
\def\l@subsubsection#1#2{\oldl@subsubsection{#1}{\sffamily#2}}
 % General Styling


\usepackage{enumitem}

% Figures
\usepackage{subcaption}

% TikZ and Graphics
\usepackage{tikz, pgfplots}
\pgfplotsset{compat=1.12}
\usetikzlibrary{patterns}
\usepgfplotslibrary{fillbetween}

\usepackage{pdfpages}
\usepackage{adjustbox}

\usepackage{lscape}
\usepackage{titling}
\usepackage[]{hyperref}


% Header Styling
\usepackage{fancyhdr}
\pagestyle{fancy}
\lhead{\sffamily \CLASS}
\rhead{\sffamily \textbf{\thepage}}
\cfoot{}

% Paragraph Styling
\setlength{\columnsep}{1cm}
\setlength{\parindent}{0pt}
\setlength{\parskip}{5pt}
\renewcommand{\baselinestretch}{1}

% TOC Styling
\usepackage{tocloft}
\iffalse
\renewcommand{\cftsecleader}{\cftdotfill{\cftdotsep}}

\renewcommand\cftchapafterpnum{\vskip6pt}
\renewcommand\cftsecafterpnum{\vskip10pt}
\renewcommand\cftsubsecafterpnum{\vskip6pt}

% Adjust sectional unit title fonts in ToC
\renewcommand{\cftchapfont}{\sffamily}
\renewcommand{\cftsecfont}{\bfseries\sffamily}
\renewcommand{\cftsecnumwidth}{2em}
\renewcommand{\cftsubsecfont}{\sffamily}
\renewcommand{\cfttoctitlefont}{\hfill\bfseries\sffamily\MakeUppercase}
\renewcommand{\cftaftertoctitle}{\hfill}

\renewcommand{\cftchappagefont}{\sffamily}
\renewcommand{\cftsecpagefont}{\bfseries\sffamily}
\renewcommand{\cftsubsecpagefont}{\sffamily}
\fi
 % General Styling
% Code Display Setup
\usepackage{listings,lstautogobble}
\usepackage{lipsum}
\usepackage{courier}
\usepackage{catchfilebetweentags}

\lstset{
	basicstyle=\small\ttfamily,
	breaklines=true, 
	frame = single,
	rangeprefix=,
	rangesuffix=,
	includerangemarker=false,
	autogobble = true
}


\usepackage{algorithmicx}
\usepackage{algpseudocode}

\newcommand{\To}{\textbf{to}~}
\newcommand{\DownTo}{\textbf{downto}~}
\renewcommand{\algorithmicdo}{\hspace{-.2em}\textbf{:}}
 % Code Display Setup
% AMS MATH Styling
\usepackage{amsmath, amssymb}
\newcommand{\qed}{\hfill\(\square\)}

%\newtheorem*{lemma}{Lemma} 
%\newtheorem*{theorem}{Theorem}
%\newtheorem*{definition}{Definition}
%\newtheorem*{prop}{Proposition}
%\renewenvironment{proof}{{\bfseries Proof.}}{}


% mathcal
\newcommand{\cA}{\ensuremath{\mathcal{A}}}
\newcommand{\cB}{\ensuremath{\mathcal{B}}}
\newcommand{\cC}{\ensuremath{\mathcal{C}}}
\newcommand{\cD}{\ensuremath{\mathcal{D}}}
\newcommand{\cE}{\ensuremath{\mathcal{E}}}
\newcommand{\cF}{\ensuremath{\mathcal{F}}}
\newcommand{\cG}{\ensuremath{\mathcal{G}}}
\newcommand{\cH}{\ensuremath{\mathcal{H}}}
\newcommand{\cI}{\ensuremath{\mathcal{I}}}
\newcommand{\cJ}{\ensuremath{\mathcal{J}}}
\newcommand{\cK}{\ensuremath{\mathcal{K}}}
\newcommand{\cL}{\ensuremath{\mathcal{L}}}
\newcommand{\cM}{\ensuremath{\mathcal{M}}}
\newcommand{\cN}{\ensuremath{\mathcal{N}}}
\newcommand{\cO}{\ensuremath{\mathcal{O}}}
\newcommand{\cP}{\ensuremath{\mathcal{P}}}
\newcommand{\cQ}{\ensuremath{\mathcal{Q}}}
\newcommand{\cR}{\ensuremath{\mathcal{R}}}
\newcommand{\cS}{\ensuremath{\mathcal{S}}}
\newcommand{\cT}{\ensuremath{\mathcal{T}}}
\newcommand{\cU}{\ensuremath{\mathcal{U}}}
\newcommand{\cV}{\ensuremath{\mathcal{V}}}
\newcommand{\cW}{\ensuremath{\mathcal{W}}}
\newcommand{\cX}{\ensuremath{\mathcal{X}}}
\newcommand{\cY}{\ensuremath{\mathcal{Y}}}
\newcommand{\cZ}{\ensuremath{\mathcal{Z}}}

% mathbb
\usepackage{bbm}
\newcommand{\bOne}{\ensuremath{\mathbbm{1}}}

\newcommand{\bA}{\ensuremath{\mathbb{A}}}
\newcommand{\bB}{\ensuremath{\mathbb{B}}}
\newcommand{\bC}{\ensuremath{\mathbb{C}}}
\newcommand{\bD}{\ensuremath{\mathbb{D}}}
\newcommand{\bE}{\ensuremath{\mathbb{E}}}
\newcommand{\bF}{\ensuremath{\mathbb{F}}}
\newcommand{\bG}{\ensuremath{\mathbb{G}}}
\newcommand{\bH}{\ensuremath{\mathbb{H}}}
\newcommand{\bI}{\ensuremath{\mathbb{I}}}
\newcommand{\bJ}{\ensuremath{\mathbb{J}}}
\newcommand{\bK}{\ensuremath{\mathbb{K}}}
\newcommand{\bL}{\ensuremath{\mathbb{L}}}
\newcommand{\bM}{\ensuremath{\mathbb{M}}}
\newcommand{\bN}{\ensuremath{\mathbb{N}}}
\newcommand{\bO}{\ensuremath{\mathbb{O}}}
\newcommand{\bP}{\ensuremath{\mathbb{P}}}
\newcommand{\bQ}{\ensuremath{\mathbb{Q}}}
\newcommand{\bR}{\ensuremath{\mathbb{R}}}
\newcommand{\bS}{\ensuremath{\mathbb{S}}}
\newcommand{\bT}{\ensuremath{\mathbb{T}}}
\newcommand{\bU}{\ensuremath{\mathbb{U}}}
\newcommand{\bV}{\ensuremath{\mathbb{V}}}
\newcommand{\bW}{\ensuremath{\mathbb{W}}}
\newcommand{\bX}{\ensuremath{\mathbb{X}}}
\newcommand{\bY}{\ensuremath{\mathbb{Y}}}
\newcommand{\bZ}{\ensuremath{\mathbb{Z}}}

% alternative mathbb
\newcommand{\NN}{\ensuremath{\mathbb{N}}}
\newcommand{\RR}{\ensuremath{\mathbb{R}}}
\newcommand{\CC}{\ensuremath{\mathbb{C}}}
\newcommand{\ZZ}{\ensuremath{\mathbb{Z}}}
\newcommand{\EE}{\ensuremath{\mathbb{E}}}
\newcommand{\PP}{\ensuremath{\mathbb{P}}}
\newcommand{\VV}{\ensuremath{\mathbb{V}}}
\newcommand{\cov}{\ensuremath{\text{Co}\VV}}
% Math Commands

\newcommand{\st}{~\big|~}
\newcommand{\stt}{\text{ st. }}
\newcommand{\ift}{\text{ if }}
\newcommand{\thent}{\text{ then }}
\newcommand{\owt}{\text{ otherwise }}

\newcommand{\norm}[1]{\left\lVert#1\right\rVert}
\newcommand{\snorm}[1]{\lVert#1\rVert}
\newcommand{\ip}[1]{\ensuremath{\left\langle #1 \right\rangle}}
\newcommand{\pp}[3][]{\frac{\partial^{#1}#2}{\partial #3^{#1}}}
\newcommand{\dd}[3][]{\frac{\d^{#1}#2}{\d #3^{#1}}}
\renewcommand{\d}{\ensuremath{\mathrm{d}}}

\newcommand{\indep}{\rotatebox[origin=c]{90}{$\models$}}




 % Math shortcuts
% Problem
\usepackage{floatrow}

\newenvironment{problem}[1][]
{\pagebreak
\noindent\rule{\textwidth}{1pt}\vspace{0.25em}
{\sffamily \textbf{#1}}
\par
}
{\par\vspace{-0.5em}\noindent\rule{\textwidth}{1pt}}

\newenvironment{solution}[1][]
{{\sffamily \textbf{#1}}
\par
}
{}

 % Problem Environment

\newcommand{\note}[1]{\textcolor{red}{\textbf{Note:} #1}}

\hypersetup{
   colorlinks=true,       % false: boxed links; true: colored links
   linkcolor=violet,          % color of internal links (change box color with linkbordercolor)
   citecolor=green,        % color of links to bibliography
   filecolor=magenta,      % color of file links
   urlcolor=cyan           % color of external links
}

\newcommand{\convhull}{\operatorname{convhull}}


\begin{document}
\maketitle


\begin{problem}[Exercise 2.2]
    Let \( C \subseteq \RR^n \) be a convex set and let \( A \) be a \( m\times n \) matrix. Show that the set \( \{Ax \st x\in C \} \) is again convex.
\end{problem}

\begin{solution}

Any two points in \( C \) can be written as \( Au \) and \( Av \) for some \( u,v,\in C \). Since \( C \) is convex, \( \forall \lambda\in[0,1]\), \( \lambda u + (1-\lambda)v \in C \). Thus,
\begin{align*}
    \lambda Au + (1-\lambda)Av = A(\lambda u+ (1-\lambda)v) \in \{Ax \st x\in C \}, && \forall  \lambda\in[0,1] 
\end{align*}

That is, \( \{Ax \st x\in C\} \) is convex. \qed

\end{solution}

\begin{problem}[Exercise 2.4]
    Show that if \( z\in\convhull(X) \), then there exist affinely independent vectors \( x_1, ..., x_m \) in \( X \) such that \( z\in \convhull\{x_1, ..., x_m\} \). 
\end{problem}

\begin{solution}

\iffalse
\begin{lemma}
Suppose \( x_1, \ldots, x_m \) are affinely independent vectors. Then \( cx_1, \ldots, cx_m \) are also affinely independent vectors for any \( c\neq 0 \).
\end{lemma}

Suppose \( x_1, \ldots, x_m \) are affinely independent, and let \( c\in\RR\setminus\{0\} \). Suppose \( \lambda_1 c x_1 + \cdots + \lambda_m c x_m = 0
 \) and \( \lambda_1 + \cdots + \lambda_m = 0 \). Then \( \lambda_1 c+ \cdots \lambda_m c = (\lambda_1 + \cdots + \lambda_m)c = 0 \), so by the affine independence of \( x_1, \ldots, x_m \), we have \( \lambda_1 = \cdots = \lambda_m = 0 \). That is, \( cx_1, \ldots, cx_m \) are affinely independent. \qed

\fi
Let \( z\in\convhull(X) \). Then for some \( t\in\NN \), \( x_1, \ldots, x_t\in X \), and \( c_1, \ldots,c_t \geq 0 \) we can write,
\begin{align*}
    z = c_1 x_1 + \cdots + c_t x_t && \text{where} && c_1 + \cdots + c_t=1
\end{align*}

WLOG assume \( c_1, \ldots , c_t > 0 \) since we can drop any \( x_j \) for which \( c_j=0 \).

If \( x_1, \ldots, x_t \) are affinely independent we are done. If not, then there are coefficients \( d_1, \ldots, d_t \) not all zero such that,
\begin{align*}
    d_1x_1 + \cdots + d_tx_t = 0 && \text{where} && d_1+\cdots d_t = 0
\end{align*}

Observe that for any \( \mu >0 \), since \( \sum_{j=1}^{t}(c_j - \mu c_j) = \sum_{j=1}^{t}c_j - \mu \sum_{j=1}^{t}d_j = 1 - \mu (0) = 1 \),
\begin{align*}
    z = (c_1 - \mu d_1)x_1 + \cdots + (c_t - \mu d_t)x_t && \text{where} && \sum_{j=1}^{t} (c_j - \mu d_j) = 1
\end{align*}

Note that for \( \mu = 0 \) we have the original combination for \( z \), and each \( (c_j-\mu d_j) = c_j > 0 \). 

Since the \( \sum_{j}^{}d_j =0  \) and not all \( d_j \) are zero, at least one \( d_j>0 \). By hypothesis, \( c_j>0 \). Thus, \( \{c_j/d_j \st 1\leq j\leq t, c_j/d_j > 0 \} \) is nonempty. Pick \( \mu = c_i / d_i  \) where \( i=\operatorname{argmin}\{ c_j/d_j \st 1\leq j\leq t , c_j/d_j > 0\} \). Note \( \mu > 0 \).

If \( d_j < 0 \), then \( \mu d_j < 0 \) so \( c_j - \mu d_j > 0 \).

If \( d_j \geq 0 \), then since \( (c_j-(c_i/d_i)d_j)/d_j = c_j/d_j - c_i/d_i \geq 0 \),
\begin{align*}
    c_j - \mu d_j = c_j - (c_i/d_i )d_j \geq c_i - (c_i/d_i) d_i = 1  
\end{align*}

So the coefficients for this combination of \( \{x_1, \ldots, x_t\}\setminus \{x_i\} \) are all non-negative with sum equal to one.

Thus \( z\in\convhull(\{x_1, \ldots, x_t \}\setminus\{x_i\}) \). Since the (finite) set generating the convex hull has fewer elements we can repeat this procedure until we have a finite affinely independent generating set. \qed

\textit{Note: I got a bit of inspiration from stackexchange when looking up Carath\'eodory's theorem. However, I wrote my solution without looking back on that page.}

\end{solution}

\begin{problem}[Exercise 2.5]
    \begin{enumerate}
        \item[(i)] Let \( C \) and \( D \) be two nonempty, bounded, closed, convex subsets of \( \RR^n \) such that \( C\cap D = \emptyset \). Derive from Theorem 2.1 that there exists and affine hyperplane \( H \) seperating \( C \) and \( D \). 
            % hint: consider the set C-D:= \{ x-y \st x\in C, y\in D\} 
        \item[(ii)] Show that in (i) we cannot delete the boundedness condition.
    \end{enumerate}
\end{problem}

\begin{solution}

\begin{enumerate}
    \item[(i)] 
        The function \( l: C\times D \to \RR_{\geq 0} \) defined as \(  l(x,y) := \norm{x-y} \) is a continuous function (norm on \( \RR^{2n} \supset C\times D \) is continuous) on a compact set (Cartesian product of compact sets is compact). Therefore, \( (c,d) = \operatorname{argmin}\{ l(x,y) \st x\in C, y\in D \} \) exists in \( C\times D \).

        Note then that \( d \) satisfies \( d = \operatorname{argmin}\{\norm{d-c} \st d\in D \} \), and \( c \) satisfies \( c=\operatorname{argmin}\{\norm{c-d} \st c\in C\} \).

        Thus, by Theorem \( 2.1 \), \( d \) is separated from \( C \) by the hyperplane \( H_d = \{ x \st (d-c)^Tx = \delta \}  \), and \( c \) is separated from \( D \) by the hyperplane \( H_c = \{x \st (c-d)^Tx = \delta \} \), where \( \delta = (\norm{c}^2-\norm{d}^2)/2 \) for both hyperplanes.
        That is, \( (d-c)^T x < \delta  \) for all \( x\in C \), and \( (c-d)^Tx < \delta \) for all \( x\in D \). 
        
        But then, \( (d-c)^T x > \delta \) for all \( x\in D \). This proves \( H = H_d \) separates \( C \) from \( D \). \qed
     


    \item[(ii)] Let \( C = \{ (x,y) \st y\leq 0 \} \) and \( D = \{ (x,y) \st y \geq 1/x, x > 0 \} \). Then both sets are convex, closed, and unbounded.
       
        Clearly any line (hyperplane in \( \RR^2 \)) which is not horizontal will intersect \( C \). But any horizontal line will intersect \( D \) if it is above the \( x \) axis, or intersect \( C \) if it is on or below the \( x \) axis. Therefore, there is no line which separates \( C \) from \( D \) since such a line cannot intersect either \( C \) or \( D \).

        More rigorously, write \( H = \{(x,y) \st (c_1,c_2)^T(x,y) = \delta \} = \{(x,y) \st c_1x+c_2y = \delta \} \).

        Suppose \( c_1 \neq 0 \). Then the points \( ((\delta+c_2+1)/c_1,-1) \) and \( ((\delta+c_2-1)/c_1,-1) \) are both in \( C \). However, 
        \begin{align*}
            (c_1,c_2)^T((\delta+c_2+1)/c_1,-1) = c_1(\delta + c_2 + 1)/c_1 + c_2(-1) = \delta + 1 > \delta \\
            (c_1,c_2)^T((\delta+c_2-1)/c_1,-1) = c_1(\delta + c_2 - 1)/c_1 + c_2(-1) = \delta - 1 < \delta 
        \end{align*}
        So the points \( ((\delta+c_2+1)/c_1,-1) \) and \( ((\delta+c_2-1)/c_1,-1) \) are on opposite sides of \( H \).

        Suppose \( c_1 = 0 \) and \( \delta/c_2 \leq 0 \). Then the points \( (0,0) \) and \( (0,\delta/c_2-1) \) are both in \( C \). However,
        \begin{align*}
            (0,c_2)^T(0,0) = 0 \\
            (0,c_2)^T(0,\delta/c_2-1) = c_2(\delta/c_2-1) = \delta - c_2
        \end{align*}
        If \( \delta < 0 \) then \( c_2 > 0 \) so \( \delta - c_2 < \delta \). If \( \delta \geq 0 \) then \( c_2 < 0 \) so \( \delta - c_2 > \delta \). Then the points \( (0,0) \) and \( (\delta/c_2-1) \) are on opposite sides of \( H \).

        Suppose \( c_1 = 0 \) and \( \delta/c_2 > 0 \). Then the points \( (2c_2/\delta, \delta/2c_2)  \) and \( (2c_2/\delta, 2\delta/c_2) \) are both in \( D \). However, 
        \begin{align*}
            (0,c_2)^T(2c_2/\delta,\delta/2c_2) = \delta/2 \\
            (0,c_2)^T(2c_2/\delta,2\delta/c_2) = 2\delta
        \end{align*}
        If \( \delta < 0 \) then \( \delta/2 > \delta \) and \( 2\delta < \delta \). If \( \delta > 0 \) then \( \delta/2 < \delta \) and \( 2\delta > \delta \). Then the points \( (2c_2/\delta, \delta/2c_2)  \) and \( (2c_2/\delta, 2\delta/c_2) \) are on opposites sides of \( H \).

        This shows that there is no hyperplane \( H \) which separates \( C \) from \( D \), as every hyperplane will split either \( C \) or \( D \).
        

        We therefore need the boundedness condition to have a strict separating hyperplane. \qed

\end{enumerate}
\end{solution}

\end{document}
