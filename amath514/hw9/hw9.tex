\documentclass[10pt]{article}
\usepackage[T1]{fontenc}

% Document Details
\newcommand{\CLASS}{AMATH 514}
\newcommand{\assigmentnum}{Assignment 9}

\usepackage[margin = 1.15in, top = 1.25in, bottom = 1.in]{geometry}

\usepackage{titling}
\setlength{\droptitle}{-6em}   % This is your set screw
\date{}
\renewcommand{\maketitle}{
	\clearpage
	\begingroup  
	\centering
	\LARGE \sffamily\textbf{\CLASS} \Large \assigmentnum\\[.8em]
	\large Tyler Chen\\[1em]
	\endgroup
	\thispagestyle{empty}
}
 % Title Styling
\usepackage{tocloft}
\renewcommand{\cfttoctitlefont}{\Large\sffamily\bfseries}
\renewcommand{\cftsecfont}{\normalfont\sffamily\bfseries}
\renewcommand{\cftsubsecfont}{\normalfont\sffamily}
\renewcommand{\cftsubsubsecfont}{\normalfont\sffamily}

\makeatletter
\let\oldl@section\l@section
\def\l@section#1#2{\oldl@section{#1}{\sffamily\bfseries#2}}

\let\oldl@subsection\l@subsection
\def\l@subsection#1#2{\oldl@subsection{#1}{\sffamily#2}}

\let\oldl@subsubsection\l@subsubsection
\def\l@subsubsection#1#2{\oldl@subsubsection{#1}{\sffamily#2}}
 % General Styling


\usepackage{enumitem}

% Figures
\usepackage{subcaption}

% TikZ and Graphics
\usepackage{tikz, pgfplots}
\pgfplotsset{compat=1.12}
\usetikzlibrary{patterns,arrows}
\usepgfplotslibrary{fillbetween}

\usepackage{pdfpages}
\usepackage{adjustbox}

\usepackage{lscape}
\usepackage{titling}
\usepackage[]{hyperref}


% Header Styling
\usepackage{fancyhdr}
\pagestyle{fancy}
\lhead{\sffamily \CLASS}
\rhead{\sffamily Chen \textbf{\thepage}}
\cfoot{}

% Paragraph Styling
\setlength{\columnsep}{1cm}
\setlength{\parindent}{0pt}
\setlength{\parskip}{5pt}
\renewcommand{\baselinestretch}{1}

% TOC Styling
\usepackage{tocloft}
\iffalse
\renewcommand{\cftsecleader}{\cftdotfill{\cftdotsep}}

\renewcommand\cftchapafterpnum{\vskip6pt}
\renewcommand\cftsecafterpnum{\vskip10pt}
\renewcommand\cftsubsecafterpnum{\vskip6pt}

% Adjust sectional unit title fonts in ToC
\renewcommand{\cftchapfont}{\sffamily}
\renewcommand{\cftsecfont}{\bfseries\sffamily}
\renewcommand{\cftsecnumwidth}{2em}
\renewcommand{\cftsubsecfont}{\sffamily}
\renewcommand{\cfttoctitlefont}{\hfill\bfseries\sffamily\MakeUppercase}
\renewcommand{\cftaftertoctitle}{\hfill}

\renewcommand{\cftchappagefont}{\sffamily}
\renewcommand{\cftsecpagefont}{\bfseries\sffamily}
\renewcommand{\cftsubsecpagefont}{\sffamily}
\fi
 % General Styling
% Code Display Setup
\usepackage{listings,lstautogobble}
\usepackage{lipsum}
\usepackage{courier}
\usepackage{catchfilebetweentags}

\lstset{
	basicstyle=\small\ttfamily,
	breaklines=true, 
	frame = single,
	rangeprefix=,
	rangesuffix=,
	includerangemarker=false,
	autogobble = true
}


\usepackage{algorithmicx}
\usepackage{algpseudocode}

\newcommand{\To}{\textbf{to}~}
\newcommand{\DownTo}{\textbf{downto}~}
\renewcommand{\algorithmicdo}{\hspace{-.2em}\textbf{:}}
 % Code Display Setup
% AMS MATH Styling
\usepackage{amsmath, amssymb}
\newcommand{\qed}{\hfill\(\square\)}

%\newtheorem*{lemma}{Lemma} 
%\newtheorem*{theorem}{Theorem}
%\newtheorem*{definition}{Definition}
%\newtheorem*{prop}{Proposition}
%\renewenvironment{proof}{{\bfseries Proof.}}{}


% mathcal
\newcommand{\cA}{\ensuremath{\mathcal{A}}}
\newcommand{\cB}{\ensuremath{\mathcal{B}}}
\newcommand{\cC}{\ensuremath{\mathcal{C}}}
\newcommand{\cD}{\ensuremath{\mathcal{D}}}
\newcommand{\cE}{\ensuremath{\mathcal{E}}}
\newcommand{\cF}{\ensuremath{\mathcal{F}}}
\newcommand{\cG}{\ensuremath{\mathcal{G}}}
\newcommand{\cH}{\ensuremath{\mathcal{H}}}
\newcommand{\cI}{\ensuremath{\mathcal{I}}}
\newcommand{\cJ}{\ensuremath{\mathcal{J}}}
\newcommand{\cK}{\ensuremath{\mathcal{K}}}
\newcommand{\cL}{\ensuremath{\mathcal{L}}}
\newcommand{\cM}{\ensuremath{\mathcal{M}}}
\newcommand{\cN}{\ensuremath{\mathcal{N}}}
\newcommand{\cO}{\ensuremath{\mathcal{O}}}
\newcommand{\cP}{\ensuremath{\mathcal{P}}}
\newcommand{\cQ}{\ensuremath{\mathcal{Q}}}
\newcommand{\cR}{\ensuremath{\mathcal{R}}}
\newcommand{\cS}{\ensuremath{\mathcal{S}}}
\newcommand{\cT}{\ensuremath{\mathcal{T}}}
\newcommand{\cU}{\ensuremath{\mathcal{U}}}
\newcommand{\cV}{\ensuremath{\mathcal{V}}}
\newcommand{\cW}{\ensuremath{\mathcal{W}}}
\newcommand{\cX}{\ensuremath{\mathcal{X}}}
\newcommand{\cY}{\ensuremath{\mathcal{Y}}}
\newcommand{\cZ}{\ensuremath{\mathcal{Z}}}

% mathbb
\usepackage{bbm}
\newcommand{\bOne}{\ensuremath{\mathbbm{1}}}

\newcommand{\bA}{\ensuremath{\mathbb{A}}}
\newcommand{\bB}{\ensuremath{\mathbb{B}}}
\newcommand{\bC}{\ensuremath{\mathbb{C}}}
\newcommand{\bD}{\ensuremath{\mathbb{D}}}
\newcommand{\bE}{\ensuremath{\mathbb{E}}}
\newcommand{\bF}{\ensuremath{\mathbb{F}}}
\newcommand{\bG}{\ensuremath{\mathbb{G}}}
\newcommand{\bH}{\ensuremath{\mathbb{H}}}
\newcommand{\bI}{\ensuremath{\mathbb{I}}}
\newcommand{\bJ}{\ensuremath{\mathbb{J}}}
\newcommand{\bK}{\ensuremath{\mathbb{K}}}
\newcommand{\bL}{\ensuremath{\mathbb{L}}}
\newcommand{\bM}{\ensuremath{\mathbb{M}}}
\newcommand{\bN}{\ensuremath{\mathbb{N}}}
\newcommand{\bO}{\ensuremath{\mathbb{O}}}
\newcommand{\bP}{\ensuremath{\mathbb{P}}}
\newcommand{\bQ}{\ensuremath{\mathbb{Q}}}
\newcommand{\bR}{\ensuremath{\mathbb{R}}}
\newcommand{\bS}{\ensuremath{\mathbb{S}}}
\newcommand{\bT}{\ensuremath{\mathbb{T}}}
\newcommand{\bU}{\ensuremath{\mathbb{U}}}
\newcommand{\bV}{\ensuremath{\mathbb{V}}}
\newcommand{\bW}{\ensuremath{\mathbb{W}}}
\newcommand{\bX}{\ensuremath{\mathbb{X}}}
\newcommand{\bY}{\ensuremath{\mathbb{Y}}}
\newcommand{\bZ}{\ensuremath{\mathbb{Z}}}

% alternative mathbb
\newcommand{\NN}{\ensuremath{\mathbb{N}}}
\newcommand{\RR}{\ensuremath{\mathbb{R}}}
\newcommand{\CC}{\ensuremath{\mathbb{C}}}
\newcommand{\ZZ}{\ensuremath{\mathbb{Z}}}
\newcommand{\EE}{\ensuremath{\mathbb{E}}}
\newcommand{\PP}{\ensuremath{\mathbb{P}}}
\newcommand{\VV}{\ensuremath{\mathbb{V}}}
\newcommand{\cov}{\ensuremath{\text{Co}\VV}}
% Math Commands

\newcommand{\st}{~\big|~}
\newcommand{\stt}{\text{ st. }}
\newcommand{\ift}{\text{ if }}
\newcommand{\thent}{\text{ then }}
\newcommand{\owt}{\text{ otherwise }}

\newcommand{\norm}[1]{\left\lVert#1\right\rVert}
\newcommand{\snorm}[1]{\lVert#1\rVert}
\newcommand{\ip}[1]{\ensuremath{\left\langle #1 \right\rangle}}
\newcommand{\pp}[3][]{\frac{\partial^{#1}#2}{\partial #3^{#1}}}
\newcommand{\dd}[3][]{\frac{\d^{#1}#2}{\d #3^{#1}}}
\renewcommand{\d}{\ensuremath{\mathrm{d}}}

\newcommand{\indep}{\rotatebox[origin=c]{90}{$\models$}}




 % Math shortcuts
% Problem
\usepackage{floatrow}

\newenvironment{problem}[1][]
{\pagebreak
\noindent\rule{\textwidth}{1pt}\vspace{0.25em}
{\sffamily \textbf{#1}}
\par
}
{\par\vspace{-0.5em}\noindent\rule{\textwidth}{1pt}}

\newenvironment{solution}[1][]
{{\sffamily \textbf{#1}}
\par
}
{}

 % Problem Environment

\newcommand{\note}[1]{\textcolor{red}{\textbf{Note:} #1}}

\hypersetup{
   colorlinks=true,       % false: boxed links; true: colored links
   linkcolor=violet,          % color of internal links (change box color with linkbordercolor)
   citecolor=green,        % color of links to bibliography
   filecolor=magenta,      % color of file links
   urlcolor=cyan           % color of external links
}


\begin{document}
\maketitle



\begin{problem}[Problem 10.5]
Let \( M = (X, \mathcal{I} ) \) be a matroid and let \( k \) be a natural number. Let \( \mathcal{I}'  := \{Y\subseteq \mathcal{I}  : |Y| \leq k\} \). Show that \( (X, \mathcal{I}' ) \) is again a matroid.
\end{problem}

\begin{solution}

\begin{enumerate}[label=(\roman*)]
    \item Since \( (X, \mathcal{I} ) \) is a matroid then \( \emptyset \in \mathcal{I} \). Clearly \( |\emptyset| \leq k \) so \( \emptyset \in \mathcal{I} ' \).
    \item Let \( Y \in \mathcal{I}' \) and \( Z\subseteq Y \). Since \( (X, \mathcal{I} ) \) is a matroid then \( Z\in \mathcal{I} \).  Moreover, \( |Z| \leq |Y| \leq k \) as \( Z\in \mathcal{I} ' \). Therefore \( Z\in \mathcal{I} ' \).
    \item Let \( Y,Z\in  \mathcal{I}' \) with \( |Y|<|Z| \). Since \( (X, \mathcal{I} ) \) is a matroid and \( Y,Z\in \mathcal{I} \) then \( Y\cup \{x\} \in \mathcal{I} \) for some \( x\in Z \setminus Y \). Since \( Z\in \mathcal{I} ' \), \( |Z|\leq k \). Therefore \( |Y\cup \{x\}| = |Y|+ 1 \leq |Z| \leq k \). Therefore \( Y\cup\{x\} \in \mathcal{I} ' \).
\end{enumerate}

This proves \( (X, \mathcal{I} ') \) is a matroid. \qed

\end{solution}

\begin{problem}[Problem 10.19]
    Let \( M = (X, \mathcal{I} ) \) be a matroid, let \( B \) be a basis of \( M \), and let \( w: X\to\RR \) be a weight function. Show that \( B \) is a basis of maximum weight if and only if \( w(B') \leq w(B) \) for every basis \( B' \) with \( |B' \setminus B| = 1 \).
\end{problem}

\begin{solution}

Let \( B,B' \) be bases. From theorem we have:
\begin{enumerate}[label=(\roman*),nolistsep]
    \item for any \( x\in B'\setminus B \), \( (B'\setminus\{x\})\cup \{y\} \) is a basis of \( M \) for some  \( y\in B\setminus B' \).
    \item for any \( x\in B'\setminus B \), \( (B\setminus\{y\})\cup \{x\} \) is a basis of \( M \) for some  \( y\in B\setminus B' \).
\end{enumerate}


\vspace{1em}
Suppose \( B \) is a basis of maximum weight and let \( B' \) be some basis (with \( |B'\setminus B| = 1 \)). Then \( w(B') \leq w(B) \). 

Conversely, suppose that \( w(B^\dagger) \leq w(B) \) for every basis \( B^\dagger \) with \( |B^\dagger \setminus B| = 1 \). 

Let \( B' \) be a basis of \( M \).

Suppose \( | B' \setminus B| = 0 \). Then, since all bases have the same size we have \( B' = B \) so \( w(B') \leq w(B) \).

Now, suppose \( |B' \setminus B| > 0 \).

We induct on \( k := |B' \setminus B| \), assuming that \( w(B^\dagger) \leq w(B) \) for all bases \( B^\dagger \) of \( M \) with \( |B^\dagger \setminus B| < k \). Clearly the original hypothesis is the base case of our induction.

Since \( |B'\setminus B| > 0 \) there is some \( x\in B'\setminus B \). Therefore, by the theorem listed above there is some \( y\in B\setminus B' \) such that \( B^\dagger = (B' \setminus \{x\}) \cup \{y\} \) is a basis for \( M \).

Observe,
\begin{align*}
    |B^\dagger\setminus B| = | ((B'\setminus \{ x \})\cup\{y\}) \setminus B| = | (B'\setminus B)\setminus\{x\} | = k - 1 
\end{align*}

Therefore, by our induction hypothesis we have,
\begin{align*}
    w(B') - w(x) + w(y) = w(B^\dagger) \leq w(B)
\end{align*}


Since,
\begin{align*}
    |((B\setminus \{y\})\cup\{x\}) \setminus B| = |\{x\}| = 1 
\end{align*}
by the base case we have,
\begin{align*}
    w((B\setminus\{y\})\cup\{x\}) \leq w(B)
\end{align*}

Therefore, 
\begin{align*}
    w(B') \leq w(B) -w(y) + w(x) = w( (B\setminus\{y\} )\cup \{x\}) \leq w(B)
\end{align*}


This proves the result. \qed

\iffalse

Recall \( |B''| = |B| \) for any two bases \( B'' \) and \( B \) of \( M \). Therefore \( |B''\setminus B| = 0 \) means \( B = B'' \) in which case \( w(B'') = w(B) \).

Start with any basis \( B_0 \) of \( B \) with \( |B_0\setminus B| > 1 \).

At step \( k \in\{0,1,\ldots \} \), if \( |B_k \setminus B| = 0 \) stop.

Then at each step there is some element \( x\in B_k\setminus B \). By theorem we have, \( B_{k+1}:= (B\setminus\{y_k\})\cup\{x_k\} \) a basis for \( M \) for some \( y_k\in B\setminus B_k \).

Note \( |((B\setminus \{y_k\})\cup\{k_0\})\setminus B| = |\{x_k\}| = 1 \). Then, by hypothesis,
\begin{align*}
    w(B) \geq w((B\setminus \{y_k\}) \cup\{x_k\}) = w(B) - w(y_k) + w(x_k)
\end{align*}
Therefore,
\begin{align*}
    w(y_k) \geq w(x_k)
\end{align*}


At the end of the process we will have sets \( X:= \{x_k\}_{k\geq 0} \) and \( Y:= \{y_k\}_{k \geq 0} \) satisfying,
\begin{align*}
    w(Y) \geq w(X)
\end{align*}

We will also have,
\begin{align*}
    B_k = ( B \setminus Y )\cup X = B'
\end{align*}

Therefore,
\begin{align*}
    w(B') = w(B_k) = w(B) - w(Y) + w(X) \leq w(B)
\end{align*}

\fi




\iffalse
Now note, \( |((B'\setminus\{x\})\cup \{y\})\setminus B| = | (B'\setminus B)\setminus\{x\} | = k  \). Therefore, by our induction hypothesis,
\begin{align*}
    w(B) \geq w((B'\setminus\{x\})\cup\{y\}) = 
\end{align*}
\fi


\end{solution}

\begin{problem}[Problem 10.22]
Derive K\"onig's matching theorem from Edmonds' matroid intersection theorem. 
\end{problem}

\begin{solution}

Let \( G = (V_1\cup V_2, E) \) be a bipartite graph.
Let \( \mathcal{I}_1 \) be the collection of all subsets \( F \) of \( E \) so that no two edges in \( F \) have a vertex in \( V_1 \) in common.
Similarly, let \( \mathcal{I}_w \) be the collection of all subsets \( F \) of \( E \) so that no two edges in \( F \) have a vertex in \( V_2 \) in common.
Then both \( M_1:=(E, \mathcal{I}_1)  \) and \( M_2:=(E, \mathcal{I}_2)  \) are partition matroids.

Now observe that elements of \( \mathcal{I}_1\cap \mathcal{I}_2 \) are matchings in \( G \), and matchings in \( G \) are elements of \( \mathcal{I}_1\cap \mathcal{I}_2 \).

Therefore,
\begin{align*}
    \nu(G) = \max_{Y\in \mathcal{I}_1\cap \mathcal{I}_2} |Y| 
\end{align*}


Let \( U\subseteq E \). Define,
\begin{align*}
    C_1 = \{ v\in V_1 : \exists e\in E \text{ with } v\in e \}
\end{align*}

Now define,
\begin{align*}
    C_2 = \{ v\in V_2 : v \text{ reachable from } V_1\setminus C_1 \}
\end{align*}

Finally, let \( C = C_1 \cup C_2 \). If \( e\in U \) then \( e\cap C_1 \neq \emptyset \). Similarly, if \( e\in E\setminus U \) then \( e\cap C_2 \neq \emptyset \). Therefore \( C \) is a vertex cover of \( G \).

Every edge in \( U \) touches a vertex in \( C_1 \), and every vertex in \( C_1 \) touches an edge in \( U \). 
Similarly, every edge in \( E\setminus U \) touches a vertex in \( C_2 \), and every vertex in \( C_2 \) touches an edge in \( E\setminus U \). 

Therefore \( r_{M_1}(U) = |C_1| \) and \( r_{M_2}(E\setminus U) = |C_2| \) so that \( r_{M_1}(U) + r_{M_2}(E\setminus U) = |C| \).

In particular, this proves,
\begin{align*}
    \min_{U\subseteq E} (r_{M_1}(U) + r_{M_2}(E\setminus U)) \geq  \tau(G)
\end{align*}




Now, let \( C \) be a minimum size vertex cover of \( G \). Define,
\begin{align*}
    U = \{ e\in E: (C\cap V_1)\cap e \neq \emptyset  \}
\end{align*}

Therefore every edge in \( U \) is covered by some vertex in \( C\cap V_1 \) and every vertex in \( C\cap V_1 \) touches some edge in \( U \). This means that that \( r_{M_1}(U) = |C\cap V_1| \).

Similarly, every edge in \( E\setminus U \) must be covered by some vertex in \( C \). Moreover, by the minimality of \( C \), every vertex in \( C \) must touch an edge in \( E\setminus U \) (otherwise \( C \) without this edge would be a smaller vertex cover). Therefore \( r_{M_2}(E\setminus U) = |C\cap V_2| \)

Since points in \( C \) are either in \( V_1 \) or \( V_2 \) we have found a \( U\subseteq E \) such that,
\begin{align*}
    r_{M_1}(U) + r_{M_2}(E\setminus U) = |C| = \tau(G)
\end{align*}

This proves,
\begin{align*}
    \min_{U\subseteq E} (r_{M_1}(U) + r_{M_2}(E\setminus U)) \leq  \tau(G)
\end{align*}



Using Edmond's matroid intersection theorem we have,
\begin{align*}
    \nu(G) 
    = \max_{Y\in \mathcal{I}_1\cap \mathcal{I}_2} |Y| 
    = \min_{U\subseteq X} (r_{M_1}(U) + r_{M_2}(E\setminus U)) 
    = \tau(G)
\end{align*}

This is Konig's matching theorem. \qed

\end{solution}

\end{document}
