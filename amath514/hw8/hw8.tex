\documentclass[10pt]{article}
\usepackage[T1]{fontenc}

% Document Details
\newcommand{\CLASS}{AMATH 514}
\newcommand{\assigmentnum}{Assignment 8}

\usepackage[margin = 1.15in, top = 1.25in, bottom = 1.in]{geometry}

\usepackage{titling}
\setlength{\droptitle}{-6em}   % This is your set screw
\date{}
\renewcommand{\maketitle}{
	\clearpage
	\begingroup  
	\centering
	\LARGE \sffamily\textbf{\CLASS} \Large \assigmentnum\\[.8em]
	\large Tyler Chen\\[1em]
	\endgroup
	\thispagestyle{empty}
}
 % Title Styling
\usepackage{tocloft}
\renewcommand{\cfttoctitlefont}{\Large\sffamily\bfseries}
\renewcommand{\cftsecfont}{\normalfont\sffamily\bfseries}
\renewcommand{\cftsubsecfont}{\normalfont\sffamily}
\renewcommand{\cftsubsubsecfont}{\normalfont\sffamily}

\makeatletter
\let\oldl@section\l@section
\def\l@section#1#2{\oldl@section{#1}{\sffamily\bfseries#2}}

\let\oldl@subsection\l@subsection
\def\l@subsection#1#2{\oldl@subsection{#1}{\sffamily#2}}

\let\oldl@subsubsection\l@subsubsection
\def\l@subsubsection#1#2{\oldl@subsubsection{#1}{\sffamily#2}}
 % General Styling


\usepackage{enumitem}

% Figures
\usepackage{subcaption}

% TikZ and Graphics
\usepackage{tikz, pgfplots}
\pgfplotsset{compat=1.12}
\usetikzlibrary{patterns,arrows}
\usepgfplotslibrary{fillbetween}

\usepackage{pdfpages}
\usepackage{adjustbox}

\usepackage{lscape}
\usepackage{titling}
\usepackage[]{hyperref}


% Header Styling
\usepackage{fancyhdr}
\pagestyle{fancy}
\lhead{\sffamily \CLASS}
\rhead{\sffamily Chen \textbf{\thepage}}
\cfoot{}

% Paragraph Styling
\setlength{\columnsep}{1cm}
\setlength{\parindent}{0pt}
\setlength{\parskip}{5pt}
\renewcommand{\baselinestretch}{1}

% TOC Styling
\usepackage{tocloft}
\iffalse
\renewcommand{\cftsecleader}{\cftdotfill{\cftdotsep}}

\renewcommand\cftchapafterpnum{\vskip6pt}
\renewcommand\cftsecafterpnum{\vskip10pt}
\renewcommand\cftsubsecafterpnum{\vskip6pt}

% Adjust sectional unit title fonts in ToC
\renewcommand{\cftchapfont}{\sffamily}
\renewcommand{\cftsecfont}{\bfseries\sffamily}
\renewcommand{\cftsecnumwidth}{2em}
\renewcommand{\cftsubsecfont}{\sffamily}
\renewcommand{\cfttoctitlefont}{\hfill\bfseries\sffamily\MakeUppercase}
\renewcommand{\cftaftertoctitle}{\hfill}

\renewcommand{\cftchappagefont}{\sffamily}
\renewcommand{\cftsecpagefont}{\bfseries\sffamily}
\renewcommand{\cftsubsecpagefont}{\sffamily}
\fi
 % General Styling
% Code Display Setup
\usepackage{listings,lstautogobble}
\usepackage{lipsum}
\usepackage{courier}
\usepackage{catchfilebetweentags}

\lstset{
	basicstyle=\small\ttfamily,
	breaklines=true, 
	frame = single,
	rangeprefix=,
	rangesuffix=,
	includerangemarker=false,
	autogobble = true
}


\usepackage{algorithmicx}
\usepackage{algpseudocode}

\newcommand{\To}{\textbf{to}~}
\newcommand{\DownTo}{\textbf{downto}~}
\renewcommand{\algorithmicdo}{\hspace{-.2em}\textbf{:}}
 % Code Display Setup
% AMS MATH Styling
\usepackage{amsmath, amssymb}
\newcommand{\qed}{\hfill\(\square\)}

%\newtheorem*{lemma}{Lemma} 
%\newtheorem*{theorem}{Theorem}
%\newtheorem*{definition}{Definition}
%\newtheorem*{prop}{Proposition}
%\renewenvironment{proof}{{\bfseries Proof.}}{}


% mathcal
\newcommand{\cA}{\ensuremath{\mathcal{A}}}
\newcommand{\cB}{\ensuremath{\mathcal{B}}}
\newcommand{\cC}{\ensuremath{\mathcal{C}}}
\newcommand{\cD}{\ensuremath{\mathcal{D}}}
\newcommand{\cE}{\ensuremath{\mathcal{E}}}
\newcommand{\cF}{\ensuremath{\mathcal{F}}}
\newcommand{\cG}{\ensuremath{\mathcal{G}}}
\newcommand{\cH}{\ensuremath{\mathcal{H}}}
\newcommand{\cI}{\ensuremath{\mathcal{I}}}
\newcommand{\cJ}{\ensuremath{\mathcal{J}}}
\newcommand{\cK}{\ensuremath{\mathcal{K}}}
\newcommand{\cL}{\ensuremath{\mathcal{L}}}
\newcommand{\cM}{\ensuremath{\mathcal{M}}}
\newcommand{\cN}{\ensuremath{\mathcal{N}}}
\newcommand{\cO}{\ensuremath{\mathcal{O}}}
\newcommand{\cP}{\ensuremath{\mathcal{P}}}
\newcommand{\cQ}{\ensuremath{\mathcal{Q}}}
\newcommand{\cR}{\ensuremath{\mathcal{R}}}
\newcommand{\cS}{\ensuremath{\mathcal{S}}}
\newcommand{\cT}{\ensuremath{\mathcal{T}}}
\newcommand{\cU}{\ensuremath{\mathcal{U}}}
\newcommand{\cV}{\ensuremath{\mathcal{V}}}
\newcommand{\cW}{\ensuremath{\mathcal{W}}}
\newcommand{\cX}{\ensuremath{\mathcal{X}}}
\newcommand{\cY}{\ensuremath{\mathcal{Y}}}
\newcommand{\cZ}{\ensuremath{\mathcal{Z}}}

% mathbb
\usepackage{bbm}
\newcommand{\bOne}{\ensuremath{\mathbbm{1}}}

\newcommand{\bA}{\ensuremath{\mathbb{A}}}
\newcommand{\bB}{\ensuremath{\mathbb{B}}}
\newcommand{\bC}{\ensuremath{\mathbb{C}}}
\newcommand{\bD}{\ensuremath{\mathbb{D}}}
\newcommand{\bE}{\ensuremath{\mathbb{E}}}
\newcommand{\bF}{\ensuremath{\mathbb{F}}}
\newcommand{\bG}{\ensuremath{\mathbb{G}}}
\newcommand{\bH}{\ensuremath{\mathbb{H}}}
\newcommand{\bI}{\ensuremath{\mathbb{I}}}
\newcommand{\bJ}{\ensuremath{\mathbb{J}}}
\newcommand{\bK}{\ensuremath{\mathbb{K}}}
\newcommand{\bL}{\ensuremath{\mathbb{L}}}
\newcommand{\bM}{\ensuremath{\mathbb{M}}}
\newcommand{\bN}{\ensuremath{\mathbb{N}}}
\newcommand{\bO}{\ensuremath{\mathbb{O}}}
\newcommand{\bP}{\ensuremath{\mathbb{P}}}
\newcommand{\bQ}{\ensuremath{\mathbb{Q}}}
\newcommand{\bR}{\ensuremath{\mathbb{R}}}
\newcommand{\bS}{\ensuremath{\mathbb{S}}}
\newcommand{\bT}{\ensuremath{\mathbb{T}}}
\newcommand{\bU}{\ensuremath{\mathbb{U}}}
\newcommand{\bV}{\ensuremath{\mathbb{V}}}
\newcommand{\bW}{\ensuremath{\mathbb{W}}}
\newcommand{\bX}{\ensuremath{\mathbb{X}}}
\newcommand{\bY}{\ensuremath{\mathbb{Y}}}
\newcommand{\bZ}{\ensuremath{\mathbb{Z}}}

% alternative mathbb
\newcommand{\NN}{\ensuremath{\mathbb{N}}}
\newcommand{\RR}{\ensuremath{\mathbb{R}}}
\newcommand{\CC}{\ensuremath{\mathbb{C}}}
\newcommand{\ZZ}{\ensuremath{\mathbb{Z}}}
\newcommand{\EE}{\ensuremath{\mathbb{E}}}
\newcommand{\PP}{\ensuremath{\mathbb{P}}}
\newcommand{\VV}{\ensuremath{\mathbb{V}}}
\newcommand{\cov}{\ensuremath{\text{Co}\VV}}
% Math Commands

\newcommand{\st}{~\big|~}
\newcommand{\stt}{\text{ st. }}
\newcommand{\ift}{\text{ if }}
\newcommand{\thent}{\text{ then }}
\newcommand{\owt}{\text{ otherwise }}

\newcommand{\norm}[1]{\left\lVert#1\right\rVert}
\newcommand{\snorm}[1]{\lVert#1\rVert}
\newcommand{\ip}[1]{\ensuremath{\left\langle #1 \right\rangle}}
\newcommand{\pp}[3][]{\frac{\partial^{#1}#2}{\partial #3^{#1}}}
\newcommand{\dd}[3][]{\frac{\d^{#1}#2}{\d #3^{#1}}}
\renewcommand{\d}{\ensuremath{\mathrm{d}}}

\newcommand{\indep}{\rotatebox[origin=c]{90}{$\models$}}




 % Math shortcuts
% Problem
\usepackage{floatrow}

\newenvironment{problem}[1][]
{\pagebreak
\noindent\rule{\textwidth}{1pt}\vspace{0.25em}
{\sffamily \textbf{#1}}
\par
}
{\par\vspace{-0.5em}\noindent\rule{\textwidth}{1pt}}

\newenvironment{solution}[1][]
{{\sffamily \textbf{#1}}
\par
}
{}

 % Problem Environment

\newcommand{\note}[1]{\textcolor{red}{\textbf{Note:} #1}}

\hypersetup{
   colorlinks=true,       % false: boxed links; true: colored links
   linkcolor=violet,          % color of internal links (change box color with linkbordercolor)
   citecolor=green,        % color of links to bibliography
   filecolor=magenta,      % color of file links
   urlcolor=cyan           % color of external links
}


\begin{document}
\maketitle



\begin{problem}[Problem 1]
    Consider the cube \( P:=[0,1]^n = \{x : 0\leq x_i \leq 1, i=1,\ldots, n \} \). Consider a sequence of points \( \{x_k\}_{k\geq 0} \) with \( x_1 = \left( \frac{1}{2}, \ldots, \frac{1}{2} \right) \) and with the only restriction that \( x_{k+1}\in \mathcal{E} (x_k,1/2) \). Prove that it takes at least \( \Omega (\sqrt{n} \log (1/\delta)) \) iterations until \( x_k \) can be within a \( \norm{\cdot}_\infty \)-distance of \( \delta \) from the vertex \( 0 \).
\end{problem}

\begin{solution}



We use the interpretation that \( s_i(x) \) is the geometric distance of \( x \) to the \( i \)-th hyperplane. For notational convenience we will start at \( x_0 \).

Observe that,
\begin{align*}
    \sum_{i=1}^{2n} \left( \dfrac{s_i(y)-s_i(x)}{s_i(x)} \right)^2 
    = \sum_{i=1}^{n} \left[ \left( \dfrac{y_i-x_i}{x_i} \right)^2 + \left( \dfrac{y_i-x_i}{1-x_i} \right)^2 \right] > \sum_{i=1}^{n} \left( \dfrac{y_i-x_i}{x_i} \right)^2
\end{align*}

Suppose \( y = d(1,\ldots, 1) \) and \( x = c(1,\ldots, 1) \). Then,
\begin{align*}
    R^2 = \sum_{i=1}^{n} \left( \dfrac{y_i-x_i}{x_i} \right)^2 = n \left( \dfrac{d-c}{c} \right)^2
    && \Longleftrightarrow &&
    d = c \pm \dfrac{c R}{\sqrt{n}} = c \left(1 \pm \dfrac{R}{\sqrt{n}}\right)
\end{align*}

We define a sequence \( \{x_k\} \) where \( x_k = c_k(1, \ldots, 1) \) for all \( k \) and satisfies the relation,
\begin{align*}
    c_{k+1} = c_k \left( 1 - \dfrac{R}{\sqrt{n}}  \right) = c_0 \left( 1- \dfrac{R}{\sqrt{n}} \right)^{k+1} 
\end{align*}

Note that \( x_{k+1} \) is outside of the ellipsoid \( \mathcal{E} (x_k,R) \) since,
\begin{align*}
    \sum_{i=1}^{2n} \left( \dfrac{s_i(x_{k+1})-s_i(x_k)}{s_i(x_k)} \right)^2 
    > \sum_{i=1}^{n} \left( \dfrac{c_{k+1}-c_k}{c_k} \right)^2 
    = n\left( \dfrac{c_{k+1}-c_k}{c_k} \right)^2 = R^2
\end{align*}

That is, at each step we jump a bit more than allowed by a ``valid'' sequence.

We now determine how long it takes for \( \{x_k\}_{k\geq 0} \) to get within \( \delta \) of the origin (in the infinity norm). That is, we seek \( k \) such that \( c_k < \delta \). Thus,
\begin{align*}
    c_0 \left( 1-\dfrac{R}{\sqrt{n}} \right)^{k} = c_k < \delta
\end{align*}

Now recall,
\begin{align*}
    \log \left( 1-a \right) = -a-a^2/2 - ... < -2a
\end{align*}


Thus, dividing by \( c_0 \) on both sides, and since the logarithm is monotonically increasing, 
\begin{align*}
    - k \left( \dfrac{2R}{\sqrt{n}} \right) < k \log \left( 1- \dfrac{R}{\sqrt{n}} \right) < \log \left( \dfrac{\delta}{c_0} \right)
\end{align*}

Therefore
\begin{align*}
    k > - \dfrac{\sqrt{n}}{2R} \log \left( \dfrac{\delta}{c_0} \right) = \dfrac{\sqrt{n}}{2R} \log \left( \dfrac{c_0}{\delta} \right)
\end{align*}

This proves \( \{x_k\}_{k\geq 0} \) takes \( \Omega(\sqrt{n}\log(1/\delta)) \) steps to be within a \( \norm{\cdot}_\infty \)-distance of \( \delta \) from the origin.
Therefore, any sequence \( \{y_k\} \) where \( y_{k+1}\in \mathcal{E} (y_k,1/2) \) and \( y_k = d_k(1,\ldots, 1) \) takes at least \( \Omega(\sqrt{n}\log(1/\delta)) \) to be within a \( \norm{\cdot}_\infty \)-distance of \( \delta \) from the origin.

It remains to show that any other sequence \( z_k \) satisfying \( z_{k+1}\in \mathcal{E}(z_k,1/2) \) does not converge faster than a sequence along the diagonal between the points \( 0 \) and \( 1 \).

Let \( \{z_k\}_{k\geq 0} \) be any sequence satisfying \( z_{k+1}\in \mathcal{E}(z_k,1/2) \). We claim the sequence \( \{ y_k \}_{k\geq 0} \) obtained by projecting \( z_k \) onto the diagonal between \( 0 \) and \( 1 \) satisfies \( y_{k+1}\in y_k \).

Let \( u,v \) be two points in \( P \) satisfying \( v\in \mathcal{E} (u,1/2) \). Let \( x,y \) be their projections onto the main diagonal.

We have \( \norm{x-y}\leq \norm{u-v} \). Moreover,
\begin{align*}
    |s_i(u)-s_i(v)| = |u_i - v_i| \leq |y_i - x_i| = |s_i(y)-s_i(x)|
\end{align*}

Therefore,
\begin{align*}
    \sum_{i=1}^{2n} \left( \dfrac{s_i(u)-s_i(v)}{s_i(u)} \right)^2 
    \leq \sum_{i=1}^{2n} \left( \dfrac{s_i(x)-s_i(y)}{s_i(u)} \right)^2 
    \leq \sum_{i=1}^{2n} \left( \dfrac{s_i(x)-s_i(y)}{s_i(x)} \right)^2 
\end{align*}
where the second inequality comes from the fact that for any \( s_i(u) < s_i(x) \) there is some \( j \) such that \( s_j(u) > s_j(x) \). Since \( x \) is on the main diagonal it is ``further away'' from all the boundaries than \( u \). (I know this is kind of handwavy)

This means \( y \in \mathcal{E} (x,1/2) \) as desired.

Therefore the projection of any ``valid'' sequence onto the main diagonal is also valid. But no sequence on the diagonal can converge fast enough so the result is proved
\qed


\end{solution}

\begin{problem}[Problem 2]
    Recall that the presented interior point method takes \( \mathcal{O} (L\sqrt{m}) \) iterations to get within an additive \( 2^{-L} \) distance to the optimum for a polytope \( P = \{ x\in R^n : Ax\leq b \}  \) with \( A\in \RR^{m\times n} \). There is indeed a way of bringing the number of iterations down to \( \mathcal{O} (L\sqrt{n}) \). A deep result of Nesterov and Nemirovsky says that there is a convex function \( \varphi: \RR^n \to \RR \) that is self-concordant which means it satisfies the following properties for some universal constant \( C > 0 \):
    \begin{enumerate}
        \item[(A)] For any \( 0\leq R \leq 1/C \) and \( x\in \mathcal{E}(x^*,R) \) one has 
            \begin{align*}
                (1-2R)\nabla^2\varphi(x) \preceq \nabla^2\varphi(x^*) \preceq (1+2R)\nabla^2\varphi(x)  
            \end{align*}
            where we redefine the ellipsoid \( \mathcal{E}(x^*,R) := \{x\in \RR^n : (x-x^*)^T \left[\nabla^2 \varphi(x) \right](x-x^*) \leq R^2 \} \).
        \item[(B)] One has \( \nabla \varphi(x)\nabla \varphi(x)^T \preceq Cn\nabla^2\varphi(x) \) for all \( x\in \operatorname{int}(P) \).
        \item[(C)] If \( x\to\partial P \), then \( \varphi(x) \to \infty \).
    \end{enumerate}
    For \( t\geq 0 \) we modify the barrier function to \( F_t(x) := t\cdot c^Tx+\varphi(x) \). Prove the following:
    \begin{enumerate}
        \item For \( x \in \mathcal{E} (x^*,R) \) with \( x^*:= x^*(t) \) and \( x':= x-\left[\nabla^2F_t(x)\right]^{-1}\nabla F_t(x) \) one has \( x'\in \mathcal{E} (x^*, \mathcal{O}(R^2)) \) for sufficiently small \( R \). 
        \item \( \max\{t\cdot c^T(x-x^*(t)) : x\in \mathcal{E}(x^*(t),R)\} \leq \mathcal{O} (R\sqrt{n}) \) for all \( t > 0 \) and \( R>0 \) small enough.
    \end{enumerate}
\end{problem}

\begin{solution}

\begin{enumerate}
    \item 
%        Let \( y = \lambda x + (1-\lambda) x^* \) for \( \lambda\in[0,1] \).

        Let \( y = x^* + h \) for some \( h \) with \( \norm{h} = \mathcal{O}(R) \) (for instance any \( y\in \mathcal{E} (x^*,R) \).

        Let \( F_{ij}(y) \) be an entry of \( \nabla^2 F_t(y) \). Then,
        \begin{align*}
            F_{ij}(y) = F_{ij}(x^*) + h^T \nabla F_{ij}(x^*) + \mathcal{O} (h^2)
        \end{align*}

        Since \( \nabla F_t(x^*) = 0 \), this shows that,
        \begin{align*}
            \nabla^2 F_t(y) = \nabla^2 F_t(x^*) + \mathcal{O} (h^2)
        \end{align*}
        
        \iffalse


        Note that \( F_t(y) = F_t(x^*+h) = F_t(x) + h^T \nabla F_t(x^*) + \mathcal{O} (h^2) \). Then, as \( |h| = \mathcal{O} (R) \), and \( \nabla F_t(x^*) = 0 \),
        \begin{align*}
            \nabla^2 F_t(y) = \nabla^2 F_t(x^*) + \mathcal{O} (R^2)
        \end{align*}
        \fi

        Then, there is some  \( k_1 \) such that,
        \begin{align*}
            \nabla^2 F_t(y) \preceq \nabla^2 F_t(^*x) + k_1R^2
        \end{align*}
        
        Similarly there is some \( k_2 \) such that,
        \begin{align*}
            \nabla^2 F_t(x^*) \preceq \nabla^2 F_t(y) + k_2R^2
        \end{align*}
       
        Define \( k = \max\{ |k_1|,|k_2| \} \). Then,
        \begin{align*}
            \nabla^2 F_t (x^*) - k R^2 \preceq \nabla^2 F_t (y) \preceq \nabla^2 F_t (x^*) + k R^2
        \end{align*}
        
        Therefore, using (A) and the fact that \( \nabla^2 F_t(x) = \nabla^2 \varphi(x) \),
        \begin{align*}
            (1-2R) \nabla^2 F_t(x) - kR^2 \preceq \nabla^2 F_t(y) \preceq (1+2R)\nabla^2 F_t(x) + k R^2
        \end{align*}
       
        Note that here \( k \) depends on \( x \). However, \( \sup_{x\in \mathcal{E}} k \) gives a bound for all \( x\in \mathcal{E} (x^*,R) \).

        Somehow get a bound of the form,
        \begin{align*}
            (1-c R) \nabla^2 F_t(x) \leq \nabla^2 F_t(y) \leq (1+cR)\nabla^2 F_t(x) 
        \end{align*}
        

        The result then follows from Lemma 8.3 as the result (claim 3) is proved using claim 2 which is proved using claim 1 shown here.

        Sorry for the sloppy work. I was in a bike accident earlier this week and haven't been able to concentrate very well since then. I would have asked for an extension, but at this point I think I need to cut my losses and move on to other work.
        
        \iffalse
        Then,
        \begin{align*}
            \nabla^2 F_t(y) 
            = \lambda \nabla^2 F_t(x) + (1-\lambda) \nabla^2 F_t(x^*)
        \end{align*}
        
        For sufficiently small \( R \) we have,
        \begin{align*}
            \lambda \nabla^2 F_t(x) + (1-\lambda) \nabla^2 F_t(x^*)
            &\leq \lambda \nabla^2 F_t(x) + (1-\lambda)(1+2R) \nabla^2F_t(x)
            \\ &= (\lambda + (1-\lambda)(1+2R))\nabla^2F_t(x)
            \\ &= (1 + (1-\lambda)2R)\nabla^2F_t(x)
        \end{align*}   
        \begin{align*}
            \lambda \nabla^2 F_t(x) + (1-\lambda) \nabla^2 F_t(x^*)
            &\geq \lambda \nabla^2 F_t(x) + (1-\lambda)(1-2R) \nabla^2F_t(x)
            \\ &= (\lambda + (1-\lambda)(1-2R))\nabla^2F_t(x)
            \\ &= (1 - (1-\lambda)2R)\nabla^2F_t(x)
        \end{align*}

        Since \( \lambda\in[0,1] \) we have \( (1-\lambda) < 1 \) so,
        \begin{align*}
            (1-2R)\nabla^2F_t(x) \leq \nabla^2 F_t(y) \leq (1+2R)\nabla^2 F_t(x) 
        \end{align*}
        
        The result then follows from Lemma 8.3 as the result (claim 3) is proved using claim 2 which is proved using claim 1 shown here. \qed
\fi
    \item
        Suppose \( x\in \mathcal{E}(x^*,R) \). 
        That is,
        \begin{align*}
            (x-x^*)^T \left[ \nabla^2 \varphi(x^*) \right] (x-x^*) \leq R^2
        \end{align*}
        
        Then, by (B),
        \begin{align*}
            (x-x^*)^T \left[ \nabla \varphi(x^*) \nabla \varphi(x^*)^T \right] (x-x^*) 
            \leq Cn (x-x^*)^T \left[ \nabla^2 \varphi(x^*) \right] (x-x^*) \leq CnR^2
        \end{align*}
        
        But,
        \begin{align*}
            (x-x^*)^T \left[ \nabla \varphi(x^*) \nabla \varphi(x^*)^T \right] (x-x^*) 
            = \left|\nabla \varphi(x^*)^T (x-x^*)\right|^2
        \end{align*}
        
        Since \( \nabla F_t(x^*) = 0 \) we have,
        \begin{align*}
            \nabla \varphi(x^*) = -t\cdot c
        \end{align*}
        
        Thus,
        \begin{align*}
            \left|\nabla \varphi(x^*)^T (x-x^*)\right|^2
            = \left| t\cdot c^T (x-x^*) \right|^2
        \end{align*}
       
        Combining these statements we have,
        \begin{align*}
            \left| t\cdot c^T (x-x^*) \right|^2
            \leq CnR^2
        \end{align*}

        This proves \( t\cdot c^T(x-x^*) \leq \mathcal{O} (R\sqrt{n} \) for all \( x\in \mathcal{E} (x^*,R) \) and sufficiently small \( R \). Then certainly 
\( \max\{t\cdot c^T(x-x^*(t)) : x\in \mathcal{E}(x^*(t),R)\} \leq \mathcal{O} (R\sqrt{n}) \) for all \( t > 0 \) and \( R>0 \) small enough. \qed
\end{enumerate}
\end{solution}

\end{document}
