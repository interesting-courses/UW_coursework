\documentclass[10pt]{article}
\usepackage[T1]{fontenc}

% Document Details
\newcommand{\CLASS}{AMATH 562}
\newcommand{\assigmentnum}{Assignment 6}

\usepackage[margin = 1.15in, top = 1.25in, bottom = 1.in]{geometry}

\usepackage{titling}
\setlength{\droptitle}{-6em}   % This is your set screw
\date{}
\renewcommand{\maketitle}{
	\clearpage
	\begingroup  
	\centering
	\LARGE \sffamily\textbf{\CLASS} \Large \assigmentnum\\[.8em]
	\large Tyler Chen\\[1em]
	\endgroup
	\thispagestyle{empty}
}
 % Title Styling
\usepackage{tocloft}
\renewcommand{\cfttoctitlefont}{\Large\sffamily\bfseries}
\renewcommand{\cftsecfont}{\normalfont\sffamily\bfseries}
\renewcommand{\cftsubsecfont}{\normalfont\sffamily}
\renewcommand{\cftsubsubsecfont}{\normalfont\sffamily}

\makeatletter
\let\oldl@section\l@section
\def\l@section#1#2{\oldl@section{#1}{\sffamily\bfseries#2}}

\let\oldl@subsection\l@subsection
\def\l@subsection#1#2{\oldl@subsection{#1}{\sffamily#2}}

\let\oldl@subsubsection\l@subsubsection
\def\l@subsubsection#1#2{\oldl@subsubsection{#1}{\sffamily#2}}
 % General Styling


\usepackage{enumitem}

% Figures
\usepackage{subcaption}

% TikZ and Graphics
\usepackage{tikz, pgfplots}
\pgfplotsset{compat=1.12}
\usetikzlibrary{patterns,arrows}
\usepgfplotslibrary{fillbetween}

\usepackage{pdfpages}
\usepackage{adjustbox}

\usepackage{lscape}
\usepackage{titling}
\usepackage[]{hyperref}


% Header Styling
\usepackage{fancyhdr}
\pagestyle{fancy}
\lhead{\sffamily \CLASS}
\rhead{\sffamily Chen \textbf{\thepage}}
\cfoot{}

% Paragraph Styling
\setlength{\columnsep}{1cm}
\setlength{\parindent}{0pt}
\setlength{\parskip}{5pt}
\renewcommand{\baselinestretch}{1}

% TOC Styling
\usepackage{tocloft}
\iffalse
\renewcommand{\cftsecleader}{\cftdotfill{\cftdotsep}}

\renewcommand\cftchapafterpnum{\vskip6pt}
\renewcommand\cftsecafterpnum{\vskip10pt}
\renewcommand\cftsubsecafterpnum{\vskip6pt}

% Adjust sectional unit title fonts in ToC
\renewcommand{\cftchapfont}{\sffamily}
\renewcommand{\cftsecfont}{\bfseries\sffamily}
\renewcommand{\cftsecnumwidth}{2em}
\renewcommand{\cftsubsecfont}{\sffamily}
\renewcommand{\cfttoctitlefont}{\hfill\bfseries\sffamily\MakeUppercase}
\renewcommand{\cftaftertoctitle}{\hfill}

\renewcommand{\cftchappagefont}{\sffamily}
\renewcommand{\cftsecpagefont}{\bfseries\sffamily}
\renewcommand{\cftsubsecpagefont}{\sffamily}
\fi
 % General Styling
% Code Display Setup
\usepackage{listings,lstautogobble}
\usepackage{lipsum}
\usepackage{courier}
\usepackage{catchfilebetweentags}

\lstset{
	basicstyle=\small\ttfamily,
	breaklines=true, 
	frame = single,
	rangeprefix=,
	rangesuffix=,
	includerangemarker=false,
	autogobble = true
}


\usepackage{algorithmicx}
\usepackage{algpseudocode}

\newcommand{\To}{\textbf{to}~}
\newcommand{\DownTo}{\textbf{downto}~}
\renewcommand{\algorithmicdo}{\hspace{-.2em}\textbf{:}}
 % Code Display Setup
% AMS MATH Styling
\usepackage{amsmath, amssymb}
\newcommand{\qed}{\hfill\(\square\)}

%\newtheorem*{lemma}{Lemma} 
%\newtheorem*{theorem}{Theorem}
%\newtheorem*{definition}{Definition}
%\newtheorem*{prop}{Proposition}
%\renewenvironment{proof}{{\bfseries Proof.}}{}


% mathcal
\newcommand{\cA}{\ensuremath{\mathcal{A}}}
\newcommand{\cB}{\ensuremath{\mathcal{B}}}
\newcommand{\cC}{\ensuremath{\mathcal{C}}}
\newcommand{\cD}{\ensuremath{\mathcal{D}}}
\newcommand{\cE}{\ensuremath{\mathcal{E}}}
\newcommand{\cF}{\ensuremath{\mathcal{F}}}
\newcommand{\cG}{\ensuremath{\mathcal{G}}}
\newcommand{\cH}{\ensuremath{\mathcal{H}}}
\newcommand{\cI}{\ensuremath{\mathcal{I}}}
\newcommand{\cJ}{\ensuremath{\mathcal{J}}}
\newcommand{\cK}{\ensuremath{\mathcal{K}}}
\newcommand{\cL}{\ensuremath{\mathcal{L}}}
\newcommand{\cM}{\ensuremath{\mathcal{M}}}
\newcommand{\cN}{\ensuremath{\mathcal{N}}}
\newcommand{\cO}{\ensuremath{\mathcal{O}}}
\newcommand{\cP}{\ensuremath{\mathcal{P}}}
\newcommand{\cQ}{\ensuremath{\mathcal{Q}}}
\newcommand{\cR}{\ensuremath{\mathcal{R}}}
\newcommand{\cS}{\ensuremath{\mathcal{S}}}
\newcommand{\cT}{\ensuremath{\mathcal{T}}}
\newcommand{\cU}{\ensuremath{\mathcal{U}}}
\newcommand{\cV}{\ensuremath{\mathcal{V}}}
\newcommand{\cW}{\ensuremath{\mathcal{W}}}
\newcommand{\cX}{\ensuremath{\mathcal{X}}}
\newcommand{\cY}{\ensuremath{\mathcal{Y}}}
\newcommand{\cZ}{\ensuremath{\mathcal{Z}}}

% mathbb
\usepackage{bbm}
\newcommand{\bOne}{\ensuremath{\mathbbm{1}}}

\newcommand{\bA}{\ensuremath{\mathbb{A}}}
\newcommand{\bB}{\ensuremath{\mathbb{B}}}
\newcommand{\bC}{\ensuremath{\mathbb{C}}}
\newcommand{\bD}{\ensuremath{\mathbb{D}}}
\newcommand{\bE}{\ensuremath{\mathbb{E}}}
\newcommand{\bF}{\ensuremath{\mathbb{F}}}
\newcommand{\bG}{\ensuremath{\mathbb{G}}}
\newcommand{\bH}{\ensuremath{\mathbb{H}}}
\newcommand{\bI}{\ensuremath{\mathbb{I}}}
\newcommand{\bJ}{\ensuremath{\mathbb{J}}}
\newcommand{\bK}{\ensuremath{\mathbb{K}}}
\newcommand{\bL}{\ensuremath{\mathbb{L}}}
\newcommand{\bM}{\ensuremath{\mathbb{M}}}
\newcommand{\bN}{\ensuremath{\mathbb{N}}}
\newcommand{\bO}{\ensuremath{\mathbb{O}}}
\newcommand{\bP}{\ensuremath{\mathbb{P}}}
\newcommand{\bQ}{\ensuremath{\mathbb{Q}}}
\newcommand{\bR}{\ensuremath{\mathbb{R}}}
\newcommand{\bS}{\ensuremath{\mathbb{S}}}
\newcommand{\bT}{\ensuremath{\mathbb{T}}}
\newcommand{\bU}{\ensuremath{\mathbb{U}}}
\newcommand{\bV}{\ensuremath{\mathbb{V}}}
\newcommand{\bW}{\ensuremath{\mathbb{W}}}
\newcommand{\bX}{\ensuremath{\mathbb{X}}}
\newcommand{\bY}{\ensuremath{\mathbb{Y}}}
\newcommand{\bZ}{\ensuremath{\mathbb{Z}}}

% alternative mathbb
\newcommand{\NN}{\ensuremath{\mathbb{N}}}
\newcommand{\RR}{\ensuremath{\mathbb{R}}}
\newcommand{\CC}{\ensuremath{\mathbb{C}}}
\newcommand{\ZZ}{\ensuremath{\mathbb{Z}}}
\newcommand{\EE}{\ensuremath{\mathbb{E}}}
\newcommand{\PP}{\ensuremath{\mathbb{P}}}
\newcommand{\VV}{\ensuremath{\mathbb{V}}}
\newcommand{\cov}{\ensuremath{\text{Co}\VV}}
% Math Commands

\newcommand{\st}{~\big|~}
\newcommand{\stt}{\text{ st. }}
\newcommand{\ift}{\text{ if }}
\newcommand{\thent}{\text{ then }}
\newcommand{\owt}{\text{ otherwise }}

\newcommand{\norm}[1]{\left\lVert#1\right\rVert}
\newcommand{\snorm}[1]{\lVert#1\rVert}
\newcommand{\ip}[1]{\ensuremath{\left\langle #1 \right\rangle}}
\newcommand{\pp}[3][]{\frac{\partial^{#1}#2}{\partial #3^{#1}}}
\newcommand{\dd}[3][]{\frac{\d^{#1}#2}{\d #3^{#1}}}
\renewcommand{\d}{\ensuremath{\mathrm{d}}}

\newcommand{\indep}{\rotatebox[origin=c]{90}{$\models$}}




 % Math shortcuts
% Problem
\usepackage{floatrow}

\newenvironment{problem}[1][]
{\pagebreak
\noindent\rule{\textwidth}{1pt}\vspace{0.25em}
{\sffamily \textbf{#1}}
\par
}
{\par\vspace{-0.5em}\noindent\rule{\textwidth}{1pt}}

\newenvironment{solution}[1][]
{{\sffamily \textbf{#1}}
\par
}
{}

 % Problem Environment

\newcommand{\note}[1]{\textcolor{red}{\textbf{Note:} #1}}

\hypersetup{
   colorlinks=true,       % false: boxed links; true: colored links
   linkcolor=violet,          % color of internal links (change box color with linkbordercolor)
   citecolor=green,        % color of links to bibliography
   filecolor=magenta,      % color of file links
   urlcolor=cyan           % color of external links
}


\begin{document}
\maketitle



\begin{problem}[Exercise 6.2]
Consider the sample space \( S=[0,1] \) with uniform probability distribution, i.e.,
\begin{align*}
    \PP([a,b])=b-a, ~ \forall 0\leq a\leq b\leq 1
\end{align*}
    Define the sequence \( \{X_n\}_{n\in\NN_0} \) as \( X_n(s) = \frac{n}{n+1}s+(1-s)^n \). Also, define the random variable \( X \) on this sample space as \( X(s) = s \). Show that \( X_n \to_{a.s.} X \).
\end{problem}

\begin{solution}

Observe that for all \( s\in(0,1] \), \( 0 \leq (1-s) < 1 \) so,
\begin{align*}
    \lim_{n\to\infty} \left[ \dfrac{n}{n+1}s + (1-s)^n \right] = s + 0 = s
\end{align*}

In particular, this means that,
\begin{align*}
    [0,1) \subseteq \left\{ s\in S : \lim_{n\to\infty} |X_n-X| = 0 \right\} 
\end{align*}

Thus, since \( \PP[1,1] = 0 \) and \( [0,1) \cap[1,1] = \emptyset \),
\begin{align*}
    \PP \left( \lim_{n\to \infty} |X_n-X|=0 \right) \geq \PP([0,1)) = \PP([0,1))+\PP([1,1]) = \PP([0,1)\cup [1,1]) = \PP([0,1]) = 1
\end{align*}

Probabilities are at most 1, implying \( X_n\to_{a.s.} X \). \qed


\end{solution}

\begin{problem}[Exercise 6.3]
    Let \( \{X_n\}_{n\in\NN_0} \) and \( \{Y_n\}_{n\in\NN_0} \) be two sequences of random variables, defined on the sample space \( S \). Suppose that we know,
    \begin{align*}
        X_n \to _{a.s.} X && Y_n \to_{a.s.} Y
    \end{align*}
    Prove that \( X_n+Y_n \to_{a.s.} X+Y \).
\end{problem}

\begin{solution}

By hypothesis,
\begin{align*}
    \PP\left(\lim_{n\to\infty}|X_n-X|=0\right)=1 &&
    \PP\left(\lim_{n\to\infty}|Y_n-Y|=0\right)=1 
\end{align*}

The intersection of sets of measure 1 is still a set of measure 1. Thus,
\begin{align*}
    1 &= \PP \left( \lim_{n\to\infty} |X_n-X| = 0 \wedge \lim_{n\to \infty} |Y_n-Y| = 0 \right) \\ 
      &= \PP \left( \lim_{n\to\infty} |X_n-X| + \lim_{n\to \infty} |Y_n-Y| = 0 \right) \\ 
      &= \PP \left( \lim_{n\to\infty} |X_n-X| + |Y_n-Y| = 0 \right) 
\end{align*}

By the triangle inequality,
\begin{align*}
    |(X_n+Y_n) - (X+Y)| = |(X_n-X)+(Y_n-Y)| \leq |X_n-X| + |Y_n-Y|
\end{align*}

So, \( |X_n-X|+|Y_n-Y| = 0 \) implies \( |(X_n+Y_n) - (X+Y)| = 0 \). Thus,
\begin{align*}
    \left\{ \omega :  \lim_{n\to\infty} |X_n(\omega)-X(\omega)| + |Y_n(\omega)-Y(\omega)| = 0  \right\}
    \subseteq \left\{ \omega :  \lim_{n\to\infty} |(X_n(\omega)+Y_n(\omega)-(X(\omega)+Y(\omega))| = 0  \right\}
\end{align*}

Finally,
\begin{align*}
    \PP \left( \lim_{n\to\infty} |(X_n+Y_n) - (X+Y)| = 0 \right) \geq 1
\end{align*}

Probabilities are at most 1, implying \( X_n+Y_n \to_{a.s.}X+Y \). \qed


\end{solution}

\begin{problem}[Exercise 6.6]
    Let \( X_1,X_2, ...,  \) be independent with \( \PP(X_n=1) = p_n \) and \( \PP(X_n=0) = 1-p_n \). Show that,
    \begin{enumerate}
        \item[(a)] \( X_n \to_{p} 0 \) if and only if \( p_n\to 0 \).
        \item[(b)] \( X_n \to_{a.s.} 0 \) if and only if \( \sum_{n}p_n < \infty \)
    \end{enumerate}
\end{problem}

\begin{solution}

\begin{enumerate}
    \item[(a)] 
        Fix \( \varepsilon \in(0,1) \) and consider \( \PP(|X_n| > \varepsilon) \). For any \( \omega\in \Omega \), \( |X_n(w)|  > \varepsilon \) if \( X_n(\omega) = 1 \), and \( |X_n(\omega)| \leq \varepsilon \) if \( X_n(\omega)=0 \). In particular, this means that regardless of the value of \( \varepsilon \), \( \PP(|X_n| > \varepsilon) \geq \PP(X_n=1) = p_n \) and \( \PP(|X_n| \leq \varepsilon) \geq \PP(X_n=0) = 1-p_n \) so that \( \PP(|X_n|>\varepsilon) \leq p_n \).

         Thus, for any \( \varepsilon \in(0,1) \), \( \PP( |X_n| > \varepsilon ) = p_n \), and clearly if \( \varepsilon > 1 \) then \( \PP(X_n > \varepsilon) = 0 \). We then have,
        \begin{align*}
            X_n\to_{p}0 
            \Longleftrightarrow \forall \varepsilon>0, \lim_{n\to\infty} \PP(|X_n|>\varepsilon) = 0 
            \Longleftrightarrow \forall \varepsilon>0, \lim_{n\to\infty} p_n = 0
            \Longleftrightarrow X_n\to_{p} 0 \tag*{\qed}
        \end{align*}

        %        Note that we assumed the sample space of the \( X_i \) is \( \{0,1\} \). I guess it could be something else, where the probability of \( X_i \) being anything outside of \( \{0,1\} \) is zero. This doesn't really the result, it just makes it a bit more tedious to prove..

    \item[(b)]



         Suppose \( \sum_n \PP(\{\omega : X_n(\omega)=1\} = \sum_n p_n < \infty \). Then, by Borel-Cantelli Lemma we have,
        \begin{align*}
            0 = \PP(\{\omega: X_n(\omega) = 1, \text{ i.o.}\}) = \PP(\{\omega: \lim_{n\to\infty} |X_n(\omega)|\neq 0)
        \end{align*}
        Equivalently,
        \begin{align*}
            1 = \PP(\{\omega : \lim_{n\to\infty} X_n(\omega)=0\}) 
            \Longleftrightarrow X_n\to_{a.s.}0
        \end{align*}


        Now, suppose \( \sum_n \PP(\{\omega : X_n(\omega)=1\} = \sum_n p_n = \infty \). Then, by Borel-Cantelli Lemma, since \( X_n \) are independent meaning \( \{\omega:X_n(\omega)=1\} \) are independent, we have,
        \begin{align*}
            1 = \PP(\{\omega: X_n(\omega) = 1, \text{ i.o.}\})
              = \PP(\{\omega : \lim_{n\to\infty} X_n(\omega)\neq0\}) 
            \Longleftrightarrow X_n \not\to_{a.s.}0
        \end{align*}

        This proves that \( X_n \to_{a.s.}0  \) if and only if \( \sum_n p_n = 0 \). \qed



\end{enumerate}


\end{solution}

\begin{problem}[Exercise 6.7]
    Suppose that \( X_1, X_2, ..., \) are independent with \( \PP(X_n > x) = x^{-5} \) for all \( x\geq 1 \) and \( n=1,2,..., \). Show that \( \limsup_{n\to\infty}(\log X_n)/\log n = c \) almost surely for some number \( c \), and find \( c \).
\end{problem}

\begin{solution}

We have,
\begin{align*}
    \limsup_{n\to\infty} \{ (\log X_n)/\log n = c \} = \{ \omega : X_n(\omega)/\log n = c, \text{for infinitely many }n \}
\end{align*}

Fix \( n \in \NN, d \in \RR\). Consider\footnote{note that when \( \log n \) is in the denominator it isn't well defined for \( n=1 \). But we interpret it as if the equalities below are actually true},
\begin{align*}
    \PP(\log X_n / \log n > d) &= \PP( \log X_n > d \log n ) 
    = \PP(X_n > e^{d\log n}) 
    = \PP(X_n > n^{d} )
    = \left( n^d \right)^{-5}
    = n^{-5d}
\end{align*}

Take \( c=1/5 \) so that for any \( \varepsilon > 0 \),
\begin{align*}
    \sum_{n=1}^{\infty}\PP(\log X_n /\log n > c + \varepsilon ) &= \sum_{n=1}^{\infty} n^{-5(c+\varepsilon)} = \sum_{n=1}^{\infty} n^{-1-5\varepsilon}< \infty \\
    \sum_{n=1}^{\infty}\PP(\log X_n /\log n > c - \varepsilon ) &= \sum_{n=1}^{\infty} n^{-5(c-\varepsilon)} = \sum_{n=1}^{\infty} n^{-1+5\varepsilon} = \infty
\end{align*}

By Borel Cantelli, and since \( (A_n, \text{i.o.})^c = (A_n^c, \text{a.b.f.m})  \),
\begin{align*}
    \PP( \log X_n / \log n > c+\varepsilon, \text{ i.o.}) = 0 
    \Longleftrightarrow \PP( \log X_n / \log n < c+\varepsilon, \text{ a.b.f.m}) = 1 
\end{align*}
Since \( X_n \) are independent, then \( \{ \log X_n /\log n > c+\varepsilon\} \) are independent so, by Borel Cantelli,
\begin{align*}
    \PP( \log X_n / \log n > c-\varepsilon, \text{ i.o.}) = 1
\end{align*}

Together these show,
\begin{align*}
    \PP(\log X_n/\log n = c, \text{for infinitely many }n)
    =\PP\left(\limsup_{n\to\infty}\{(\log X_n)/\log n = 1/5 \}\right) = 1 \tag*{\qed}
\end{align*}

\end{solution}

\end{document}
