\documentclass[10pt]{article}
\usepackage[T1]{fontenc}

% Document Details
\newcommand{\CLASS}{AMATH 562}
\newcommand{\assigmentnum}{Assignment 10}

\usepackage[margin = 1.15in, top = 1.25in, bottom = 1.in]{geometry}

\usepackage{titling}
\setlength{\droptitle}{-6em}   % This is your set screw
\date{}
\renewcommand{\maketitle}{
	\clearpage
	\begingroup  
	\centering
	\LARGE \sffamily\textbf{\CLASS} \Large \assigmentnum\\[.8em]
	\large Tyler Chen\\[1em]
	\endgroup
	\thispagestyle{empty}
}
 % Title Styling
\usepackage{tocloft}
\renewcommand{\cfttoctitlefont}{\Large\sffamily\bfseries}
\renewcommand{\cftsecfont}{\normalfont\sffamily\bfseries}
\renewcommand{\cftsubsecfont}{\normalfont\sffamily}
\renewcommand{\cftsubsubsecfont}{\normalfont\sffamily}

\makeatletter
\let\oldl@section\l@section
\def\l@section#1#2{\oldl@section{#1}{\sffamily\bfseries#2}}

\let\oldl@subsection\l@subsection
\def\l@subsection#1#2{\oldl@subsection{#1}{\sffamily#2}}

\let\oldl@subsubsection\l@subsubsection
\def\l@subsubsection#1#2{\oldl@subsubsection{#1}{\sffamily#2}}
 % General Styling


\usepackage{enumitem}

% Figures
\usepackage{subcaption}

% TikZ and Graphics
\usepackage{tikz, pgfplots}
\pgfplotsset{compat=1.12}
\usetikzlibrary{patterns,arrows}
\usepgfplotslibrary{fillbetween}

\usepackage{pdfpages}
\usepackage{adjustbox}

\usepackage{lscape}
\usepackage{titling}
\usepackage[]{hyperref}


% Header Styling
\usepackage{fancyhdr}
\pagestyle{fancy}
\lhead{\sffamily \CLASS}
\rhead{\sffamily Chen \textbf{\thepage}}
\cfoot{}

% Paragraph Styling
\setlength{\columnsep}{1cm}
\setlength{\parindent}{0pt}
\setlength{\parskip}{5pt}
\renewcommand{\baselinestretch}{1}

% TOC Styling
\usepackage{tocloft}
\iffalse
\renewcommand{\cftsecleader}{\cftdotfill{\cftdotsep}}

\renewcommand\cftchapafterpnum{\vskip6pt}
\renewcommand\cftsecafterpnum{\vskip10pt}
\renewcommand\cftsubsecafterpnum{\vskip6pt}

% Adjust sectional unit title fonts in ToC
\renewcommand{\cftchapfont}{\sffamily}
\renewcommand{\cftsecfont}{\bfseries\sffamily}
\renewcommand{\cftsecnumwidth}{2em}
\renewcommand{\cftsubsecfont}{\sffamily}
\renewcommand{\cfttoctitlefont}{\hfill\bfseries\sffamily\MakeUppercase}
\renewcommand{\cftaftertoctitle}{\hfill}

\renewcommand{\cftchappagefont}{\sffamily}
\renewcommand{\cftsecpagefont}{\bfseries\sffamily}
\renewcommand{\cftsubsecpagefont}{\sffamily}
\fi
 % General Styling
% Code Display Setup
\usepackage{listings,lstautogobble}
\usepackage{lipsum}
\usepackage{courier}
\usepackage{catchfilebetweentags}

\lstset{
	basicstyle=\small\ttfamily,
	breaklines=true, 
	frame = single,
	rangeprefix=,
	rangesuffix=,
	includerangemarker=false,
	autogobble = true
}


\usepackage{algorithmicx}
\usepackage{algpseudocode}

\newcommand{\To}{\textbf{to}~}
\newcommand{\DownTo}{\textbf{downto}~}
\renewcommand{\algorithmicdo}{\hspace{-.2em}\textbf{:}}
 % Code Display Setup
% AMS MATH Styling
\usepackage{amsmath, amssymb}
\newcommand{\qed}{\hfill\(\square\)}

%\newtheorem*{lemma}{Lemma} 
%\newtheorem*{theorem}{Theorem}
%\newtheorem*{definition}{Definition}
%\newtheorem*{prop}{Proposition}
%\renewenvironment{proof}{{\bfseries Proof.}}{}


% mathcal
\newcommand{\cA}{\ensuremath{\mathcal{A}}}
\newcommand{\cB}{\ensuremath{\mathcal{B}}}
\newcommand{\cC}{\ensuremath{\mathcal{C}}}
\newcommand{\cD}{\ensuremath{\mathcal{D}}}
\newcommand{\cE}{\ensuremath{\mathcal{E}}}
\newcommand{\cF}{\ensuremath{\mathcal{F}}}
\newcommand{\cG}{\ensuremath{\mathcal{G}}}
\newcommand{\cH}{\ensuremath{\mathcal{H}}}
\newcommand{\cI}{\ensuremath{\mathcal{I}}}
\newcommand{\cJ}{\ensuremath{\mathcal{J}}}
\newcommand{\cK}{\ensuremath{\mathcal{K}}}
\newcommand{\cL}{\ensuremath{\mathcal{L}}}
\newcommand{\cM}{\ensuremath{\mathcal{M}}}
\newcommand{\cN}{\ensuremath{\mathcal{N}}}
\newcommand{\cO}{\ensuremath{\mathcal{O}}}
\newcommand{\cP}{\ensuremath{\mathcal{P}}}
\newcommand{\cQ}{\ensuremath{\mathcal{Q}}}
\newcommand{\cR}{\ensuremath{\mathcal{R}}}
\newcommand{\cS}{\ensuremath{\mathcal{S}}}
\newcommand{\cT}{\ensuremath{\mathcal{T}}}
\newcommand{\cU}{\ensuremath{\mathcal{U}}}
\newcommand{\cV}{\ensuremath{\mathcal{V}}}
\newcommand{\cW}{\ensuremath{\mathcal{W}}}
\newcommand{\cX}{\ensuremath{\mathcal{X}}}
\newcommand{\cY}{\ensuremath{\mathcal{Y}}}
\newcommand{\cZ}{\ensuremath{\mathcal{Z}}}

% mathbb
\usepackage{bbm}
\newcommand{\bOne}{\ensuremath{\mathbbm{1}}}

\newcommand{\bA}{\ensuremath{\mathbb{A}}}
\newcommand{\bB}{\ensuremath{\mathbb{B}}}
\newcommand{\bC}{\ensuremath{\mathbb{C}}}
\newcommand{\bD}{\ensuremath{\mathbb{D}}}
\newcommand{\bE}{\ensuremath{\mathbb{E}}}
\newcommand{\bF}{\ensuremath{\mathbb{F}}}
\newcommand{\bG}{\ensuremath{\mathbb{G}}}
\newcommand{\bH}{\ensuremath{\mathbb{H}}}
\newcommand{\bI}{\ensuremath{\mathbb{I}}}
\newcommand{\bJ}{\ensuremath{\mathbb{J}}}
\newcommand{\bK}{\ensuremath{\mathbb{K}}}
\newcommand{\bL}{\ensuremath{\mathbb{L}}}
\newcommand{\bM}{\ensuremath{\mathbb{M}}}
\newcommand{\bN}{\ensuremath{\mathbb{N}}}
\newcommand{\bO}{\ensuremath{\mathbb{O}}}
\newcommand{\bP}{\ensuremath{\mathbb{P}}}
\newcommand{\bQ}{\ensuremath{\mathbb{Q}}}
\newcommand{\bR}{\ensuremath{\mathbb{R}}}
\newcommand{\bS}{\ensuremath{\mathbb{S}}}
\newcommand{\bT}{\ensuremath{\mathbb{T}}}
\newcommand{\bU}{\ensuremath{\mathbb{U}}}
\newcommand{\bV}{\ensuremath{\mathbb{V}}}
\newcommand{\bW}{\ensuremath{\mathbb{W}}}
\newcommand{\bX}{\ensuremath{\mathbb{X}}}
\newcommand{\bY}{\ensuremath{\mathbb{Y}}}
\newcommand{\bZ}{\ensuremath{\mathbb{Z}}}

% alternative mathbb
\newcommand{\NN}{\ensuremath{\mathbb{N}}}
\newcommand{\RR}{\ensuremath{\mathbb{R}}}
\newcommand{\CC}{\ensuremath{\mathbb{C}}}
\newcommand{\ZZ}{\ensuremath{\mathbb{Z}}}
\newcommand{\EE}{\ensuremath{\mathbb{E}}}
\newcommand{\PP}{\ensuremath{\mathbb{P}}}
\newcommand{\VV}{\ensuremath{\mathbb{V}}}
\newcommand{\cov}{\ensuremath{\text{Co}\VV}}
% Math Commands

\newcommand{\st}{~\big|~}
\newcommand{\stt}{\text{ st. }}
\newcommand{\ift}{\text{ if }}
\newcommand{\thent}{\text{ then }}
\newcommand{\owt}{\text{ otherwise }}

\newcommand{\norm}[1]{\left\lVert#1\right\rVert}
\newcommand{\snorm}[1]{\lVert#1\rVert}
\newcommand{\ip}[1]{\ensuremath{\left\langle #1 \right\rangle}}
\newcommand{\pp}[3][]{\frac{\partial^{#1}#2}{\partial #3^{#1}}}
\newcommand{\dd}[3][]{\frac{\d^{#1}#2}{\d #3^{#1}}}
\renewcommand{\d}{\ensuremath{\mathrm{d}}}

\newcommand{\indep}{\rotatebox[origin=c]{90}{$\models$}}




 % Math shortcuts
% Problem
\usepackage{floatrow}

\newenvironment{problem}[1][]
{\pagebreak
\noindent\rule{\textwidth}{1pt}\vspace{0.25em}
{\sffamily \textbf{#1}}
\par
}
{\par\vspace{-0.5em}\noindent\rule{\textwidth}{1pt}}

\newenvironment{solution}[1][]
{{\sffamily \textbf{#1}}
\par
}
{}

 % Problem Environment

\newcommand{\note}[1]{\textcolor{red}{\textbf{Note:} #1}}

\hypersetup{
   colorlinks=true,       % false: boxed links; true: colored links
   linkcolor=violet,          % color of internal links (change box color with linkbordercolor)
   citecolor=green,        % color of links to bibliography
   filecolor=magenta,      % color of file links
   urlcolor=cyan           % color of external links
}


\begin{document}
\maketitle

\begin{problem}[Exercise 10.1]
    Let \( P=(P_t)_{t\geq 0} \) be a Poisson process with intensity \( \lambda \).
    \begin{enumerate}[label=(\alph*)]
        \item What is the L\'evy Measure \( \nu \) of \( P \).
        \item Let \( \d X_t = \d P_t \). Define \( u(x,t):=\EE \left[ \varphi(X_T) | X_t = x \right] \). Find \( u(t,x) \) and verify it solves the Kolmogorov Backward equation.
    \end{enumerate}
\end{problem}

\begin{solution}[Solution]
\begin{enumerate}[label=(\alph*)]
    \item 
        %If \( 1\notin U \), since \( \Delta P_t \in \{0,1\} \), then \( \nu(U) = 0 \). Assume \( 1\in U \). We then have,
        We have,
        \begin{align*}
            \nu(U) = \EE \left[ N(1,U) \right] 
            = \EE \left[ \sum_{0\leq s\leq 1} \mathbbm{1}_{\Delta P_s \in U} \right]
            = \EE \left[ \sum_{i=1}^{P_1} \mathbbm{1}_{1\in U} \right]
            = \EE \left[ P_1 \right] \mathbbm{1}_{1\in U}
            = \lambda \mathbbm{1}_{1\in U}
        \end{align*}
       

    \item Integrating \( \d X_t = \d P_t \) from \( 0 \) to \( t \) gives, \( X_t - X_0 = P_t - P_0 \). Since \( P_0 = 0 \) we have,
        \begin{align*}
            X_t = X_0 + P_t
        \end{align*}

        First observe,
        \begin{align*}
            \PP (X_T = k | X_t = x)
            =\PP (X_0 + P_T = k | X_0 + P_t = x)
            =\PP (P_T = k - X_0 | P_t = x - X_0)
        \end{align*}
    
        Since \( P \) has independent increments, and since \( P \) is Markov,
        \begin{align*}
            \PP (P_T = k - X_0 | P_t = x - X_0)
            =\PP (P_{T-t} = k - x)
            = \dfrac{(\lambda(T-t))^{k-x}}{(k-x)!}e^{-\lambda (T-t)}
        \end{align*} 
        
        Thus,
        \begin{align*}
            u(t,x) = \EE \left[ \varphi(X_T) | X_t = x \right]
            = \sum_{k=x}^{\infty} \varphi(k) \PP (X_T = k | X_t = x)
            = \sum_{k=x}^{\infty} \varphi(k) \dfrac{(\lambda(T-t))^{k-x}}{(k-x)!}e^{-\lambda (T-t)}
        \end{align*}
        
        Reindexing with \( n= k-x \),
        \begin{align*}
            u(t,x)  
            = e^{-\lambda (T-t)} \sum_{k=x}^{\infty} \varphi(k) \dfrac{(\lambda(T-t))^{k-x}}{(k-x)!}
            = e^{-\lambda (T-t)} \sum_{n=0}^{\infty} \varphi(n+x) \dfrac{(\lambda(T-t))^{n}}{n!}
        \end{align*}
        

        We now compute the generator \( \mathcal{A} (t) \) for \( P \). By definition,
        \begin{align*}
            \mathcal{A} (t) \varphi(x) 
            = \lim_{s\to t^+} \dfrac{1}{s-t} \left[ \mathcal{P}(t,s) \varphi(x) - \varphi(x)  \right]
            = \lim_{s\to t^+} \dfrac{1}{s-t} \left[ \EE \left[ \varphi(X_s) | X_t=x \right] - \varphi(x)  \right]
        \end{align*}
       
        In a small interval \( \d t \) the probability \( X_{t+\d t} = X_t + 1 \) is \( \lambda \d t \) and probability \( X_{t+\d t} = X_t \) is \( (1-\lambda)\d t \). Therefore,
        \begin{align*}
            \mathcal{A} (t) \varphi(x) = \dfrac{1}{\d t} \left[\varphi(x+1) \lambda + \varphi(x) (1-\lambda) - \varphi(x) \right]
            = \lambda(\varphi(x+1) - \varphi(x))
        \end{align*}
        
    
        Since the \( t \)-derivative of the \( n=0 \) term is zero,
        \begin{align*}
            \sum_{n=0}^{\infty} \varphi(n+x) \partial_t \left[ \dfrac{(\lambda(T-t))^n}{n!} \right]
            &= \sum_{n=1}^{\infty} \varphi(n+x) \partial_t \left[ \dfrac{(\lambda(T-t))^n}{n!} \right]
            \\&= \sum_{n=1}^{\infty} \varphi(n+x) (n)(-\lambda)\dfrac{(\lambda(T-t))^{n-1}}{n!}
            \\&= -\lambda\sum_{n=1}^{\infty} \varphi(n+x) \dfrac{(\lambda(T-t))^{n-1}}{(n-1)!}
        \end{align*}
        
        Observe, by the chain rule and assuming we can bring a derivative through a sum,
        \begin{align*}
            \partial_t u(t,x) 
            &= \left[ \partial_t e^{-\lambda(T-t)} \right] \sum_{n=0}^{\infty} \varphi(n+x) \dfrac{(\lambda(T-t))^n}{n!} 
            + e^{-\lambda(T-t)} \sum_{n=0}^{\infty} \varphi(n+x) \partial_t \left[ \dfrac{(\lambda(T-t))^n}{n!} \right]
            \\&= \lambda e^{-\lambda(T-t)} \sum_{n=0}^{\infty} \varphi(n+x) \dfrac{(\lambda(T-t))^n}{n!} 
            - \lambda e^{-\lambda(T-t)} \sum_{n=1}^{\infty} \varphi(n+x) \dfrac{(\lambda(T-t))^{n-1}}{(n-1)!} 
            \\&= \lambda e^{-\lambda(T-t)} \sum_{n=0}^{\infty} \varphi(n+x) \dfrac{(\lambda(T-t))^n}{n!} 
            - \lambda e^{-\lambda(T-t)} \sum_{n=m}^{\infty} \varphi(m+1+x) \dfrac{(\lambda(T-t))^{m}}{m!}
            \\&= \lambda(u(t,x) - u(t,x+1))
        \end{align*}
        
        Therefore the KBE is satisfied as
        \begin{align*}
            [\partial_t+\mathcal{A}] u(t,x)
            = \lambda(u(t,x) - u(t,x+1)) - \lambda(u(t,x+1)-u(t,x)) = 0,
            && 
            u(T,x) = \varphi(x) 
        \end{align*}
         
\end{enumerate}
\end{solution}

\begin{problem}[Exercise 10.2]

\end{problem}

\begin{solution}[Solution]
    
\end{solution}



\begin{problem}[Exercise 10.3]
    Let \( X = (X_t)_{t\geq 0} \) be a process defined by,
    \begin{align*}
        \d X_t &= \mu_t X_t \d t + \sigma_t X_t \d W_t + \int_{\RR} \left( e^{\gamma_t(z)}-1 \right)X_{t^-} \tilde{N}(\d t, \d z) \\
        \d Y_t &= b_t Y_t \d t + a_t Y_t \d W_t + \int_{\RR} \left( e^{g_t(z)}-1 \right)Y_{t^-}\tilde{N}(\d t,\d z)
    \end{align*} 
    where \( W \) is a one-dimensional Brownian motion, \( \tilde{N} \) is a one-dimensional compensated Poisson random measure on \( \RR \), and \( \mu,b,\sigma,a,\gamma, g\) are \( \mathbb{F} \)-adapted stochastic processes.
    \begin{enumerate}[label=(\alph*)]
        \item Define \( Z_t:=X_t/Y_t \). Compute the differential \( \d Z_t \). Your answer should not involve \( X_t \) or \( Y_t \).
        \item Find \( \mu_t \) so that \( Z \) is a martingale.
    \end{enumerate}
\end{problem}


\begin{solution}[Solution]
\begin{enumerate}[label=(\alph*)]
    \item Define \( f(x,y) = x/y \). Then \( Z_t = f(X_t,Y_t) \).

        We have,
        \begin{align*}
            [(e^{\gamma_t(z)}-1)X_t; (e^{g_t(z)}-1)Y_t] \cdot \nabla f(X_{t^-},Y_{t^-}) 
            = (e^{\gamma_t(z)}-1) X_{t^-} f_x(X_{t^-},Y_{t^-}) + (e^{g_t(z)}-1)Y_{t^-} f_y(X_{t^-},Y_{t^-})
        \end{align*}

        We use It\^o's formula to compute,
        \begin{align*}
            \d Z_t = \d f(X_t,Y_t)
            &= \left( \mu_t X_t f_x + b_tY_t f_y
            + \frac{1}{2} \left( (\sigma_t X_t)^2 f_{xx} + 2(\sigma_t X_t)(a_tY_) f_{xy} + (a_tY_t)^2f_{yy} \right) \right) \d t
            \\& \hspace{1.5em}+ \left( \sigma_t X_t f_x + a_t Y_t f_y \right) \d W_t
            \\& \hspace{2em}+ \int_{\RR} \left( f \left(X_{t^-}+(e^{\gamma_t(z)}-1)X_{t^-},Y_{t^-} + (e^{g_t(z)}-1)Y_{t^-} \right)  - f(X_{t^-},Y_{t^-}) \right) \tilde{N}(\d t,\d z)
            \\& \hspace{2.5em}+ \int_{\RR} \Big( f \left(X_{t^-}+(e^{\gamma_t(z)}-1)X_{t^-},Y_{t^-} + (e^{g_t(z)}-1)Y_{t^-} \right)  - f(X_{t^-},Y_{t^-}) 
            \\&  \hspace{6em} - (e^{\gamma_t(z)}-1) X_{t^-} f_x(X_{t^-},Y_{t^-}) - (e^{g_t(z)}-1)Y_{t^-} f_y(X_{t^-},Y_{t^-}) \Big) \nu(\d z) \d t
        \end{align*}
       

        Now, using \( f_x = 1/y, f_y = -x/y^2, f_{xy} = -1/y^2, f_{xx} = 0, f_{yy} = 2x/y^3 \) we have,
        \begin{align*}
             \mu_t X_t f_x + b_tY_t f_y %\right) \d t
            = \mu_t X_t \left( \dfrac{1}{Y_t} \right) + b_t Y_t \left( \dfrac{-X_t}{Y_t^2} \right)
            = \mu_t Z_t - b_t Z_t
        \end{align*}
        \begin{align*}
            (\sigma_t X_t)^2 f_{xx} + 2(\sigma_t X_t)(a_tY_t) f_{xy} +  (a_tY_t)^2 f_{yy} 
            = 2(\sigma_t X_t)(a_tY_t) \left( \dfrac{-1}{Y_t^2} \right) + a_t^2 Y_t^2 \left( \dfrac{2X_t}{Y_t^3} \right)
            %= a_t^2 \dfrac{X_t}{Y_t} \d t
            = -2\sigma_ta_tZ_t + 2a_t^2 Z_t 
        \end{align*}
        \begin{align*}
            \sigma_t X_t f_x + a_t Y_t f_y 
            = \sigma_t X_t \left( \dfrac{1}{Y_t} \right) + a_t Y_t \left( \dfrac{-X_t}{Y_t^2} \right)
            = \sigma_t Z_t - a_t Z_t 
        \end{align*}
        \begin{align*}
             f \left(X_{t^-}+(e^{\gamma_t(z)}-1)X_{t^-},Y_{t^-} + (e^{g_t(z)}-1)Y_{t^-} \right)  - f(X_{t^-},Y_{t^-}) 
            =  \dfrac{e^{\gamma_t(z)}}{e^{g_t(z)}} Z_{t^-} - Z_{t^-}
        \end{align*}
        \begin{align*}    
            & \phantom{{}={}}(e^{\gamma_t(z)}-1) X_{t^-} f_x(X_{t^-},Y_{t^-}) + (e^{g_t(z)}-1)Y_{t^-} f_y(X_{t^-},Y_{t^-})
            \\&= (e^{\gamma_t(z)}-1) X_{t^-} \left( \dfrac{1}{Y_{t^-}} \right)
            + (e^{g_t(z)}-1) Y_{t^-} \left( \dfrac{-X_{t^-}}{Y_{t^-}^2} \right)
            \\&= (e^{\gamma_t(z)}-1) Z_{t^-} - (e^{g_t(z)}-1)Z_{t^-}
        \end{align*}
       
        Inserting these evaluated expressions into the original expression for \( \d Z_t \) gives,
        \begin{align*}
            \d Z_t &= \left( \mu_t - b_t - \sigma_t a_t + a_t^2 \right)Z_t \d t
            + \left( \sigma_t - a_t \right) Z_t \d W_t
            \\& \hspace{2em}+ \int_{\RR} \left( \dfrac{e^{\gamma_t(z)}}{e^{g_t(z)}}-1 \right)Z_{t^-} \tilde{N}(\d t, \d z)
            \\& \hspace{3em}+ \int_{\RR} \left( \dfrac{e^{\gamma_t(z)}}{e^{g_t(z)}} - e^{\gamma_t(z)}+e^{g_t(z)}-1 \right)Z_{t^-} \nu(\d z)\d t
        \end{align*}
        
    \item 
        We need the \( \d t \) term to be zero. Therefore pick,
        \begin{align*}
            \mu_t = b_t + \sigma_t a_t - a_t^2
            -\int_{\RR} \left( \dfrac{e^{\gamma_t(z)}}{e^{g_t(z)}} - e^{\gamma_t(z)}+e^{g_t(z)}-1 \right) \nu(\d z)\d t
        \end{align*}
\end{enumerate}
\end{solution}

\begin{problem}[Exercise 10.4]
    Let \( \eta = (\eta_t)_{t\geq 0} \) be a one-dimensional L\'evy Process and define \( X = (X_t)_{t\geq 0} \) by
    \begin{align*}
        \d X_t = \kappa(\theta - X_t)\d t + \d \eta_t
    \end{align*}
    
    \begin{enumerate}[label=(\alph*)]
        \item Find \( X_t \) explicitly as a function of \( \eta \).
        \item Assume \( \eta_t = \sigma  W_t + \int_\RR z \tilde{N}(t,\d z) \). Compute \( m(t):= \EE X_t \) and \( c(t,s):= \EE(X_t-m(t))(X_s-m(s)) \).
    \end{enumerate}
\end{problem}

\begin{solution}[Solution]
\begin{enumerate}[label=(\alph*)]
    \item 
        Let \( Y_t = X_t - \theta \) and \( Z_t = e^{\kappa t}Y_t = f(t,Y_t)  \), where \( f(t,y) = e^{\kappa t}y \).

        Then,
        \begin{align*}
            \d Y_t = \d X_t = -\kappa Y_t \d t + \d \eta_t
        \end{align*}

        Recall the product rule (which applies to L\'evy It\^o processes),
        \begin{align*}
            \d(U_tV_t) = U_{t^-}\d V_t + V_{t^-}\d U_t + \d[U,V]_t
        \end{align*}
        

        Therefore,
        \begin{align*}
            \d Z_t 
            = \d (e^{\kappa t} Y_t) 
            = e^{\kappa t^-}\d Y_t + Y_{t^-} \d e^{\kappa t} + \d[e^{\kappa t},Y]_t 
        \end{align*}
      

        % Since \( e^{\kappa t} \) is has a continuous derivative then, \( [e^{\kappa t},Y]_t = 0 \) \textbf{VERIFY THIS IS TRUE!!!!}. Therefire,

        Using our heuristics we have \( \d (e^{\kappa t}) \d Y_t = 0 \). Therefore, since \( t^- \) and \( t \) can be ``treated the same'' on \( \d t \) terms which are continuous,
        \begin{align*}
            \d Z_t = e^{\kappa t^-} \d Y_t + \kappa e^{\kappa t} Y_{t^-} 
            = e^{\kappa t^-} \d \eta_t
        \end{align*}
        
        Integrating we have,
        \begin{align*}
            Z_t = Z_0 + \int_{0}^{t} e^{\kappa s} \d \eta_s
        \end{align*}

        Therefore, since \( Y_t = e^{-\kappa t}Z_t \), \( Z_0 = Y_0 \) so,
        \begin{align*}
            Y_t 
            = e^{-\kappa t} \left( Y_0 + \int_{0}^{t} e^{\kappa s} \d \eta_s \right)
        \end{align*}
        
        Finally, since \( X_t = \theta + Y_t \), \( Y_0 = X_0 - \theta \) so,
        \begin{align*}
            X_t = \theta + e^{-\kappa t} \left( X_0 - \theta + \int_{0}^{t} e^{\kappa s} \d \eta_s \right)
            = \theta + e^{-\kappa t}(X_0 - \theta) + \int_{0}^{t} e^{\kappa(s-t)} \d \eta_s
        \end{align*}
        
        
    \item
        We have,
        \begin{align*}
            \d \eta_t = \sigma \d W_t + \int_{\RR} z \tilde{N}(\d t, \d z)
        \end{align*}
       
        Observe, that since integrals with respect to \( \d W_t \) and \( \int_{\RR} \tilde{N}(\d t, \d z) \) are martingales so,
        \begin{align*}
            \EE \left[ \int_{0}^{t} e^{\kappa(s-t)} \d \eta_s \right]
            = \EE \left[ \int_{0}^{t} e^{\kappa(s-t)} \sigma \d W_t + \int_{0}^{t} e^{\kappa(s-t)} \int_{\RR} z \tilde{N}(\d t, \d z) \right] = 0
        \end{align*}
        
        Therefore,
        \begin{align*}
            m(t) = \EE \left[ X_t \right]
            = \EE \left[ \theta + e^{-\kappa t}(X_0 - \theta) + \int_{0}^{t} e^{\kappa(s-t)} \d \eta_s \right]
            = \theta + e^{-\kappa t}(X_0 - \theta)
        \end{align*}
        

        Clearly,
        \begin{align*}
            X_t - m(t) = \int_{0}^{t}e^{\kappa(u-t)} \d \eta_u
        \end{align*}

        \iffalse
        Observe,
        \begin{align*}
            \EE[\d\eta_u \d\eta_v]
            &= \EE \left[ \sigma^2 \d W_u \d W_v + \d W_u \int_{\RR} z \tilde{N}(\d v, \d z) + \d W_v \int_{\RR} y \tilde{N} (\d u, \d y) + \iint_{\RR^2} y z \tilde{N}(\d u, \d u) \tilde{N} (\d v, \d z)  \right]
            \\&= \EE \left[ \sigma^2 \d W_u \d W_v \right] + \EE\left[\int_{\RR} z \d W_u \tilde{N}(\d v, \d z) \right] 
            \\& \hspace{5em} + \EE\left [ \int_{\RR} y \d W_v \tilde{N} (\d u, \d y)\right] + \EE \left[\iint_{\RR^2} y z \tilde{N}(\d u, \d y) \tilde{N} (\d v, \d z)  \right]
            \\&= \sigma^2 \delta_{u,v} \d u + 0 + 0 + \iint_{\RR^2}\delta_{u,v} \delta(z-y) N(\d u, \d z)\d y
        \end{align*}
        \fi

        Without loss of generality assume \( t\geq s \). Then, using the independent increments property to write the expectation of a product as the product of expectations,
        \begin{align*}
            \EE \left[ \left( X_t - m(t) \right) \left( X_s-m(s) \right) \right]
            &= \EE \left[ \left( \int_{0}^{t} e^{\kappa(u-t)} \d \eta_u \right) \left( \int_{0}^{s}e^{\kappa(v-s)}\d \eta_v \right) \right]
            \\&= \EE \left[ \left( \int_{0}^{s} e^{\kappa(u-t)} \d \eta_u + \int_{s}^{t} e^{\kappa(u-t)} \d \eta_u\right) \left( \int_{0}^{s}e^{\kappa(v-s)}\d \eta_v \right) \right]
            \\&= \EE \left[ e^{-\kappa(t+s)} \left(\int_{0}^{s} e^{\kappa u} \d \eta_u \right)^2 + e^{-\kappa(t+s)}\left( \int_{s}^{t} e^{\kappa u} \d \eta_u\right) \left( \int_{0}^{s}e^{\kappa v}\d \eta_v \right) \right] 
            \\&= e^{-\kappa(t+s)}\EE \left[ \left(\int_{0}^{s} e^{\kappa u} \d \eta_u \right)^2 \right] + e^{-\kappa(t+s)} \EE \left[ \int_{s}^{t} e^{\kappa u} \d \eta_u\right] \EE\left[ \int_{0}^{s}e^{\kappa v}\d \eta_v \right] 
        \end{align*}
        
        We now note that, L\'evy processes without a \( \d t \) term are martingales so that,
        \begin{align*}
            \EE \left[ \int_{0}^{s} e^{\kappa u} \d \eta_u \right]
            = \EE \left[ \int_{0}^{s} e^{\kappa u} \left(\sigma \d W_u + \int_\RR z \tilde{N}(\d u, \d z) \right) \right] 
            = 0
        \end{align*}
        
        Define,
        \begin{align*}
            Z_s = \int_{0}^{s} e^{\kappa u} \d \eta_u
        \end{align*}

        Then,
        \begin{align*}
            \d Z_s = e^{\kappa s} \d \eta_s 
            = \sigma e^{\kappa s} \d W_s + \int_{\RR} e^{\kappa s} z \tilde{N}(\d s,\d z)
        \end{align*}
        
        
        Using It\^o's isometry we have,
        \begin{align*}
            \EE \left[ \left( \int_{0}^{s} e^{\kappa u} \d \eta_u \right)^2 \right]
            = \EE \left[ \int_{0}^{s} \left(\sigma^2 e^{2\kappa u} + \int_\RR e^{2\kappa u}z^2 \nu(\d z) \right)\d u \right]
            = \EE\left[ \left(\sigma^2 + \int_\RR z^2 \nu(\d z) \right) \dfrac{e^{2\kappa s}-1}{2\kappa} \right] 
        \end{align*}

        Therefore, 
        \begin{align*}
            c(t,s) 
            = e^{-\kappa(t+s)}\dfrac{e^{2\kappa s}-1}{2\kappa} \left(\sigma^2 + \int_\RR z^2 \nu(\d z) \right)
            = \dfrac{e^{\kappa(s-t)}-e^{-\kappa(t+s)}}{2\kappa} \left(\sigma^2 + \int_\RR z^2 \nu(\d z) \right)
        \end{align*} 

        We can remove our assumption that \( t\geq s \) and write,
        \begin{align*}
            c(t,s)  
            = \dfrac{e^{-\kappa|t-s|} - e^{-\kappa(t+s)}}{2\kappa} \left(\sigma^2 + \int_\RR z^2 \nu(\d z) \right)
        \end{align*}
\end{enumerate}
\end{solution}

\begin{problem}[Exercise 10.5]
Let \( X \) be the following one-dimensional jump-diffusion
\begin{align*}
    \d X_t = \mu(t,X_t)\d t + \sigma(t,X_t)\d W_t + \int_{\RR} \gamma(t,X_{t^-},z) \tilde{N}(t,\d z),
\end{align*}
    where \( W \) is a one-dimensional Brownian motion and \( \tilde{N} \) is a one-dimensional compensated Poisson random measure on \( \RR \). Derive using the L\'evy-It\^o formula the infinitesimal generator \( \mathcal{A} (t) \) of the \( X \) process,
    \begin{align*}
        \mathcal{A}(t) \varphi(x) := \lim_{s\to t^+} \dfrac{\EE \left[ \varphi(X_s) | X_t = x \right]-\varphi(x)}{s-t}
    \end{align*}
\end{problem}


\begin{solution}[Solution]
Since \( \EE[\varphi(X_t)|X_t = x] = \varphi(x) \),
\begin{align*}
    \EE \left[ \varphi(X_s) | X_t = x \right] - \varphi(x)
    = \EE \left[ \varphi(X_t) + \int_{t}^{s}\d \varphi(X_u) \right] - \varphi(x)
    = \EE \left[ \int_{t}^{s}\d \varphi(X_u) \right]
\end{align*}



From the L\'evy-It\^o formula we have,
\begin{align*}
    \d \varphi(X_u)
    &= \left( \mu(u,X_u) \varphi'(X_u) + \dfrac{1}{2}\sigma(u,X_u)^2\varphi''(X_u)\right)\d u
    + \sigma(u,X_u)\varphi'(X_u) \d W_u
    \\&\hspace{2em}+ \int_{\RR} \Big(\varphi(X_{u^-}+\gamma(u,X_{u^-},z)) - \varphi(X_{u^-}) \Big) \tilde{N}(\d u, \d z)
    \\&\hspace{3em}+ \int_{\RR} \Big(\varphi(X_{u^-}+\gamma(u,X_{u^-},z)) - \varphi(X_{u^-}) - \gamma(u,X_{u^-},z)\varphi'(X_{u^-})\Big) \nu(\d z) \d u
\end{align*}

We note that as integrals with respect to \( W \) and \( \tilde{N}  \) are martingales that,
\begin{align*}
    \EE \left[ \int_{t}^{s} \d \varphi(X_u) \right] 
    &= \EE \Bigg[ \int_{t}^{s} \bigg( \mu(u,X_u) \varphi'(X_u) + \dfrac{1}{2}\sigma(u,X_u)^2\varphi''(X_u)\d u
    \\&\hspace{3em}+ \int_{\RR} \Big(\varphi(X_{u^-}+\gamma(u,X_{u^-},z)) - \varphi(X_{u^-}) - \gamma(u,X_{u^-},z)\varphi'(X_{u^-})\Big) \nu(\d z) \bigg) \d u \Bigg]
\end{align*}

Thus, taking the limit as \( s\to t^+ \),
\begin{align*}
    \mathcal{A} (t) \varphi(x) 
    %&= \lim_{s\to t^+} \dfrac{1}{s-t} \EE \left[ \int_{s}^{t} \d \varphi(X_u) \right]
    &= \left( \mu(t,X_t) \partial_x + \dfrac{1}{2} \sigma(t,X_t) \partial_x^2 + \int_{\RR} \nu(\d z) \left(\theta_{\gamma(t,X_t,z)} - 1 - \gamma(t,X_t,z) \partial_x\right) \right)\varphi(x)
\end{align*}
\end{solution}


\end{document}
