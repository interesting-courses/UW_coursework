\documentclass[10pt]{article}
\usepackage[T1]{fontenc}

% Document Details
\newcommand{\CLASS}{AMATH 562}
\newcommand{\assigmentnum}{Assignment 8}

\usepackage[margin = 1.15in, top = 1.25in, bottom = 1.in]{geometry}

\usepackage{titling}
\setlength{\droptitle}{-6em}   % This is your set screw
\date{}
\renewcommand{\maketitle}{
	\clearpage
	\begingroup  
	\centering
	\LARGE \sffamily\textbf{\CLASS} \Large \assigmentnum\\[.8em]
	\large Tyler Chen\\[1em]
	\endgroup
	\thispagestyle{empty}
}
 % Title Styling
\usepackage{tocloft}
\renewcommand{\cfttoctitlefont}{\Large\sffamily\bfseries}
\renewcommand{\cftsecfont}{\normalfont\sffamily\bfseries}
\renewcommand{\cftsubsecfont}{\normalfont\sffamily}
\renewcommand{\cftsubsubsecfont}{\normalfont\sffamily}

\makeatletter
\let\oldl@section\l@section
\def\l@section#1#2{\oldl@section{#1}{\sffamily\bfseries#2}}

\let\oldl@subsection\l@subsection
\def\l@subsection#1#2{\oldl@subsection{#1}{\sffamily#2}}

\let\oldl@subsubsection\l@subsubsection
\def\l@subsubsection#1#2{\oldl@subsubsection{#1}{\sffamily#2}}
 % General Styling


\usepackage{enumitem}

% Figures
\usepackage{subcaption}

% TikZ and Graphics
\usepackage{tikz, pgfplots}
\pgfplotsset{compat=1.12}
\usetikzlibrary{patterns,arrows}
\usepgfplotslibrary{fillbetween}

\usepackage{pdfpages}
\usepackage{adjustbox}

\usepackage{lscape}
\usepackage{titling}
\usepackage[]{hyperref}


% Header Styling
\usepackage{fancyhdr}
\pagestyle{fancy}
\lhead{\sffamily \CLASS}
\rhead{\sffamily Chen \textbf{\thepage}}
\cfoot{}

% Paragraph Styling
\setlength{\columnsep}{1cm}
\setlength{\parindent}{0pt}
\setlength{\parskip}{5pt}
\renewcommand{\baselinestretch}{1}

% TOC Styling
\usepackage{tocloft}
\iffalse
\renewcommand{\cftsecleader}{\cftdotfill{\cftdotsep}}

\renewcommand\cftchapafterpnum{\vskip6pt}
\renewcommand\cftsecafterpnum{\vskip10pt}
\renewcommand\cftsubsecafterpnum{\vskip6pt}

% Adjust sectional unit title fonts in ToC
\renewcommand{\cftchapfont}{\sffamily}
\renewcommand{\cftsecfont}{\bfseries\sffamily}
\renewcommand{\cftsecnumwidth}{2em}
\renewcommand{\cftsubsecfont}{\sffamily}
\renewcommand{\cfttoctitlefont}{\hfill\bfseries\sffamily\MakeUppercase}
\renewcommand{\cftaftertoctitle}{\hfill}

\renewcommand{\cftchappagefont}{\sffamily}
\renewcommand{\cftsecpagefont}{\bfseries\sffamily}
\renewcommand{\cftsubsecpagefont}{\sffamily}
\fi
 % General Styling
% Code Display Setup
\usepackage{listings,lstautogobble}
\usepackage{lipsum}
\usepackage{courier}
\usepackage{catchfilebetweentags}

\lstset{
	basicstyle=\small\ttfamily,
	breaklines=true, 
	frame = single,
	rangeprefix=,
	rangesuffix=,
	includerangemarker=false,
	autogobble = true
}


\usepackage{algorithmicx}
\usepackage{algpseudocode}

\newcommand{\To}{\textbf{to}~}
\newcommand{\DownTo}{\textbf{downto}~}
\renewcommand{\algorithmicdo}{\hspace{-.2em}\textbf{:}}
 % Code Display Setup
% AMS MATH Styling
\usepackage{amsmath, amssymb}
\newcommand{\qed}{\hfill\(\square\)}

%\newtheorem*{lemma}{Lemma} 
%\newtheorem*{theorem}{Theorem}
%\newtheorem*{definition}{Definition}
%\newtheorem*{prop}{Proposition}
%\renewenvironment{proof}{{\bfseries Proof.}}{}


% mathcal
\newcommand{\cA}{\ensuremath{\mathcal{A}}}
\newcommand{\cB}{\ensuremath{\mathcal{B}}}
\newcommand{\cC}{\ensuremath{\mathcal{C}}}
\newcommand{\cD}{\ensuremath{\mathcal{D}}}
\newcommand{\cE}{\ensuremath{\mathcal{E}}}
\newcommand{\cF}{\ensuremath{\mathcal{F}}}
\newcommand{\cG}{\ensuremath{\mathcal{G}}}
\newcommand{\cH}{\ensuremath{\mathcal{H}}}
\newcommand{\cI}{\ensuremath{\mathcal{I}}}
\newcommand{\cJ}{\ensuremath{\mathcal{J}}}
\newcommand{\cK}{\ensuremath{\mathcal{K}}}
\newcommand{\cL}{\ensuremath{\mathcal{L}}}
\newcommand{\cM}{\ensuremath{\mathcal{M}}}
\newcommand{\cN}{\ensuremath{\mathcal{N}}}
\newcommand{\cO}{\ensuremath{\mathcal{O}}}
\newcommand{\cP}{\ensuremath{\mathcal{P}}}
\newcommand{\cQ}{\ensuremath{\mathcal{Q}}}
\newcommand{\cR}{\ensuremath{\mathcal{R}}}
\newcommand{\cS}{\ensuremath{\mathcal{S}}}
\newcommand{\cT}{\ensuremath{\mathcal{T}}}
\newcommand{\cU}{\ensuremath{\mathcal{U}}}
\newcommand{\cV}{\ensuremath{\mathcal{V}}}
\newcommand{\cW}{\ensuremath{\mathcal{W}}}
\newcommand{\cX}{\ensuremath{\mathcal{X}}}
\newcommand{\cY}{\ensuremath{\mathcal{Y}}}
\newcommand{\cZ}{\ensuremath{\mathcal{Z}}}

% mathbb
\usepackage{bbm}
\newcommand{\bOne}{\ensuremath{\mathbbm{1}}}

\newcommand{\bA}{\ensuremath{\mathbb{A}}}
\newcommand{\bB}{\ensuremath{\mathbb{B}}}
\newcommand{\bC}{\ensuremath{\mathbb{C}}}
\newcommand{\bD}{\ensuremath{\mathbb{D}}}
\newcommand{\bE}{\ensuremath{\mathbb{E}}}
\newcommand{\bF}{\ensuremath{\mathbb{F}}}
\newcommand{\bG}{\ensuremath{\mathbb{G}}}
\newcommand{\bH}{\ensuremath{\mathbb{H}}}
\newcommand{\bI}{\ensuremath{\mathbb{I}}}
\newcommand{\bJ}{\ensuremath{\mathbb{J}}}
\newcommand{\bK}{\ensuremath{\mathbb{K}}}
\newcommand{\bL}{\ensuremath{\mathbb{L}}}
\newcommand{\bM}{\ensuremath{\mathbb{M}}}
\newcommand{\bN}{\ensuremath{\mathbb{N}}}
\newcommand{\bO}{\ensuremath{\mathbb{O}}}
\newcommand{\bP}{\ensuremath{\mathbb{P}}}
\newcommand{\bQ}{\ensuremath{\mathbb{Q}}}
\newcommand{\bR}{\ensuremath{\mathbb{R}}}
\newcommand{\bS}{\ensuremath{\mathbb{S}}}
\newcommand{\bT}{\ensuremath{\mathbb{T}}}
\newcommand{\bU}{\ensuremath{\mathbb{U}}}
\newcommand{\bV}{\ensuremath{\mathbb{V}}}
\newcommand{\bW}{\ensuremath{\mathbb{W}}}
\newcommand{\bX}{\ensuremath{\mathbb{X}}}
\newcommand{\bY}{\ensuremath{\mathbb{Y}}}
\newcommand{\bZ}{\ensuremath{\mathbb{Z}}}

% alternative mathbb
\newcommand{\NN}{\ensuremath{\mathbb{N}}}
\newcommand{\RR}{\ensuremath{\mathbb{R}}}
\newcommand{\CC}{\ensuremath{\mathbb{C}}}
\newcommand{\ZZ}{\ensuremath{\mathbb{Z}}}
\newcommand{\EE}{\ensuremath{\mathbb{E}}}
\newcommand{\PP}{\ensuremath{\mathbb{P}}}
\newcommand{\VV}{\ensuremath{\mathbb{V}}}
\newcommand{\cov}{\ensuremath{\text{Co}\VV}}
% Math Commands

\newcommand{\st}{~\big|~}
\newcommand{\stt}{\text{ st. }}
\newcommand{\ift}{\text{ if }}
\newcommand{\thent}{\text{ then }}
\newcommand{\owt}{\text{ otherwise }}

\newcommand{\norm}[1]{\left\lVert#1\right\rVert}
\newcommand{\snorm}[1]{\lVert#1\rVert}
\newcommand{\ip}[1]{\ensuremath{\left\langle #1 \right\rangle}}
\newcommand{\pp}[3][]{\frac{\partial^{#1}#2}{\partial #3^{#1}}}
\newcommand{\dd}[3][]{\frac{\d^{#1}#2}{\d #3^{#1}}}
\renewcommand{\d}{\ensuremath{\mathrm{d}}}

\newcommand{\indep}{\rotatebox[origin=c]{90}{$\models$}}




 % Math shortcuts
% Problem
\usepackage{floatrow}

\newenvironment{problem}[1][]
{\pagebreak
\noindent\rule{\textwidth}{1pt}\vspace{0.25em}
{\sffamily \textbf{#1}}
\par
}
{\par\vspace{-0.5em}\noindent\rule{\textwidth}{1pt}}

\newenvironment{solution}[1][]
{{\sffamily \textbf{#1}}
\par
}
{}

 % Problem Environment

\newcommand{\note}[1]{\textcolor{red}{\textbf{Note:} #1}}

\hypersetup{
   colorlinks=true,       % false: boxed links; true: colored links
   linkcolor=violet,          % color of internal links (change box color with linkbordercolor)
   citecolor=green,        % color of links to bibliography
   filecolor=magenta,      % color of file links
   urlcolor=cyan           % color of external links
}


\begin{document}
\maketitle

\begin{problem}[Exercise 8.1]
    Compute \( \d(W_t^4) \). Write \( W_T^4 \) as an integral with respect to \( W \) plus an integral with respect to \( t \). Use this representation of \( W_T^4 \) to show that \( \EE W_T^4 = 3T^2 \). Compute \( \EE W_T^6 \) using the same technique.
\end{problem}

\begin{solution}[Solution]
Write \( f(x) = x^4 \) so that \( f(W_t) = W_t^4 \). Then, \( f'(x) = 4x^3 \) and \( f''(x) = 12x^2 \). Therefore, It\^o's formula gives,
\begin{align*}
    \d W_t^4 &= f'(W_t)\d W_t + \frac{1}{2}f''(W_t) \d[W,W]_t 
    = 4W_t^3 \d W_t + \frac{12}{2} W_t^2 \d[W,W]_t 
\end{align*}

Thus, writing \( \d[W,W]_t = \d t \) we have,
\begin{align*}
    \d W_t^4 = 4W_t^3 \d W_t + 6W_t^2 \d t
\end{align*}
    
Thus, since \( W_0 = 0 \),
\begin{align*}
    W_T^4 = W_T^4 - W_0^4 = 4\int_{0}^{T} W_t^3\d W_t + 6 \int_{0}^{T}W_t^2 \d t
\end{align*}

Recall It\^o integrals are martingales so that,
\begin{align*}
    \EE \left[ \int_{0}^{T}W_t^3\d W_t \right] = 0
    %\EE \left[ \EE \left[ \int_{0}^{T}W_t^3 \d W_t \Bigg| \mathcal{F}_T \right] \right] = \EE \left[ \int_{0}^{0}W_t^3 \d W_t \right] = 0
\end{align*}

Note also that since \( \EE \left[ W_t^2 \right] = t \),
\begin{align*}
    \EE \left[ \int_{0}^{T}W_t^2 \d t \right] = \int_{0}^{T}\EE \left[ W_t^2 \right]\d t = \int_{0}^{T} t \d t = \dfrac{T^2}{2}
\end{align*}

Therefore,
\begin{align*}
    \EE \left[ W_T^4 \right] &= 4\EE \left[ \int_{0}^{T}W_t^3 \d W_t \right] + 6 \EE \left[ \int_{0}^{T}W_t^2 \d t \right] = 6\dfrac{T^2}{2} = 3 T^2
\end{align*}

Similarly, we have,
\begin{align*}
    W_T^6 = 6\int_{0}^{T} W_t^5 \d W_t + \dfrac{6\cdot 5}{2}\int_{0}^{T} W_t^4 \d t 
    %= 30 \int_{0}^{T} \dfrac{t^3}{3} \d t = 30 \dfrac{T^4}{12} = \dfrac{5}{2}T^4 
\end{align*}

Therefore, since \( \EE \left[ W_t^4 \right] = 3t^2 \),
\begin{align*}
    \EE\left[W_T^6\right] = 6 \EE\left[\int_{0}^{T} W_t^5 \d W_t\right] + 15 \EE\left[\int_{0}^{T} W_t^4 \d t \right] = 15 \int_{0}^{T} \EE \left[ W_t^4 \right] \d t = 15 \int_{0}^{T} 3t^2 \d t = 15T^3 
\end{align*}
\end{solution}

\begin{problem}[Exercise 8.2]
Find an explicit expression for \( Y_T \) where,
\begin{align*}
    \d Y_t = r \d t + \alpha Y_t \d W_t
\end{align*}

    Hint: Multiply the above equation by \( F_t := \exp(- \alpha W_t + \frac{1}{2} \alpha^2t) \).

\end{problem}


\begin{solution}[Solution]
Let \( f(x,y) = \exp(-\alpha x + \frac{1}{2} \lambda^2 y) \) so that,
\begin{align*}
    f_x(W_t,t) = -\alpha F_t && f_y(W_t,t) = \frac{\alpha^2}{2} F_t && f_{xx}(W_t,t) =  \alpha^2 F_t
\end{align*}

Then \( F_t = f(W_t,t) \), so by It\^o's formula and the heuristic \( (\d W_t)^2 =\d t, (\d t)^2 = \d t \d W_t = 0 \),
\begin{align*}
    \d F_t = \d f(W_t,t) &= f_y(W_t,t) \d t +  f_x(W_t,t) \d W_t + \frac{1}{2} f_{xx}(W_t,t) (\d W_t)^2 
    \\&= \frac{\alpha^2}{2}  F_t \d t -\alpha F_t \d W_t + \frac{\alpha^2}{2} F_t \d t
    \\&= \alpha^2 F_t \d t- \alpha F_t \d W_t
\end{align*}


Using our heuristics we have,
\begin{align*}
    \d[F,Y]_t = (\d F_t)(\d Y_t) = \left( \alpha^2 F_t \d t - \alpha F_t \d W_t \right) \left( r \d t + \alpha Y_t \d W_t \right)
    = -\alpha^2 F_tY_t (\d W_t)^2 
    = -\alpha^2 F_tY_t \d t
\end{align*}


By the product rule we have,
\begin{align*}
    \d (F_tY_t) &=  F_t \d Y_t + Y_t \d F_t + \d[F,Y]_t
    \\&= F_t (r\d t+\alpha Y_t \d W_t) + Y_t (\alpha^2 F_t \d t-\alpha F_t \d W_t) - \alpha^2 F_t Y_t \d t
    \\&= r F_t \d t 
\end{align*}

In integral form,
\begin{align*}
    F_t Y_t - F_0Y_0 = \int_{0}^{t}r F_s \d s = \int_{0}^{t} r e^{-\alpha W_s + \frac{1}{2} \alpha^2 s} \d s
\end{align*}

We can add \( F_0Y_0 = Y_0 \) and divide by \( F_t \) yielding,
\begin{align*}
    Y_t = Y_0 + r e^{\alpha W_t - \frac{1}{2}\alpha^2 t} \int_{0}^{t} e^{-\alpha W_s + \frac{1}{2} \alpha^2 s} \d s
\end{align*}
\end{solution}

\begin{problem}[Exercise 8.3]
Suppose \( X \), \( \Delta \), and \( \Pi \) are given by,
\begin{align*}
    \d X_t = \sigma X_t \d W_t, 
    && \Delta_t = \pp{f}{x} (t,X_t),
    && \Pi_t = X_t \Delta_t
\end{align*}
where \( f \) is some smooth function. Show that if \( f \) satisfies,
\begin{align*}
    \left( \pp{}{t} + \dfrac{1}{2}\sigma^2x^2 \pp[2]{}{x} \right) f(t,x) = 0
\end{align*}
    for all \( (t,x) \), then \( \Pi \) is a martingale with respect to a filtration \( \mathcal{F}_t \) for \( W \).
\end{problem}


\begin{solution}[Solution]
We have,
\begin{align*}
    \pp{}{x} \left( \pp{}{t} + \dfrac{1}{2}\sigma^2x^2 \pp[2]{}{x}\right)  
    = \dfrac{\partial^2}{\partial x\partial t} + \dfrac{1}{2}\sigma^2 \left[ x^2 \pp[3]{}{x} + 2x \pp[2]{}{x} \right]
\end{align*}

Thus, using the condition for \( f \) we have,
\begin{align*}
    \dfrac{\partial^2 f}{\partial x\partial t} + \dfrac{1}{2}\sigma^2 X_t^2 \pp[3]{f}{x} = - \sigma^2 X_t \pp[2]{f}{x}
\end{align*}

Using our heuristics we have,
\begin{align*}
    \d[X,X] = \sigma^2 X_t^2 (\d W_t)^2 = \sigma^2 X_t^2 \d t
\end{align*}

Similarly, 
\begin{align*}
    \d[X,t] = \d[t,X] = \d[t,t] = 0
\end{align*}

Therefore, by It\^o's formula,
\begin{align*}
%    \d \Delta_t &=  F_x(t,X_t) \d t + F_t(t,X_t) \d X_t + \dfrac{1}{2} F_{xx}(t,X_t) \d [X,X]
    \d \Delta_t &= \dfrac{\partial^2 f}{\partial x\partial t}(t,X_t) \d t + \pp[2]{f}{x}(t,X_t) \d X_t + \dfrac{1}{2} \d[X,X]
    \\&=  \dfrac{\partial^2 f}{\partial x\partial t}(t,X_t) \d t + \sigma X_t \pp[2]{f}{x}(t,X_t) \d W_t + \dfrac{1}{2}\sigma^2X_t^2 \pp[3]{f}{x}(t,X_t) \d t
    \\&= -\sigma^2 X_t \pp[2]{f}{x}(t,X_t) \d t + \sigma X_t \pp[2]{f}{x}(t,X_t)\d W_t 
\end{align*}

Therefore,
\begin{align*}
    \d[X,\Delta]_t = (\d X_t)(\d \Delta_t) 
    = \sigma^2 X_t^2 \dfrac{\partial^2f}{\partial x^2}(t,X_t) (\d W_t)^2 
    = \sigma^2 X_t^2 \dfrac{\partial^2f}{\partial x^2}(t,X_t) \d t
\end{align*}


Finally, we have,
\begin{align*}
    \d \Pi_t &= \d(X_t \Delta _t) = X_t \d\Delta _t + \Delta_t \d X_t + \d[X,\Delta]_t \\
    &= X_t \left( -\sigma^2 X_t \pp[2]{f}{x}(t,X_t) \d t + \sigma X_t \pp[2]{f}{x}(t,X_t)\d W_t \right) 
    + \sigma X_t \pp{f}{x}(t,X_t) \d W_t 
    + \sigma^2 X_t^2 \pp[2]{f}{x} \d t 
    \\&= \sigma X_t \left( X_t \pp[2]{f}{x}(t,X_t) + \pp{f}{x}(t,X_t)\right) \d W_t
\end{align*}

Since there is no \( \d t \) dependence this is an It\^o integral and therefore a martingale with respect to a filtration for \( W \). (there are probably some technical assumptions we need about \( X \) and \( f \), but in class we never dealt with these)\qed

\end{solution}

\begin{problem}[Exercise 8.4]
Suppose \( X \) is given by,
\begin{align*}
    \d X_t = \mu(t,X_t)dt + \sigma(t,X_t) \d W_t
\end{align*}
For any smooth function \( f \) define,
\begin{align*}
    M_t^f:= f(t,X_t) - f(0,X_0) - \int_{0}^{t} \left( \pp{}{s} + \mu(s,X_s)\pp{}{x} + \dfrac{1}{2}\sigma^2(s,X_s) \pp[2]{}{x} \right) f(s,X_s)ds
\end{align*}
    Show that \( M^f \) is a martingale with respect to a filtration \( \mathcal{F}_t \) for \( W \).
\end{problem}


\begin{solution}[Solution]
We first compute,
\begin{align*}
    \d[X,X]_t = (\d X_t)(\d X_t) = \sigma^2(t,X_t) (\d W_t)^2 = \sigma^2(t,X_t)\d t
\end{align*}

We then have,
\begin{align*}
    \d f(t,X_t) &= \pp{f}{t}(t,X_t) \d t + \pp{f}{x}(t,X_t) \d X_t + \dfrac{1}{2}\pp[2]{f}{x}\d[X,X]_t 
    \\&= \pp{f}{t}(t,X_t) \d t + \pp{f}{x}(t,X_t)  [\mu(t,X_t)\d t +  \sigma(t,X_t)\d W_t] + \dfrac{1}{2}\sigma^2(t,X_t) \pp[2]{f}{x} \d t 
    \\&= \left( \pp{}{t} + \mu(t,X_t) \pp{}{x} + \dfrac{1}{2}\sigma^2(t,X_t) \pp[2]{}{x} \right) f(t,X_t) \d t + \sigma(t,X_t) \pp{f}{x} \d W_t 
\end{align*}

Finally, since \( f(0,X_0) \) is a constant,
\begin{align*}
    \d M_t^f &= \d f(t,X_t) - \left( \pp{}{t} + \mu(t,X_t)\pp{}{x} + \dfrac{1}{2}\sigma^2(t,X_t) \pp[2]{}{x} \right) f(t,X_t) \d t \\
    &= \sigma(t,X_t) \pp{f}{x} \d W_t 
\end{align*}

Since there is no \( \d t \) dependence this an It\^o integral and therefore a martingale with respect to a filtration for \( W \). \qed
\end{solution}


\end{document}
