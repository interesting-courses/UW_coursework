\documentclass[10pt]{article}
\usepackage[T1]{fontenc}

% Document Details
\newcommand{\CLASS}{AMATH 561}
\newcommand{\assigmentnum}{Assignment 2}

\usepackage[margin = 1.15in, top = 1.25in, bottom = 1.in]{geometry}

\usepackage{titling}
\setlength{\droptitle}{-6em}   % This is your set screw
\date{}
\renewcommand{\maketitle}{
	\clearpage
	\begingroup  
	\centering
	\LARGE \sffamily\textbf{\CLASS} \Large \assigmentnum\\[.8em]
	\large Tyler Chen\\[1em]
	\endgroup
	\thispagestyle{empty}
}
 % Title Styling
\usepackage{tocloft}
\renewcommand{\cfttoctitlefont}{\Large\sffamily\bfseries}
\renewcommand{\cftsecfont}{\normalfont\sffamily\bfseries}
\renewcommand{\cftsubsecfont}{\normalfont\sffamily}
\renewcommand{\cftsubsubsecfont}{\normalfont\sffamily}

\makeatletter
\let\oldl@section\l@section
\def\l@section#1#2{\oldl@section{#1}{\sffamily\bfseries#2}}

\let\oldl@subsection\l@subsection
\def\l@subsection#1#2{\oldl@subsection{#1}{\sffamily#2}}

\let\oldl@subsubsection\l@subsubsection
\def\l@subsubsection#1#2{\oldl@subsubsection{#1}{\sffamily#2}}
 % General Styling


\usepackage{enumitem}

% Figures
\usepackage{subcaption}

% TikZ and Graphics
\usepackage{tikz, pgfplots}
\pgfplotsset{compat=1.12}
\usetikzlibrary{patterns}
\usepgfplotslibrary{fillbetween}

\usepackage{pdfpages}
\usepackage{adjustbox}

\usepackage{lscape}
\usepackage{titling}
\usepackage[]{hyperref}


% Header Styling
\usepackage{fancyhdr}
\pagestyle{fancy}
\lhead{\sffamily \CLASS}
\rhead{\sffamily \textbf{\thepage}}
\cfoot{}

% Paragraph Styling
\setlength{\columnsep}{1cm}
\setlength{\parindent}{0pt}
\setlength{\parskip}{5pt}
\renewcommand{\baselinestretch}{1}

% TOC Styling
\usepackage{tocloft}
\iffalse
\renewcommand{\cftsecleader}{\cftdotfill{\cftdotsep}}

\renewcommand\cftchapafterpnum{\vskip6pt}
\renewcommand\cftsecafterpnum{\vskip10pt}
\renewcommand\cftsubsecafterpnum{\vskip6pt}

% Adjust sectional unit title fonts in ToC
\renewcommand{\cftchapfont}{\sffamily}
\renewcommand{\cftsecfont}{\bfseries\sffamily}
\renewcommand{\cftsecnumwidth}{2em}
\renewcommand{\cftsubsecfont}{\sffamily}
\renewcommand{\cfttoctitlefont}{\hfill\bfseries\sffamily\MakeUppercase}
\renewcommand{\cftaftertoctitle}{\hfill}

\renewcommand{\cftchappagefont}{\sffamily}
\renewcommand{\cftsecpagefont}{\bfseries\sffamily}
\renewcommand{\cftsubsecpagefont}{\sffamily}
\fi
 % General Styling
% Code Display Setup
\usepackage{listings,lstautogobble}
\usepackage{lipsum}
\usepackage{courier}
\usepackage{catchfilebetweentags}

\lstset{
	basicstyle=\small\ttfamily,
	breaklines=true, 
	frame = single,
	rangeprefix=,
	rangesuffix=,
	includerangemarker=false,
	autogobble = true
}


\usepackage{algorithmicx}
\usepackage{algpseudocode}

\newcommand{\To}{\textbf{to}~}
\newcommand{\DownTo}{\textbf{downto}~}
\renewcommand{\algorithmicdo}{\hspace{-.2em}\textbf{:}}
 % Code Display Setup
% AMS MATH Styling
\usepackage{amsmath, amssymb}
\newcommand{\qed}{\hfill\(\square\)}

%\newtheorem*{lemma}{Lemma} 
%\newtheorem*{theorem}{Theorem}
%\newtheorem*{definition}{Definition}
%\newtheorem*{prop}{Proposition}
%\renewenvironment{proof}{{\bfseries Proof.}}{}


% mathcal
\newcommand{\cA}{\ensuremath{\mathcal{A}}}
\newcommand{\cB}{\ensuremath{\mathcal{B}}}
\newcommand{\cC}{\ensuremath{\mathcal{C}}}
\newcommand{\cD}{\ensuremath{\mathcal{D}}}
\newcommand{\cE}{\ensuremath{\mathcal{E}}}
\newcommand{\cF}{\ensuremath{\mathcal{F}}}
\newcommand{\cG}{\ensuremath{\mathcal{G}}}
\newcommand{\cH}{\ensuremath{\mathcal{H}}}
\newcommand{\cI}{\ensuremath{\mathcal{I}}}
\newcommand{\cJ}{\ensuremath{\mathcal{J}}}
\newcommand{\cK}{\ensuremath{\mathcal{K}}}
\newcommand{\cL}{\ensuremath{\mathcal{L}}}
\newcommand{\cM}{\ensuremath{\mathcal{M}}}
\newcommand{\cN}{\ensuremath{\mathcal{N}}}
\newcommand{\cO}{\ensuremath{\mathcal{O}}}
\newcommand{\cP}{\ensuremath{\mathcal{P}}}
\newcommand{\cQ}{\ensuremath{\mathcal{Q}}}
\newcommand{\cR}{\ensuremath{\mathcal{R}}}
\newcommand{\cS}{\ensuremath{\mathcal{S}}}
\newcommand{\cT}{\ensuremath{\mathcal{T}}}
\newcommand{\cU}{\ensuremath{\mathcal{U}}}
\newcommand{\cV}{\ensuremath{\mathcal{V}}}
\newcommand{\cW}{\ensuremath{\mathcal{W}}}
\newcommand{\cX}{\ensuremath{\mathcal{X}}}
\newcommand{\cY}{\ensuremath{\mathcal{Y}}}
\newcommand{\cZ}{\ensuremath{\mathcal{Z}}}

% mathbb
\usepackage{bbm}
\newcommand{\bOne}{\ensuremath{\mathbbm{1}}}

\newcommand{\bA}{\ensuremath{\mathbb{A}}}
\newcommand{\bB}{\ensuremath{\mathbb{B}}}
\newcommand{\bC}{\ensuremath{\mathbb{C}}}
\newcommand{\bD}{\ensuremath{\mathbb{D}}}
\newcommand{\bE}{\ensuremath{\mathbb{E}}}
\newcommand{\bF}{\ensuremath{\mathbb{F}}}
\newcommand{\bG}{\ensuremath{\mathbb{G}}}
\newcommand{\bH}{\ensuremath{\mathbb{H}}}
\newcommand{\bI}{\ensuremath{\mathbb{I}}}
\newcommand{\bJ}{\ensuremath{\mathbb{J}}}
\newcommand{\bK}{\ensuremath{\mathbb{K}}}
\newcommand{\bL}{\ensuremath{\mathbb{L}}}
\newcommand{\bM}{\ensuremath{\mathbb{M}}}
\newcommand{\bN}{\ensuremath{\mathbb{N}}}
\newcommand{\bO}{\ensuremath{\mathbb{O}}}
\newcommand{\bP}{\ensuremath{\mathbb{P}}}
\newcommand{\bQ}{\ensuremath{\mathbb{Q}}}
\newcommand{\bR}{\ensuremath{\mathbb{R}}}
\newcommand{\bS}{\ensuremath{\mathbb{S}}}
\newcommand{\bT}{\ensuremath{\mathbb{T}}}
\newcommand{\bU}{\ensuremath{\mathbb{U}}}
\newcommand{\bV}{\ensuremath{\mathbb{V}}}
\newcommand{\bW}{\ensuremath{\mathbb{W}}}
\newcommand{\bX}{\ensuremath{\mathbb{X}}}
\newcommand{\bY}{\ensuremath{\mathbb{Y}}}
\newcommand{\bZ}{\ensuremath{\mathbb{Z}}}

% alternative mathbb
\newcommand{\NN}{\ensuremath{\mathbb{N}}}
\newcommand{\RR}{\ensuremath{\mathbb{R}}}
\newcommand{\CC}{\ensuremath{\mathbb{C}}}
\newcommand{\ZZ}{\ensuremath{\mathbb{Z}}}
\newcommand{\EE}{\ensuremath{\mathbb{E}}}
\newcommand{\PP}{\ensuremath{\mathbb{P}}}
\newcommand{\VV}{\ensuremath{\mathbb{V}}}
\newcommand{\cov}{\ensuremath{\text{Co}\VV}}
% Math Commands

\newcommand{\st}{~\big|~}
\newcommand{\stt}{\text{ st. }}
\newcommand{\ift}{\text{ if }}
\newcommand{\thent}{\text{ then }}
\newcommand{\owt}{\text{ otherwise }}

\newcommand{\norm}[1]{\left\lVert#1\right\rVert}
\newcommand{\snorm}[1]{\lVert#1\rVert}
\newcommand{\ip}[1]{\ensuremath{\left\langle #1 \right\rangle}}
\newcommand{\pp}[3][]{\frac{\partial^{#1}#2}{\partial #3^{#1}}}
\newcommand{\dd}[3][]{\frac{\d^{#1}#2}{\d #3^{#1}}}
\renewcommand{\d}{\ensuremath{\mathrm{d}}}

\newcommand{\indep}{\rotatebox[origin=c]{90}{$\models$}}




 % Math shortcuts
% Problem
\usepackage{floatrow}

\newenvironment{problem}[1][]
{\pagebreak
\noindent\rule{\textwidth}{1pt}\vspace{0.25em}
{\sffamily \textbf{#1}}
\par
}
{\par\vspace{-0.5em}\noindent\rule{\textwidth}{1pt}}

\newenvironment{solution}[1][]
{{\sffamily \textbf{#1}}
\par
}
{}

 % Problem Environment

\newcommand{\note}[1]{\textcolor{red}{\textbf{Note:} #1}}

\hypersetup{
   colorlinks=true,       % false: boxed links; true: colored links
   linkcolor=violet,          % color of internal links (change box color with linkbordercolor)
   citecolor=green,        % color of links to bibliography
   filecolor=magenta,      % color of file links
   urlcolor=cyan           % color of external links
}


\begin{document}
\maketitle

\begin{problem}[Exercise 2.1]
    Let \( \Omega=\{a,b,c,d\} \) and \( \mathcal{F} = 2^\Omega \) (the set of all subsets of \( \Omega \)). We define a probability measure \( \PP \) as follows
    \begin{align*}
        \PP(a)=1/6, && \PP(b)=1/3, && \PP(c)=1/4, && \PP(d)=1/4
    \end{align*}
    Next, define three random variables,
    \begin{align*}
        X(a)=1, && X(b)=1, && X(c)=-1, && X(d)=-1\\
        Y(a)=1, && Y(b)=-1, && Y(c)=1, && Y(d)=-1,
    \end{align*}
    and \( Z=X+Y \).
    \begin{enumerate}
        \item[(a)] List the sets in \( \sigma(X) \).
        \item[(b)] What are the values of \( \EE[Y|X] \) for \( \{a,b,c,d\} \)? Verify the partial averaging property: \( \EE[\mathbbm{1}_A\EE[Y|X]]=\EE[\mathbbm{1}_A Y] \).
        \item[(c)] What are the values of \( \EE[Z|X] \) for \( \{a,b,c,d\} \)? Verify the partial averaging property.
    \end{enumerate}
\end{problem}

\begin{solution}[Solution]
\begin{enumerate}
    \item[(a)] Recall that \( \sigma(X) = \{ \{X\in A\}\subseteq\Omega : A\in\mathcal{B}(\RR) \} = \{\{w : X(w)\in A\} : A\in\mathcal{B}(\RR)\} \). Therefore,
        \begin{align*}
            \sigma(X) = \{ \emptyset,\{a,b\},\{c,d\}, \Omega \}
        \end{align*}
    \item[(b)] We compute, 
        \begin{align*}
            \EE[Y|X](a) &= \EE[Y|X=X(a)] = \EE[Y|X=1] = \dfrac{1\PP(a)-1\PP(b)}{\PP(a)+\PP(b)} = -\dfrac{1}{3} \\
            \EE[Y|X](b) &= \EE[Y|X=X(b)] = \EE[Y|X=1] = \dfrac{1\PP(a)-1\PP(b)}{\PP(a)+\PP(b)} = -\dfrac{1}{3} \\ 
            \EE[Y|X](c) &= \EE[Y|X=X(c)] = \EE[Y|X=-1] = \dfrac{1\PP(c)-1\PP(d)}{\PP(c)+\PP(d)} = 0 \\
            \EE[Y|X](d) &= \EE[Y|X=X(d)] = \EE[Y|X=-1] = \dfrac{1\PP(c)-1\PP(d)}{\PP(c)+\PP(d)} = 0 \\
        \end{align*}

        For each set \( A\in\sigma(X) \) we verify that \( \EE[\mathbbm{1}_A\EE[Y|X]]=\EE[\mathbbm{1}_A Y] \) as follows,        
        
        \begin{align*}
        \begin{array}{c|r|r}
             A  &  \EE[\mathbbm{1}_A\EE[Y|X]]  & \EE[\mathbbm{1}_AY]  \\ \hline   
             \emptyset  & 0 & 0 \\
             \{a,b\}  &  -\frac{1}{3}\PP(a)-\frac{1}{3}\PP(b) = -\frac{1}{6}  &  1\PP(a)-1\PP(b) = -\frac{1}{6}  \\
             \{c,d\}  &  0\PP(c)+0\PP(d) =\phantom{-} 0  &  1\PP(c)-1\PP(d) =\phantom{-} 0  \\
             \Omega  &  -\frac{1}{3}\PP(a)-\frac{1}{3}\PP(b)+0\PP(c)+0\PP(d) = -\frac{1}{6}  &  1\PP(a)-1\PP(b)+1\PP(c)-1\PP(d) = -\frac{1}{6}               \end{array}
        \end{align*}

    \item[(c)]  Write,
        \begin{align*}
            Z(a) = 2, && Z(b) = 0, && Z(c) = 0, && Z(d) = -2
        \end{align*}
        We compute,
        \begin{align*}
            \EE[Z|X](a) &= \EE[Z|X=X(a)] = \EE[Z|X=1] = \dfrac{2\PP(a)+0\PP(b)}{\PP(a)+\PP(b)} = \dfrac{2}{3} \\
            \EE[Z|X](b) &= \EE[Z|X=X(b)] = \EE[Z|X=1] = \dfrac{2\PP(a)+0\PP(b)}{\PP(a)+\PP(b)} = \dfrac{2}{3} \\
            \EE[Z|X](c) &= \EE[Z|X=X(c)] = \EE[Z|X=-1] = \dfrac{0\PP(c)+2\PP(d)}{\PP(c)+\PP(d)} = -1 \\
            \EE[Z|X](d) &= \EE[Z|X=X(d)] = \EE[Z|X=-1] = \dfrac{0\PP(c)+2\PP(d)}{\PP(c)+\PP(d)} = -1 \\
        \end{align*}
        
        For each set \( A\in\sigma(X) \) we verify that \( \EE[\mathbbm{1}_A\EE[Z|X]]=\EE[\mathbbm{1}_A Z] \) as follows,        
        
        \begin{align*}
        \begin{array}{c|r|r}
             A  &  \EE[\mathbbm{1}_A\EE[Z|X]]  & \EE[\mathbbm{1}_A Z]  \\ \hline   
             \emptyset  & 0 & 0 \\
             \{a,b\}  &  \frac{2}{3}\PP(a)+\frac{2}{3}\PP(b) = \frac{1}{3}  &  2\PP(a)+0\PP(b) = \frac{1}{3}  \\
             \{c,d\}  &  -1\PP(c)+-1\PP(d) = -\frac{1}{2}  &  0\PP(c)-2\PP(d) = -\frac{1}{2}  \\
        \Omega  &  \frac{2}{3}\PP(a)+\frac{2}{3}\PP(b)-1\PP(c)-1\PP(d) = -\frac{1}{6}  &  2\PP(a)+0\PP(b)+0\PP(c)-2\PP(d) = -\frac{1}{6}               \end{array}
        \end{align*}

\end{enumerate}
\end{solution}

\begin{problem}[Exercise 2.2]
    Fix a probability space \( (\Omega,\mathcal{F},\PP) \). Let \( Y \) be a square integrable random variable: \( \EE Y^2<\infty \) and let \( \mathcal{G} \) be a sub-\( \sigma \)-algebra of \( \mathcal{F} \). Show that
    \begin{align*}
        \mathbb{V}(Y-\EE[Y | \mathcal{G}])\leq \mathbb{V}(Y-X) && \forall X\in\mathcal{G}
    \end{align*}
\end{problem}

\begin{solution}[Solution]

Suppose further \( \EE[(Y-X)^2]<\infty \) (we make this assumption so that \( \mathbb{V}[Y-X] \) exists).

Clearly \( (\EE[Y|\mathcal{G}]-X)\in \mathcal{G} \). Then, by partial averaging, \( \EE[(\EE[Y|\mathcal{G}]-X)\EE[Y|\mathcal{G}]] = \EE[(\EE[Y|\mathcal{G}]-X)Y] \). Therefore,
\begin{align*}
    \EE[(\EE[Y|\mathcal{G}]-X)(Y-\EE[Y|\mathcal{G}])] &= \EE[(\EE[Y|\mathcal{G}]-X)Y]-\EE[(\EE[Y|\mathcal{G}]-X)\EE[Y|\mathcal{G}]] \\
    &= \EE[(\EE[Y|\mathcal{G}]-X)Y] - \EE[(\EE[Y|\mathcal{G}]-X)Y] \\
    &= 0
\end{align*}

Then, since \( \EE[(Y-X)^2] \), exists,
\begin{align*}
    \EE[(Y-X)^2] &= \EE[((Y-\EE[Y|\mathcal{G}])+(\EE[Y|\mathcal{G}]-X))^2] \\
    &= \EE[(Y-\EE[Y|\mathcal{G}])^2]+\EE[(\EE[Y|\mathcal{G}]-X)^2] + 2\EE[(Y-\EE[Y|\mathcal{G}])(\EE[Y|\mathcal{G}]-X)] \\
    &= \EE[(Y-\EE[Y|\mathcal{G}])^2]+\EE[(\EE[Y|\mathcal{G}]-X)^2]
\end{align*}

Again by partial averaging, \( \EE[\EE[Y|\mathcal{G}]] = \EE[\mathbbm{1}_{\Omega}\EE[Y|\mathcal{G}]] = \EE[\mathbbm{1}_{\Omega}Y] = \EE[Y] \) so that \( \EE[Y-\EE[Y|G]] = 0 \). Then,
\begin{align*}
    \EE[Y-X]^2 &= \EE[Y-\EE[Y|\mathcal{G}]+\EE[Y|\mathcal{G}]-X]^2 \\
    &= \left(\EE[Y-\EE[Y|\mathcal{G}]]+\EE[\EE[Y|\mathcal{G}]-X]\right)^2 \\
    &= \EE[Y-\EE[Y|\mathcal{G}]]^2+\EE[\EE[Y|\mathcal{G}]-X]^2+2\EE[Y-\EE[Y|\mathcal{G}]]\EE[\EE[Y|\mathcal{G}]-X] \\
    &= \EE[Y-\EE[Y|\mathcal{G}]]^2+\EE[\EE[Y|\mathcal{G}]-X]^2
\end{align*}

Thus, subtracting this result from the first,
\begin{align*}
    \EE[(Y-X)^2]-\EE[Y-X]^2 &= \EE[(Y-\EE[Y|\mathcal{G}])^2]-\EE[Y-\EE[Y|\mathcal{G}]]^2+\EE[(\EE[Y|\mathcal{G}]-X)^2]-\EE[\EE[Y|\mathcal{G}]-X]^2 \\
    \mathbb{V}[Y-X] &= \mathbb{V}[Y-\EE[Y|\mathcal{G}]]+\mathbb{V}[\EE[Y|\mathcal{G}]-X]
\end{align*}

Therefore, since \( \mathbb{V}[\EE[Y|\mathcal{G}-X]\geq 0 \), for any \( X\in\mathcal{G} \), 
\begin{align*}
    \mathbb{V}[Y-X] &\geq \mathbb{V}[Y-\EE[Y|\mathcal{G}]] \tag*{\qed}
\end{align*}

\problem{Exercise 2.3}{%
    Give an example of a probability space \( (\Omega,\mathcal{F},\PP) \), a random variable \( X \) and a function \( f \) such that \( \sigma(f(X)) \) is strictly smaller than \( \sigma(X) \) but \( \sigma(f(X))\neq\{\emptyset,\Omega\} \). Give a function \( g \) such that \( \sigma(g(X))=\{\emptyset,\Omega\} \).
}

Let \( \Omega = \{a,b,c\} \) and \( \mathcal{F}=2^\Omega \). Define \( \PP(a)=\PP(b)=\PP(c)=1/3 \).

Define \( X \) as \( X(a)=0,~ X(b)=-1,~ X(c)=1 \). 

Thus, \( \sigma(X) = \{\emptyset, \{a\}, \{b\}, \{c\}, \{b,c\}, \{a,c\}, \{a,b\}, \Omega \} \)

Since \( X(\Omega)\subset\RR \), define \( f,g:\RR\to\RR \) by \( f(x)=x^2 \) and \( g(x) = 0 \).
Then \( f(X(a))=0,~ f(X(b)) = f(X(c)) = 1 \) and \( g(Y(a))=g(Y(b))=g(Y(c))=0 \).

Therefore \( \sigma(f(X)) = \{ \emptyset, \{a\},\{b,c\} ,\Omega \} \) so \( \sigma(f(X))\subsetneq \sigma(X) \) so \( \sigma(f(X)) \) is strictly smaller than \( \sigma(X) \).

Similarly, \( \sigma(g(X)) = \{\emptyset, \Omega\}  \).
\end{solution}

\begin{problem}[Exercise 2.4]
On a probability space \( (\Omega,\mathcal{F},\PP) \) define random variables \( X \) and \( Y_0,Y_1,Y_2, ... \) and suppose \( \EE[X]<\infty \). Define \( F_n:=\sigma(Y_0,Y_1,...,Y_n) \) and \(X_n=\EE[X|F_n] \). Show that the sequence \(X_0,X_1,X_2,... \) is a martingale under \( \PP \) with respect to the filtration \( (F_n)_{n\geq 0}\).
\end{problem}

\begin{solution}[Solution]

Since \( (F_n)_{n\geq0} \) is a filtration, then \( F_n \) is a sub \( \sigma \)-algebra of \( F_{n+1} \). Therefore, by iterated conditioning,
\begin{align*}
    \EE[X_{n+1} | F_n] = \EE[\EE[X | F_{n+1}]| F_n] = \EE[X | F_n] = X_n
\end{align*}

This proves the sequence \( X_0, X_1, X_2, ... \) is a martingale under \( \PP \) with respect to the filtration \( (F_n)_{n\geq 0} \). \qed
\end{solution}

\begin{problem}[Exercise 2.5]
Let \( X_0,X_1,... \)be i.i.d Bernoulli random variables with parameter \( p \) (i.e., \( P(X_i= 1) =p\)). Define \( S_n=\sum_{i=1}^{n}X_i \) where \( S_0=0 \). Define
\begin{align*}
    Z_n:=\left(\dfrac{1-p}{p}\right)^{2S_n-n} && n=0,1,2,...
\end{align*}
Let \( \mathcal{F}_n:=\sigma(X_0,X_1,...,X_n) \). Show that \( Z_n \) is a martingale with respect to this filtration.
\end{problem}

\begin{solution}[Solution]
Observe,
\begin{align*}
    Z_{n+1} = \left(\dfrac{1-p}{p}\right)^{2S_{n+1}-(n+1)} = \left(\dfrac{1-p}{p}\right)^{2S_n-n}\left(\dfrac{1-p}{p}\right)^{2X_{n+1}-1} = Z_n\left(\dfrac{1-p}{p}\right)^{2X_{n+1}-1} 
\end{align*}

Then, since \( X_{n+1} \) is independent of all other \( X_j \), \( X_{n+1} \) is is independent of \( F_n \). Thus, using the definition of expectation of a discrete random variable,
\begin{align*}
    \EE\left[\left(\dfrac{1-p}{p}\right)^{2X_{n+1}-1} \Bigg| F_n \right] &= \EE\left[\left(\dfrac{1-p}{p}\right)^{2X_{n+1}-1}\right] \\
    &= (p)\left(\dfrac{1-p}{p}\right)^{2\cdot 1-1}+(1-p)\left(\dfrac{1-p}{p}\right)^{2\cdot 0-1} \\
    &= (1-p)+p \\
    &=1
\end{align*}

Therefore, by taking out what is known,
\begin{align*}
    \EE[Z_{n+1}|F_n] = \EE\left[Z_n((1-p)/p)^{2X_{n+1}-1}|F_n\right] = Z_n\EE\left[((1-p)/p)^{2X_{n+1}-1}|F_n\right] = Z_n
\end{align*}

This proves \( (Z_n)_{n\geq 0} \) is a martingale with respect to this filtration. \qed

%\textbf{ADD RIGOR this is missing the rigor of rigorous math}

\end{solution}
\end{document}
