\documentclass[11pt]{article}
\usepackage[T1]{fontenc}

% Document Details
\newcommand{\CLASS}{AMATH 586}
\newcommand{\assigmentnum}{Numerical SDE Solvers}

\usepackage[margin = 1.5in, top = 1.25in, bottom = 1.25in]{geometry}
\usepackage{titling}
\setlength{\droptitle}{-6em}   % This is your set screw
\date{}
\renewcommand{\maketitle}{
	\clearpage
	\begingroup  
	\centering
	\LARGE \sffamily\textbf{\CLASS} \Large \assigmentnum\\[.8em]
	\large Tyler Chen\\[1em]
	\endgroup
	\thispagestyle{empty}
}
 % Title Styling


\usepackage{enumitem}

% Figures
\usepackage{subcaption}

% TikZ and Graphics
\usepackage{tikz, pgfplots}
\pgfplotsset{compat=1.12}
\usetikzlibrary{patterns}
\usepgfplotslibrary{fillbetween}

\usepackage{pdfpages}
\usepackage{adjustbox}

\usepackage{lscape}
\usepackage{titling}
\usepackage[]{hyperref}


% Header Styling
\usepackage{fancyhdr}
\pagestyle{fancy}
\lhead{\sffamily \CLASS}
\rhead{\sffamily \textbf{\thepage}}
\cfoot{}

% Paragraph Styling
\setlength{\columnsep}{1cm}
\setlength{\parindent}{0pt}
\setlength{\parskip}{5pt}
\renewcommand{\baselinestretch}{1}

% TOC Styling
\usepackage{tocloft}
\iffalse
\renewcommand{\cftsecleader}{\cftdotfill{\cftdotsep}}

\renewcommand\cftchapafterpnum{\vskip6pt}
\renewcommand\cftsecafterpnum{\vskip10pt}
\renewcommand\cftsubsecafterpnum{\vskip6pt}

% Adjust sectional unit title fonts in ToC
\renewcommand{\cftchapfont}{\sffamily}
\renewcommand{\cftsecfont}{\bfseries\sffamily}
\renewcommand{\cftsecnumwidth}{2em}
\renewcommand{\cftsubsecfont}{\sffamily}
\renewcommand{\cfttoctitlefont}{\hfill\bfseries\sffamily\MakeUppercase}
\renewcommand{\cftaftertoctitle}{\hfill}

\renewcommand{\cftchappagefont}{\sffamily}
\renewcommand{\cftsecpagefont}{\bfseries\sffamily}
\renewcommand{\cftsubsecpagefont}{\sffamily}
\fi
 % General Styling
% Code Display Setup
\usepackage{listings,lstautogobble}
\usepackage{lipsum}
\usepackage{courier}
\usepackage{catchfilebetweentags}

\lstset{
	basicstyle=\small\ttfamily,
	breaklines=true, 
	frame = single,
	rangeprefix=,
	rangesuffix=,
	includerangemarker=false,
	autogobble = true
}


\usepackage{algorithmicx}
\usepackage{algpseudocode}

\newcommand{\To}{\textbf{to}~}
\newcommand{\DownTo}{\textbf{downto}~}
\renewcommand{\algorithmicdo}{\hspace{-.2em}\textbf{:}}
 % Code Display Setup
% AMS MATH Styling
\usepackage{amsmath, amssymb}
\newcommand{\qed}{\hfill\(\square\)}

%\newtheorem*{lemma}{Lemma} 
%\newtheorem*{theorem}{Theorem}
%\newtheorem*{definition}{Definition}
%\newtheorem*{prop}{Proposition}
%\renewenvironment{proof}{{\bfseries Proof.}}{}


% mathcal
\newcommand{\cA}{\ensuremath{\mathcal{A}}}
\newcommand{\cB}{\ensuremath{\mathcal{B}}}
\newcommand{\cC}{\ensuremath{\mathcal{C}}}
\newcommand{\cD}{\ensuremath{\mathcal{D}}}
\newcommand{\cE}{\ensuremath{\mathcal{E}}}
\newcommand{\cF}{\ensuremath{\mathcal{F}}}
\newcommand{\cG}{\ensuremath{\mathcal{G}}}
\newcommand{\cH}{\ensuremath{\mathcal{H}}}
\newcommand{\cI}{\ensuremath{\mathcal{I}}}
\newcommand{\cJ}{\ensuremath{\mathcal{J}}}
\newcommand{\cK}{\ensuremath{\mathcal{K}}}
\newcommand{\cL}{\ensuremath{\mathcal{L}}}
\newcommand{\cM}{\ensuremath{\mathcal{M}}}
\newcommand{\cN}{\ensuremath{\mathcal{N}}}
\newcommand{\cO}{\ensuremath{\mathcal{O}}}
\newcommand{\cP}{\ensuremath{\mathcal{P}}}
\newcommand{\cQ}{\ensuremath{\mathcal{Q}}}
\newcommand{\cR}{\ensuremath{\mathcal{R}}}
\newcommand{\cS}{\ensuremath{\mathcal{S}}}
\newcommand{\cT}{\ensuremath{\mathcal{T}}}
\newcommand{\cU}{\ensuremath{\mathcal{U}}}
\newcommand{\cV}{\ensuremath{\mathcal{V}}}
\newcommand{\cW}{\ensuremath{\mathcal{W}}}
\newcommand{\cX}{\ensuremath{\mathcal{X}}}
\newcommand{\cY}{\ensuremath{\mathcal{Y}}}
\newcommand{\cZ}{\ensuremath{\mathcal{Z}}}

% mathbb
\usepackage{bbm}
\newcommand{\bOne}{\ensuremath{\mathbbm{1}}}

\newcommand{\bA}{\ensuremath{\mathbb{A}}}
\newcommand{\bB}{\ensuremath{\mathbb{B}}}
\newcommand{\bC}{\ensuremath{\mathbb{C}}}
\newcommand{\bD}{\ensuremath{\mathbb{D}}}
\newcommand{\bE}{\ensuremath{\mathbb{E}}}
\newcommand{\bF}{\ensuremath{\mathbb{F}}}
\newcommand{\bG}{\ensuremath{\mathbb{G}}}
\newcommand{\bH}{\ensuremath{\mathbb{H}}}
\newcommand{\bI}{\ensuremath{\mathbb{I}}}
\newcommand{\bJ}{\ensuremath{\mathbb{J}}}
\newcommand{\bK}{\ensuremath{\mathbb{K}}}
\newcommand{\bL}{\ensuremath{\mathbb{L}}}
\newcommand{\bM}{\ensuremath{\mathbb{M}}}
\newcommand{\bN}{\ensuremath{\mathbb{N}}}
\newcommand{\bO}{\ensuremath{\mathbb{O}}}
\newcommand{\bP}{\ensuremath{\mathbb{P}}}
\newcommand{\bQ}{\ensuremath{\mathbb{Q}}}
\newcommand{\bR}{\ensuremath{\mathbb{R}}}
\newcommand{\bS}{\ensuremath{\mathbb{S}}}
\newcommand{\bT}{\ensuremath{\mathbb{T}}}
\newcommand{\bU}{\ensuremath{\mathbb{U}}}
\newcommand{\bV}{\ensuremath{\mathbb{V}}}
\newcommand{\bW}{\ensuremath{\mathbb{W}}}
\newcommand{\bX}{\ensuremath{\mathbb{X}}}
\newcommand{\bY}{\ensuremath{\mathbb{Y}}}
\newcommand{\bZ}{\ensuremath{\mathbb{Z}}}

% alternative mathbb
\newcommand{\NN}{\ensuremath{\mathbb{N}}}
\newcommand{\RR}{\ensuremath{\mathbb{R}}}
\newcommand{\CC}{\ensuremath{\mathbb{C}}}
\newcommand{\ZZ}{\ensuremath{\mathbb{Z}}}
\newcommand{\EE}{\ensuremath{\mathbb{E}}}
\newcommand{\PP}{\ensuremath{\mathbb{P}}}
\newcommand{\VV}{\ensuremath{\mathbb{V}}}
\newcommand{\cov}{\ensuremath{\text{Co}\VV}}
% Math Commands

\newcommand{\st}{~\big|~}
\newcommand{\stt}{\text{ st. }}
\newcommand{\ift}{\text{ if }}
\newcommand{\thent}{\text{ then }}
\newcommand{\owt}{\text{ otherwise }}

\newcommand{\norm}[1]{\left\lVert#1\right\rVert}
\newcommand{\snorm}[1]{\lVert#1\rVert}
\newcommand{\ip}[1]{\ensuremath{\left\langle #1 \right\rangle}}
\newcommand{\pp}[3][]{\frac{\partial^{#1}#2}{\partial #3^{#1}}}
\newcommand{\dd}[3][]{\frac{\d^{#1}#2}{\d #3^{#1}}}
\renewcommand{\d}{\ensuremath{\mathrm{d}}}

\newcommand{\indep}{\rotatebox[origin=c]{90}{$\models$}}




 % Math shortcuts

\newcommand{\indep}{\rotatebox[origin=c]{90}{$\models$}}

%\usepackage{algorithm}
%\usepackage[noend]{algpseudocode}

% Problem
\newenvironment{problem}[1]{\vspace{2em}{\large\sffamily\textbf{#1}}\itshape\par}{}



\begin{document}
\maketitle

\section{Introduction}
A Stochastic Differential Equation (SDE) is an equation of the form,
\begin{align}
    \d X_t =  \mu(X_t,t)\d t + \sigma(X_t,t)\d W_t \label{diffform}
\end{align}
where \( W_t \) denotes a standard Brownian Motion \cite{lorig}.

A solution to (\ref{diffform}) is an stochastic process which satisfies (\ref{diffform}). In particular, a solution can be written in integral form as,
\begin{align}
    X_{T} - X_{0} = \int_{0}^{T} \mu(t,X_t)\d t + \int_{0}^{T} \sigma(t,X_t) \d W_t \label{intform}
\end{align}

\section{Brownian Motion / It\^o Processes}
There are many characterizations of Brownian Motion. Perhaps the most standard is the following definition.
\begin{definition}
A Brownian Motion is a stochastic process \( W = (W_t)_{t\geq0} \) defined on some probability space \( (\Omega,\mF,\bP) \) satisfying,
\begin{enumerate}[nolistsep]
    \item \( W_0 = 0 \)
    \item \( (W_d-W_c) \indep (W_b-W_a) \) for \( 0\leq a\leq b\leq c\leq d \))
    \item \( (W_t - W_s) \sim \mN(0,t-s) \) for \( 0\leq s\leq t \)
    \item the map \( t \to W_t \) is continuous almost surely
\end{enumerate}
\end{definition}

An It\^o drift-diffusion process is a process of the form,
\begin{align*}
    X_T = X_t +  \int_t^T \mu(s,X_s)\d s + \int_t^T \sigma(s,X_s)\d W_s
\end{align*}


\section{Stochastic Calculus}
We first introduce Riemann--Stieltjes integrals.

\begin{definition}
For real valued functions \( f \) and \( g \) the Riemann--Stieltjes integral is defined as,
\begin{align*}
    \int_{a}^{b} f(x) \d g(x) := \lim\limits_{\norm{\Pi}\to 0} \sum_{i=0}^{n-1} f(c_i)(g(x_{i+1}) - g(x_i))
\end{align*}
where \( \Pi = \{a=x_0<x_1 < \cdots < x_n = b\} \) is a partition of \( [a,b] \), \( \norm{\Pi} \) is the length of the largest subinterval, and \( c_i \) is any point in \( [x_i,x_{i+1}] \).
\end{definition}

We note that if \( g(x) = x \) the Riemann--Stieltjes integral is the standard Riemann integral and that if \( g \) is continuously differentiable,
\begin{align*}
    f(g(T)) - f(g(t)) = \int_t^T \d f(g(s)) = \int_t^T f'(g(s))g'(s) \d s
\end{align*}

Brownian motion and many processes involving Brownian motion are not differentiable. It\^o's Lemma gives us a way to compute the analogous result for a class of stochastic processes called It\^o (drift-diffussion) processes. For our purposes we can think of It\^o processes as processes with an integral with respect to \( t \) and an integral with respect to \( W_t \).

\begin{align*}
    \d f = \pp{f}{t}\d t + \pp{f}{x}\d x + \dfrac{1}{2} \pp[2]{f}{x}\d x^2 + \cdots
\end{align*}

We can replace \( \d x \) with \( \d X_t = \mu \d t + \sigma \d W_t \) and simplify using the heuristics,
\begin{align}
    \d t \d t = 0, && \d t \d W_t = 0, && \d W_t \d W_t = \d t \label{heuristics}
\end{align}

Thus,
\begin{align*}
    \d f &= \pp{f}{t}\d t + \pp{f}{x}(\mu \d t + \sigma \d W_t) + \dfrac{1}{2}\pp[2]{f}{x} ( \mu^2 \d t^2 + \mu\sigma \d t\d W_t + \sigma^2 \d W_t ) \\
    &= \left( \pp{f}{t} + \mu \pp{f}{x} + \dfrac{\sigma^2}{2} \pp[2]{f}{x} \right)\d t + \sigma \pp{f}{x}\d W_t
\end{align*}

It\^o's Lemma can be generalized to functions and processes of higher dimension.
\begin{lemma}[It\^o] For \( f:\bR^n\to\bR^n \) sufficiently differentiable and It\^o process \( X_t = [X_t^1, X_t^2, \ldots X_t^n]^T \),
\begin{align}
    \d f(X_t) = \sum_{i=1}^{n} \left[\pp{}{x_i}f(X_t)\right] \d X_t^i + \dfrac{1}{2} \sum_{i=1}^{n} \sum_{j=1}^{n}\left[ \pp{}{x_i} \pp{}{x_j} f(X_t) \right] \d [X^i,X^j]_t \label{itond}
\end{align}
\end{lemma}

Similar to before, we compute \( \d[X^i,X^j]_t \) by expanding \( (\d X^i)(\d X^j) \) and using the heuristics in (\ref{heuristics}).

Consider the special case when \( n=2 \), \( X_t^1 = t\), and \( X_t^2 = X_t \). Using our heuristics in (\ref{heuristics}) we have \( \d[X^1,X^2]_t = (\d t)(\d X_t) = 0 \) and \( \d[X^1,X^1] = (\d t)(\d t) = 0 \). Therefore, by (\ref{itond}),
\begin{align}
    \d f(t,X_t) &= \pp{}{t} f(t,X_t) + \pp{}{x} f(t,X_t) \d X_t + \dfrac{1}{2} \pp[2]{}{x}f(t,X_t) \d [X,X]_t
    \\ &= \left( \pp{}{t} + \dfrac{1}{2}\pp[2]{}{x} \right)f(t,X_t)\d t + \pp{}{x} f(t,X_t) \d X_t \label{2dito}
\end{align}

\section{Example Processes}
In this section we provide some results about a few important stochastic processes. Proofs of the results presented here are readily available on the internet.

For constants \( \theta, \mu, \sigma \), an Ornstein--Uhlenbeck (OU) process satisfies,
\begin{align}
    \d X_t = \theta(\mu-X_t) \d t + \sigma \d W_t, && \theta > 0 \label{OUeqn}
\end{align}

We note that if \( X_t \) is away from \( \mu \) it will tend towards this value in expectation. Figure~\ref{OU_em} shows Euler--Maruyama method applied to (\ref{OUeqn}). As expected, the trajectories all end up ``centered'' about \( \mu \). More precisely, (\ref{OUeqn}) has solution,
\begin{align*}
    X_t = X_0 \exp(-\theta t) + \mu(1-\exp(-\theta t)) + \sigma \int_0^t \exp(-\theta(t-s))\d W_s
\end{align*}

The mean of \( X_t \) is,
\begin{align*}
    \bE[X_t] = (X_0 - \mu)\exp \left( -\theta t \right)+ \mu
\end{align*}
Likewise, the variance of \( X_t \) is,
\begin{align*}
    \bE\left[(X_t-\bE[X_t])^2\right] = \dfrac{\sigma^2}{2\theta}(1-\exp(-2\theta t))
\end{align*}


For constants \( \mu \) and \( \sigma \) a Geometric Brownian Motion satisfies,
\begin{align}
    \d X_t = \mu X_t \d t + \sigma X_t \d W_t, && t\in[0,T] \label{GBMeqn}
\end{align}

The solution to (\ref{GBMeqn}) is,
\begin{align*}
    X_t = X_0 \exp\left( \left( \mu - \dfrac{\sigma^2}{2} \right)t + W_t \right)
\end{align*}

\bibliography{../project}{}
\bibliographystyle{amsplain}


\end{document}
